\section{The fully relative cap product}

\v{C}ech cohomology appeared as the natural algebra acting on $H^*(X,X-K)$,
where $K$ is a closed subspace of $X$: 
\[
\cap:\cHH^p(K)\otimes H_n(X,X-K)\to H_q(X,X-K)\,,\quad p+q=n\,.
\]
If we fix $x_K\in H_n(X,X-K)$, then capping with $x_K$ gives a map 
\[
\cap x_K:\cHH^p(K)\to H_q(X,X-K)\,,\quad p+q=n\,. 
\]
We will be very interested in showing that this map is an isomorphism 
under certain conditions. This is a kind of duality result, comparing
cohomology and relative homology and reversing the dimensions. 
We'll try to show that such a map is an isomorphism by embedding it 
in a map of long exact sequences and using the five-lemma. 

For a start, let's think about how these maps vary as we change $K$.
So let $L$ be a closed subset of $K$, so $X-K\subseteq X-L$ and we get 
a ``restriction map''
\[
i_\ast:H_n(X,X-K)\to H_n(X,X-L)\,.
\]
Define $x_L$ as the image of $x_K$. The diagram
\begin{equation*}
\xymatrix{
\cHH^p(K)\ar[r]\ar[d]_{-\cap x_K} & \cHH^p(L) \ar[d]_{-\cap x_L} \\
H_q(X,X-K)\ar[r] & H_q(X,X-L)
}
\end{equation*}
commutes by the projection formula. This embeds into a ladder
shown in the theorem below. We will accompany this ladder with a second
one, to complete the picture.
\begin{theorem}
There is a ``fully relative'' cap product
\[
\cap:\cHH^p(K,L)\otimes H_n(X,X-K)\to H_q(X-L,X-K)\,,\quad p+q=n\,,
\]
such that for any $x_K\in H_n(X,X-K)$ the ladder
\begin{equation*}
\xymatrix{
\cdots \ar[r] & \cHH^p(K,L)\ar[r]\ar[d]^{\cap x_{K}} & \cHH^p(K)\ar[r]\ar[d]^{\cap x_K} & \cHH^p(L)\ar[r]^\delta\ar[d]^{\cap x_L} & \cHH^{p+1}(K,L)\ar[r]\ar[d]^{\cap x_{K}} & \cdots\\
	\cdots\ar[r] & H_q(X-L,X-K)\ar[r]& H_q(X,X-K)\ar[r] & H_q(X,X-L)\ar[r]^\partial & H_{q-1}(X-L,X-K)\ar[r] & \cdots
}
\end{equation*}
commutes, where $x_L$ is the restriction of $x_K$ to $H_n(X,X-L)$, 
and for any $x\in H_n(X)$
\[
\xymatrix{
\cdots \ar[r] & \cHH^p(X,K) \ar[d]^{\cap x} \ar[r] & 
\cHH^p(X,L) \ar[d]^{\cap x} \ar[r] & 
\cHH^p(K,L) \ar[d]^{\cap x_K} \ar[r]^\delta &
\cHH^{p+1}(X,K) \ar[d]^{\cap x} \ar[r] & \cdots\\
\cdots \ar[r] & H_q(X-K) \ar[r] & H_q(X-L) \ar[r] & 
H_q(X-L,X-K) \ar[r]^\partial & H_{q-1}(X-K) \ar[r] & \cdots
}\]
commutes, where $x_K$ is the restriction of $x$ to $H_n(X,X-K)$. 
\end{theorem}
\begin{proof}
What I have to do is define a cap product along the bottom row of the
diagram (with $p+q=n$)
\begin{equation*}
\xymatrix{
\cHH^p(K)\otimes H_n(X,X-K)\ar[r]^{\cap} & H_q(X,X-K)\\
\cHH^p(K,L)\otimes H_n(X,X-K)\ar[u]\ar@{.>}[r]^{\cap} & H_q(X-L,X-K) \ar[u]
}
\end{equation*}

This requires going back to the origin of the cap product. 
Our map $\cHH^p(K)\otimes H_n(X,X-K)\to H_q(X,X-K)$ came (via excision)
 from a chain map $S^p(U)\otimes S_n(U,U-K)\to S_q(U,U-K)$ where $U\supseteq K$, defined by $\beta\otimes\sigma\mapsto\beta(\sigma\circ\alpha_p)\cdot(\sigma\circ\omega_q)$. Now given inclusions
\[
\begin{array}{ccc} L & \subseteq & K \\
\downsubseteq & & \downsubseteq \\
V & \subseteq & U
\end{array}
\]
we can certainly fill in the bottom row of the diagram
\begin{equation*}
\xymatrix{
S^p(U)\otimes S_n(U)/S_n(U-K)\ar[r] & S_q(U)/S_q(U-K)\\
S^p(U,V)\otimes S_n(U)/S_n(U-K)\ar[r]\ar[u] & S_q(U-L)/S_q(U-K)\ar[u]
}
\end{equation*}
Since cochains in $S^p(U,V)$ kill chains in $V$, we can extend the bottom
row to 
\begin{equation*}
\xymatrix{
	S^p(U)\otimes S_n(U,U-K)\ar[r] & S_q(U,U-K)\\
	S^p(U,V)\otimes (S_n(U-L)+S_n(V))/S_n(U-K)\ar[r]\ar[d]^{\simeq}\ar[u] & S_q(U-L)/S_q(U-K)\ar[u]\\
	S^p(U,V)\otimes S_n(U)/S_n(U-K) & 
}
\end{equation*}
But $L\subseteq V$, so $(U-L)\cup V=U$, and the locality principle then
guarantees that $S_n(U-L)+S_n(V)\to S_n(U)$ is a quasi-isomorphism. 
By excision, $H_n(U,U-K)\to H_n(X,X-K)$ is an isomorphism. Now use
our standard map $\mu:H_*(C)\otimes H_*(D)\to H_*(C\otimes D)$. 

This gives the construction of the fully relative cap product. 
We leave the checks of commutativity to the listener.
\end{proof}

The diagram
\begin{equation*}
\xymatrix{
	\cHH^p(L)\ar[r]^\delta\ar[d]^{-\cap x_L} & \cHH^{p+1}(K,L)\ar[d]^{-\cap x_K}\\
	H_q(X,X-L)\ar[r]^{\partial} & H_{q-1}(X-L,X-K)
}
\end{equation*}
provides us with the memorable formula
\begin{equation*}
(\delta b)\cap x_K=\partial(b\cap x_L)\,.
\end{equation*}

The construction of the Mayer-Vietoris sequences now gives:
\begin{theorem}
\label{thm-mayer-vietoris-ladder}
Let $A, B$ be closed in a normal space or compact in a Hausdorff space. 
The \v{C}ech cohomology and singular homology Mayer-Vietoris sequences are 
compatible: for any $x_{A\cup B}\in H_n(X,X-A\cup B)$, 
there is a commutative ladder (where again we use the notation
$H_q(X|A)=H_q(X,X-A)$, and again $p+q=n$)
\begin{equation*}
\xymatrix{
\cdots \ar[r] & \cHH^p(A\cup B)\ar[r]\ar[d]^{\cap x_{A\cup B}} & 
\cHH^p(A)\oplus \cHH^p(B)\ar[r]\ar[d]^{(\cap x_A)\oplus(\cap x_B)} & 
\cHH^p(A\cap B)\ar[r]\ar[d]^{\cap x_{A\cap B}} & 
\cHH^{p+1}(A\cup B)\ar[r]\ar[d]^{\cap x_{A\cup B}} & \cdots\\
\cdots \ar[r] & H_q(X|A\cup B)\ar[r]& H_q(X|A)\oplus H_q(X|B)\ar[r]& H_q(X|A\cap B)\ar[r]& H_{q-1}(X|A\cup B)\ar[r]&\cdots
}
\end{equation*}
in which the  homology classes $x_A,x_B,x_{A\cap B}$  
are restrictions of the class $x_{A\cup B}$ in the diagram
\begin{equation*}
\xymatrix{
	 & H_n(X,X-A)\ar[dr] & \\
	H_n(X,X-A\cup B)\ar[ur]\ar[dr] & & H_n(X,X-A\cap B)\\
	 & H_n(X,X-B)\ar[ur]
}
\end{equation*}
\end{theorem}

