\section{K\"{u}nneth and Eilenberg-Zilber}
We want to compute the homology of a product. Long ago, in Lecture 7, we constructed a bilinear map $S_p(X)\times S_q(Y)\to S_{p+q}(X\times Y)$, called the cross product. So we get a linear map $S_p(X)\otimes S_q(Y)\to S_{p+q}(X\times Y)$, and it satisfies the Leibniz formula, i.e., $d(x\times y)=dx\times y+(-1)^px\times dy$. The method we used works with any coefficient ring, not just the integers. 
\begin{definition}
Let $C_\ast,D_\ast$ be two chain complexes. Their {\em tensor product} is 
the chain complex with
\[
(C_\ast\otimes D_\ast)_n=\bigoplus_{p+q=n}C_p\otimes D_q\,.
\]
The differential $(C_\ast\otimes D_\ast)_n\to (C_\ast\otimes D_\ast)_{n-1}$ sends $C_p\otimes D_q$ into the submodule $C_{p-1}\otimes D_q\bigoplus C_p\otimes D_{q-1}$ by 
\[
x\otimes y\mapsto dx\otimes y+(-1)^p x\otimes dy\,.
\]
\end{definition}
So the cross product is a map of chain complexes $S_\ast(X)\otimes S_\ast(Y)\to S_\ast(X\times Y)$. There are two questions:\\
(1) Is this map an isomorphism in homology? \\
(2) How is the homology of a tensor product of chain complexes related to the
tensor product of their homologies? 

It's easy to see what happens in dimension zero, because 
$\pi_0(X)\times \pi_0(Y)=\pi_0(X\times Y)$
implies that 
$H_0(X)\otimes H_0(Y)\xrightarrow{\cong}H_0(X\times Y)$.

Let's dispose of the purely algebraic question (2) first. 

\begin{theorem}
Let $R$ be a PID and $C_\ast,D_\ast$ be chain complexes of $R$-modules. 
Assume that $C_n$ is a free $R$-module for all $n$. There is a short
exact sequence
\begin{equation*}
0\to\bigoplus_{p+q=n} H_p(C)\otimes H_q(D)\to H_n(C_\ast\otimes D_\ast)
\to \bigoplus_{p+q=n-1}\Tor^R_1( H_p(C), H_q(D))\to 0
\end{equation*}
natural in these data, that splits (but not naturally).
\end{theorem}
\begin{proof}
This is exactly the same as the proof for the UCT. It's a good idea to work through this on your own.
\end{proof}

\begin{corollary}\label{acyclic-tensor}
Let $R$ be a PID and assume $C'_n$ and $C_n$ are $R$ free for all $n$.
If $C_*'\to C_\ast$ and $D_*'\to D_\ast$ 
are homology
isomorphisms then so is $C_*'\otimes D_*'\to C_\ast\otimes D_\ast$. 
\end{corollary}

Our attack on question (1) is via the method of ``acyclic models.'' This is
really a special case of the Fundamental Theorem of Homological Algebra,
Theorem \ref{thm-ftha}.

\begin{definition}
Let $\cc$ be a category, and fix a set $\MM$ of objects in $\cc$, 
to be called the ``models.'' 
A functor $F:\cc\to\Ab$ is $\MM$-{\em free} if it is the free abelian 
group generated by a coproduct of corepresentable functors. That is, $F$ is a 
direct sum of functors of the form $\Z\cc(M,-)$ where $M\in\MM$.
\end{definition}
\begin{example}
Since we are interested in the singular homology of a product of two spaces, 
it may be sensible to take as $\cc$ the category of ordered pairs of spaces,
$\cc=\mathbf{Top}^2$, and for $\MM$ the set of pairs of
simplicies, $\MM=\{(\Delta^p,\Delta^q):p,q\geq0\}$. Then 
\[
S_n(X\times Y)=\Z[\mathbf{Top}(\Delta^n\times X)\times
\mathbf{Top}(\Delta^n,Y)]=
\Z\mathbf{Top}^2((\Delta^n,\Delta^n),(X,Y))\,.
\]
is $\MM$-free.
\end{example}
\begin{example}
With the same category and models, 
\[
(S_*(X)\otimes S_*(Y))_n=\bigoplus_{p+q=n}S_p(X)\otimes S_q(Y)\,,
\]
is $\MM$-free, since the tensor product has as free basis the set
\[
\coprod_{p+q=n}\mathrm{Sin}_p(X)\times\mathrm{Sin}_q(Y)
=\coprod_{p+q=n}\mathbf{Top}^2((\Delta^p,\Delta^q),(X,Y))\,.
\]
\end{example}
\begin{definition}
A natural transformation of functors $\theta:F\to G$ is an $\MM$-{\em epimorphism} if $\theta_M:F(M)\to G(M)$ is a surjection of abelian groups for every $M\in\MM$. A {\em sequence} of natural transformations is a composable pair 
$G^\prime\to G\to G^{\prime\prime}$ with trivial composition. 
Let $K$ be the objectwise kernel of $G\to G^{\prime\prime}$. There is a factorization $G^\prime\to K$. The sequence is $\MM$-{\em exact} if $G^\prime\to K$ is a $\MM$-epimorphism. Equivalently, $G^\prime(M)\to G(M)\to G^{\prime\prime}(M)$ is exact for all $M\in\MM$.
\end{definition}
\begin{example}
We claim that 
\[
\cdots\to S_n(X\times Y)\to S_{n-1}(X\times Y)\to\cdots\to S_0(X\times Y)\to H_0(X\times Y)\to 0
\]
is $\MM$-exact. Just plug in $(\Delta^p,\Delta^q)$: you get an exact sequence, since $\Delta^p\times\Delta^q$ is contractible.
\end{example}
\begin{example}
The sequence
\[
\cdots\to(S_\ast(X)\otimes S_\ast(Y))_n\to(S_\ast(X)\otimes S_\ast(Y))_{n-1}\to
\cdots\to S_0(X)\otimes S_0(Y)\to H_0(X)\otimes H_0(Y)\to 0\,.
\]
is also $\MM$-exact, by Corollary \ref{acyclic-tensor}.
\end{example}
The terms ``$\MM$-free'' and ``$\MM$-exact'' relate to each other in the 
expected way:
\begin{lemma}
Let $\cc$ be a category with a set of models $\MM$ and let $F,G,G^\prime:\cc\to\mathbf{Ab}$ be functors. Suppose that $F$ is $\MM$-free, let $G^\prime\to G$ be a $\MM$-epimorphism, and let $f:F\to G$ be any natural transformation.  
Then there is a lifting:
\begin{equation*}
\xymatrix{ & G^\prime\ar[d]\\
F\ar@{-->}[ur]^{\overline{f}}\ar[r]^f & G}
\end{equation*}
\end{lemma}
\begin{proof}
Clearly we may assume that $F(X)=\Z\cc(M,X)$. Suppose that $X=M\in\MM$. 
We get:
\begin{equation*}
\xymatrix{ & G^\prime(M)\ar@{->>}[d]\\
\Z\cc(M,M)\ar@{-->}[ur]^{\overline{f}_M}\ar[r]^{f_M} & G(M)}
\end{equation*}
Consider $1_M\in\Z\cc(M,M)$. Its image  $f_M(1_M)\in G(M)$ is hit by
some element in $c_M\in G'(M)$, since $G'\to G$ is an $\MM$-epimorphism. Define
$\overline f_M(1_M)=c_M$.

Now we exploit naturality! Any $\varphi:M\to X$ produces a commutative 
diagram
\begin{equation*}
\xymatrix{\cc(M,M)\ar[r]^{\overline{f}_M} \ar[d]^{\varphi_*}
& G^\prime(M)\ar[d]^{\varphi_*} \\
\cc(M,X)\ar[r]^{\overline{f}_X} & G^\prime(X)}
\end{equation*}
Chase $1_M$ around the diagram, to see what the value of 
$\overline{f}_X(\varphi)$ must be:
\[
\overline{f}_X(\varphi)=\overline{f}_X(\varphi_*(1_M))=
\varphi_*(\overline{f}_M(1_M))=\varphi_*(c_M)\,.
\]
Now extend linearly. You should check that this does define a natural
transformation. 
\end{proof}

This is precisely the condition required to prove the Fundamental Theorem
of Homological Algebra. So we have the 
\begin{theorem}[Acyclic Models]
Let $\MM$ be a set of models in a category $\cc$. Let $\theta:F\to G$ be 
a natural transformation of functors from $\cc$ to $\Ab$. Let 
$F_*$ and $G_*$ be functors from $\cc$ to chain complexes, with augmentations
$F_0\to F$ and $G_0\to G$. Assume that $F_n$ is $\MM$-free for all $n$, and
that $G_*\to G\to0$ is an $\MM$-exact sequence. Then there is a unique
chain homotopy class of chain maps $F_*\to G_*$ covering $\theta$. 
\end{theorem}
\begin{corollary}
Suppose furthermore that $\theta$ is a natural isomorphism. If
each $G_n$ is $\MM$-free and $F_*\to F\to0$ is an $\MM$-exact
sequence, then any natural chain map $F_*\to G_*$ covering $\theta$ is  
a natural chain homotopy equivalence. 
\end{corollary}

Applying this to our category $\Top^2$ with models as before, we get 
the following theorem that completes work we did in Lecture 7.
\begin{theorem}[Eilenberg-Zilber theorem]
There are unique chain homotopy classes of natural chain maps:
\begin{equation*}
S_\ast(X)\otimes S_\ast(Y)\leftrightarrows S_\ast(X\times Y)
\end{equation*}
covering the usual isomorphism
\[
H_0(X)\otimes H_0(Y)\cong H_0(X\times Y)\,,
\]
and they are natural chain homotopy inverses. 
\end{theorem}
\begin{corollary}
There is a canonical natural 
isomorphism $H(S_\ast(X)\otimes S_\ast(Y))\cong H_\ast(X\times Y)$.
\end{corollary}

Combining this theorem with the algebraic K\"unneth theorem, we get:
\begin{theorem}[K\"{u}nneth theorem]
Take coefficients in a PID $R$. There is a short exact sequence 
\begin{equation*}
0\to\bigoplus_{p+q=n} H_p(X)\otimes_RH_q(Y)\to
 H_n(X\times Y)\to\bigoplus_{p+q=n-1}\Tor^R_1( H_p(X), H_q(Y))\to 0
\end{equation*}
natural in $X$, $Y$. It splits as $R$-modules, but not naturally.
\end{theorem}
\begin{example}
If $R=k$ is a field, every module is free, so the $\Tor$ term vanishes, and you get a K\"{u}nneth {\em isomorphism}:
\begin{equation*}
\times: H_\ast(X;k)\otimes_k H_\ast(Y;k)\xrightarrow{\cong}H_\ast(X\times Y;k)
\end{equation*}
\end{example}
This is rather spectacular. For example, what is $ H_\ast(\RP^3\times\RP^3;k)$,
where $k$ is a field? Well, if $k$ has characteristic different from 2, 
$\RP^3$ has the same homology as $S^3$, so the product has the same 
homology as $S^3\times S^3$: the dimensions are $1,0,0,2,0,0,1$. 
If $\mathrm{char}\,k=2$, on the other hand, the cohomology modules are either
0 or $k$, and we need to form the graded tensor product:
\[
\begin{array}{cccc} k & k & k & k \\ k & k & k & k \\ 
k & k & k & k \\ k & k & k & k 
\end{array}
\]
so the dimensions of the homology of the product are 
$1,2,3,4,3,2,1$. 

The palindromic character of this sequence will be explained by Poincar\'e 
duality. Let's look also at what happens over the integers. Then we have 
the table of tensor products
\[
\begin{array}{c|cccc}
& \Z & \Z/2\Z & 0 & \Z \\
\hline
\Z & \Z & \Z/2\Z & 0 & \Z \\
\Z/2\Z & \Z/2\Z & \Z/2\Z & 0 & \Z/2\Z \\
0 & 0 & 0 & 0 & 0 \\
\Z & \Z & \Z/2\Z & 0 & \Z 
\end{array}
\]
There is only one nonzero $\Tor$ group, namely 
\[
\Tor_1^\Z(H_1(\RP^3),H_1(\RP^3))=\Z/2\Z.
\]
Putting this together, we get the groups
\[
\begin{array}{c|c}
H_0 & \Z \\
H_1 & \Z/2\Z\oplus\Z/2\Z\\
H_2 & \Z/2\Z \\
H_3 & \Z \oplus \Z \oplus\Z/2\Z \\
H_4 & \Z/2\Z\oplus\Z/2\Z\\
H_5 & 0 \\
H_6 & \Z
\end{array}
\]
The failure of perfect symmetry here is interesting, and will also be explained
by Poincar\'e duality. 






