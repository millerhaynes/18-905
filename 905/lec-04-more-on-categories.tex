\section{Categorical language}

Let $\mathrm{Vect}_k$ be the category of vector spaces over a field $k$, and linear transformations between them. Given a vector space $V$, you can consider the dual $V^\ast=\Hom(V,k)$. Does this give us a functor? If you have a linear transformation $f:V\to W$, you get a map $f^\ast:W^\ast\to V^\ast$, so this is like a functor, but the induced map goes the wrong way. This operation does preserve composition and identities, in an appropriate sense. This is an example of a {\em contravariant functor}. 

I'll leave it to you to spell out the definition, but notice that there is a univeral example of a contravariant functor out of a category $\cc$: $\cc\to\cc^{op}$, where $\cc^{op}$ has the same objects as $\cc$, but $\cc^{op}(X,Y)$ is declared to be the set $\cc(Y,X)$. The identity morphisms remain the same. To describe the composition in $\cc^{op}$, I'll write $f^{op}$ for $f\in\cc(Y,X)$ regarded as an element of $\cc^{op}(X,Y)$; then $f^{op}\circ g^{op}=(g\circ f)^{op}$. 

Then a contravariant functor from $\cc$ to $\cd$ is the same thing as a (``covariant'') functor from $\cc^{op}$ to $\cd$. 

Let $\cc$ be a category, and let $Y\in\mathrm{ob}(\cc)$. We get a map $\cc^{op}\to\mathbf{Set}$ that takes $X\mapsto \cc(X,Y)$, and takes a map $X\to W$ to the map defined by composition $\cc(W,Y)\to \cc(X,Y)$. This is called the functor {\em represented by} $Y$. It is very important to note that $\cc(-,Y)$ is contravariant, while, on the other hand, for any fixed $X$, $\cc(X,-)$ is a covariant functor.

\begin{example}
Recall that the simplex category $\Deltab$ has objects the totally ordered sets
$[n]=\{0,1,\ldots,n\}$, with order preserving maps as morphisms. The ``standard simplex'' gives us a functor $\Delta\colon\Deltab^{op}\to\mathbf{Top}$. Now fix a space $X$, and consider 
\[
[n]\mapsto\mathbf{Top}(\Delta^n,X)\,.
\]
This gives us a contravariant functor $\Deltab\to\mathbf{Top}$, or a covariant
functor $\Deltab^{op}\to\mathbf{Top}$. This functor carries in it all the face and degeneracy maps we discussed earlier, and their compositions. Let us make a definition.
\end{example}

\begin{definition} Let $\cc$ be any category. A {\em simplicial object} in $\cc$ is a functor $K:\Deltab^{op}\to\cc$. Simplicial objects in $\cc$ form a category with natural transformations as morphisms. Similarly, {\em semi-simplicial object} in $\cc$ is a functor $\Deltab_{inj}^{op}\to\cc$,
\end{definition}

So the singular functor $\Sin_\ast$ gives a functor from spaces to simplicial sets (and so, by restriction, to semi-simplicial sets). 

I want to interject one more bit of categorical language that will often be useful to us. 

\begin{definition}
A morphism $f:X\to Y$ in a category $\cc$ is a \textit{split epimorphism} (``split epi'' for short) if there exists $g:Y\to X$ (called a section or a splitting) such that the composite $Y\xrightarrow{g}X\xrightarrow{f}Y$ is the identity.
\end{definition}
\begin{example}
In the category of sets, a map $f:X\to Y$ is a split epimorphism exactly when, 
for every element of $Y$ there exists some element of $X$ whose image in $Y$ is the original element. So $f$ is surjective. Is every surjective map a split epimorphism? This is equivalent to the axiom of choice! because a section of $f$ is precisely a choice of $x\in f^{-1}(y)$ for every $y\in Y$.
\end{example}
Every categorical definition is accompanied by a dual definition. 
\begin{definition}
A map $g:Y\to X$ is a {\em split monomorphism} (``split mono'' for short) if there is $f:X\to Y$ such that $f\circ g=1_Y$.
\end{definition}
\begin{example}
Again let $\cc=\Set$. Any split monomorphism is an injection: If $y,y^\prime\in Y$, and $g(y)=g(y^\prime)$, we want to show that $y=y^\prime$. Apply $f$, to get $f(g(y))=y=f(g(y^\prime))=y^\prime$. But the injection $\varnothing\to Y$ 
is a split monomorphism only if $Y=\varnothing$. So there's an assymetry 
in the category of sets.
\end{example}
\begin{lemma}
A map is an isomorphism if and only if it is both a split epimorphism and a
split monomorphism.
\end{lemma}
\begin{proof}
Easy!
\end{proof}
The importance of these definitions is this: Functors will not in general
respect ``monomorphisms'' or ``epimorphisms,'' but:
\begin{lemma}
Any functor sends split epis to split epis and split monos to split monos.
\end{lemma}
\begin{proof}
Apply $F$ to the diagram establishing $f$ as a split epi or mono.
\end{proof}
\begin{example}
Suppose $\cc=\mathbf{Ab}$, and you have a split epi $f:A\to B$. Let $g:B\to A$ be a section. We also have the inclusion $i:\ker f\to A$, and hence a map
\[
[\,g\quad i\,]:B\oplus\ker f\to A\,.
\]
I leave it to you to check that this map is an isomorphism, and to formulate a
dual statement.
\end{example}
