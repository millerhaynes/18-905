\section{Euler characteristic, and homology approximation}


\begin{theorem} Let $X$ be a finite CW-complex with $a_n$ $n$-cells. Then 
\[
\chi(X)=\sum^\infty_{k=0}(-1)^k a_k
\]
depends only on the homotopy type of $X$; it is independent of the choice of
CW structure. 
\end{theorem}
This integer $\chi(X)$ is called the {\em Euler characteristic} of $X$. We will prove this theorem by showing that $\chi(X)$ equals a number computed from the homology groups of $X$, which are themselves homotopy invariants. 

We'll need a little bit of information about the structure of finitely generated abelian groups.

Let $A$ be an abelian group. The set of {\em torsion} elements of $A$,
\[
\mathrm{Tors}(A)=\{a\in A:na=0\,\,\text{for some}\,\,n\neq0\}\,,
\]
is a subgroup of $A$. A group is \emph{torsion free}
if $\mathrm{Tors}(A)=0$. For any $A$ the quotient group 
$A/\mathrm{Tors}(A)$ is torsion free. 

For a general abelian group, that's about all you can say. But now assume $A$ is finitely generated. Then $\mathrm{Tors}(A)$ is a finite abelian group and
$A/\mathrm{Tors}(A)$ is a finitely generated free abelian group, isomorphic to $\Z^r$ for some integer $r$ called the \emph{rank} of $A$. Pick elements of
$A$ that map to a set of generators of $A/\mathrm{Tors}(A)$, and use them
to define a map $A/\mathrm{Tors}A\to A$ splitting the projection map. This shows that if $A$ is finitely generated then
\[
A\cong\mathrm{Tors}(A)\oplus\Z^r\,.
\]

A finite abelian group $A$ is necessarily of the form 
\[
\Z/n_1\oplus\Z/n_2\oplus\cdots\oplus\Z/n_t\,\,\hbox{where}\,\,
n_1|n_2|\cdots|n_t\,.
\]
The $n_i$ are the ``torsion coefficients'' of $A$. 
They are well defined natural numbers.

\begin{lemma} Let $0\to A\to B\to C\to 0$ be a short exact sequence of finitely
generated abelian groups. Then
\[
\mathrm{rank}\,A-\mathrm{rank}\,B+\mathrm{rank}\,C=0\,.
\]
\end{lemma}

\begin{theorem} Let $X$ be a finite CW complex. Then
\[
\chi(X)=\sum_k(-1)^k\mathrm{rank}\,H_k(X)\,.
\]
\end{theorem}

\begin{proof}
Pick a CW-structure with, say, $a_k$ $k$-cells for each $k$. We have the 
cellular chain complex $C_*$. Write $H_*,Z_*$, and $B_*$ for the homology,
the cycles, and the boundaries, in this chain complex. From the definitions, 
we have two families of short exact sequences: 
\[
0\to Z_k\to C_k\to B_{k-1}\to 0
\]
and
\[
0\to B_k\to Z_k\to H_k\to 0\,.
\]
Let's use them and facts about rank rewrite the alternating sum:
\begin{align*}
\sum_k (-1)^ka_k & = \sum_k(-1)^k\mathrm{rank}(C_k)\\
& = \sum_k(-1)^k(\mathrm{rank}\,(Z_k)+\mathrm{rank}\,(B_{k-1}))\\
& = \sum_k(-1)^k(\mathrm{rank}\,(B_k)+\mathrm{rank}\,(H_k)+
\mathrm{rank}\,(B_{k-1}))
\end{align*}
The terms $\mathrm{rank}\,B_k\,+\,\mathrm{rank}\,B_{k-1}$ cancel because it's an alternating sum. This leaves $\sum_k(-1)^k\mathrm{rank}\,H_k$. But 
$H_k\cong H_k^\text{sing}(X)$.
\end{proof}

In the early part of the 20th century, ``homology groups'' were not discussed. 
It was Emmy Noether who first described things that way. Instead, 
people worked mainly with the sequence of ranks, 
\[
\beta_k=\mathrm{rank}\,H_k(X)\,,
\]
which are known (following Poincar\'e) as the {\em Betti numbers} of $X$.

Given a CW-complex $X$ of finite type, can we give a lower bound on the number of $k$-cells in terms of the homology of $X$? Let's see. $H_k(X)$ is finitely generated because $C_k(X)\hookleftarrow Z_k(X)\twoheadrightarrow H_k(X)$. Thus 
\[
H_k(X)=\bigoplus^{t(k)}_{i=1}\Z/n_i(k)\Z\oplus \Z^{r(k)}
\]
where the $n_1(k)|\cdots|n_{t(k)}(k)$ are the torsion coefficients of $H_k(X)$ 
and $r(k)$ is the rank.

The minimal chain complex with $H_k=\Z^r$ and $H_q=0$ for $q\neq k$ is just the chain complex with $0$ everywhere else except for $\Z^r$ in the $k$th degree. The minimal chain complex {\em of free abelian groups} with $ H_k=\Z/n\Z$ and $ H_q=0$ for $q\neq k$ is the chain complex with $0$ everywhere else except in dimensions $k+1$ and $k$, where we have $\Z\xrightarrow{n}\Z$ 
These small complexes are called {\em elementary chain complexes}.

This implies that
a lower bound on the minimal number of $k$-cells is 
\[
r(k)+t(k)+t(k-1)\,.
\]
The first two terms give generators for $H_k$, and the last gives relations
for $H_{k-1}$.

These elementary chain complexes can be realized as the reduced cellular chains of CW complexes (at least if $k>0$). A wedge of $r$ copies of $S^k$ has reduced cellular chains $\Z^r$ in dimension $k$ and 0 in other dimensions.
(We are dividing out by the acyclic subcomplex generated by the $0$ and 
$-1$ dimensional generators.)
To construct a CW complex with reduced chains $\Z\xrightarrow{n}\Z$ in dimensions $k+1$ and $k$, start with $S^k$ as $k$-skeleton and attach a $k+1$-cell
by a map of degree $n$.
For example, when $k=1$ and $n=2$, you have $\RP^2$. These CW complexes are
called ``Moore spaces.'' 

This maximally efficient construction of a CW complex in a homotopy type
can in fact be achieved:

\begin{theorem}[Wall, \cite{wall}]
Let $X$ be a simply connected CW-complex of finite type. Then there exists a CW complex $Y$ with $r(k)+t(k)+t(k-1)$ $k$-cells, for all $k$, and a homotopy equivalence $Y\to X$.
\end{theorem}

We will prove this theorem in 18.906.

The construction of Moore spaces can be generalized:
\begin{prop} For any graded abelian group $A_*$ with $A_k=0$ for $k\leq0$, 
there exists a CW complex $X$ with $\widetilde H_*(X)=A_*$.
\end{prop}
\begin{proof}
Let $A$ be any abelian group. Pick generators for $A$. 
They determine a surjection
from a free abelian group $F_0$. The kernel of that surjection is free,
being a subgroup of a free abelian group. Write $G_0$ for minimal set of
generators of $F_0$, and $G_1$ for a minimal set of generators for $F_1$.

Let $k\geq1$. Define $X_k$ to be the wedge of $|G_0|$ copies of $S^k$,
so $H_k(X_k)=\Z\langle G_0\rangle$. Now define an attaching map 
\[
\alpha:\coprod_{b\in G_1} S^k_b\to X_k
\]
by specifying it on each summand $S^k_b$. The generator 
$b\in G_1$ is given by a linear combination of the generators
of $F_0$, say 
\[
b=\sum_{i=1}^s n_ia_i\,.
\]
We want to mimic this in topology. To do this, first
map $S^k\to\bigvee^s S^k$ by pinching $(s-1)$ tangent circles to points. 
In homology, this map takes a generator of $H_k(S^k)$ to the sum of the
generators of the $k$-dimensional homology of the various spheres in the
bouquet.
Map the $i$th sphere in the wedge to $S^k_{a_i}\subseteq X_k$ by a map of
degree $n_i$. The map on the summand $S^k_b$ is then the composite of these 
two maps,
\[
S^k_b\to\bigvee_{i=1}^s S^k\to\bigvee_a S^k_a\,.
\]
Altogether, we get a map $\alpha$ that realizes $F_1\to F_0$ in $H_k$.
So using it as an attaching map produces a CW complex $X$ with 
$\widetilde H_q(X)=A$ for $q=k$ and 0 otherwise. Write $M(A,k)$ for 
a CW complex pruduced in this way.

Finally, given a graded abelian group $A_*$, for the wedge over $k$ of the
spaces $M(A_k,k)$. 
\end{proof}

Such a space $M(A,k)$, with $\widetilde H_q(M(A,k))=A$ for $q=k$ and 0 
otherwise, is called a {\em Moore space of type} $(A,k)$ \cite{moore}. 
The notation is a bit deceptive, since $M(A,k)$ cannot be made into a functor
$\mathbf{Ab}\to\mathrm{Ho}\mathbf{Top}$.


