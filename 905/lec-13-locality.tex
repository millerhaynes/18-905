\section{Proof of the Locality Principle}

We have constructed the subdivision operator $\$:S_*(X)\to S_*(X)$, with the
idea that it will shrink chains and by iteration eventually render any chain 
$\mathscr{A}$-small. Does $\$$ succeed in making simplices smaller? Let's 
look first at the affine case. Recall that the ``diameter'' of a subset $X$ of 
a metric space is given by
\[
\mathrm{diam}(X)=\sup\{d(x,y):x,y\in X\}\,.
\]

\begin{lemma}
Let $\sigma$ be an affine $n$-simplex, and $\tau$ a simplex in $\$\sigma$.
Then $\mathrm{diam}(\tau)\leq \frac{n}{n+1}\mathrm{diam}(\sigma)$.
\end{lemma}
\begin{proof}
Suppose that the vertices of $\sigma$ are $v_0,v_1,\ldots,v_n$. Let $b$ be the 
barycenter of $\sigma$, and write the vertices of $\tau$ as $w_0=b,w_1,\ldots,w_n$. We want to estimate $|w_i-w_j|$. First, compute
\[
|b-v_i|  =\left|\frac{v_0+\cdots+v_n-(n+1)v_i}{n+1}\right|
=\left|\frac{(v_0-v_i)+(v_1-v_i)+\cdots+(v_n-v_i)}{n+1}\right|\,.
\]
One of the terms in the numerator is zero, so we can continue: 
\[
|b-v_i| \leq \frac{n}{n+1}\max_{i,j}|v_i-v_j| 
= \frac{n}{n+1}\mathrm{diam}(\sigma)
\]
Since $w_i\in\sigma$, 
\[
|b-w_i| \leq\max_i|b-v_i| \leq \frac{n}{n+1}\mathrm{diam}(\sigma)\,.
\]
For the other cases, we use induction:
\[
|w_i-w_j| \leq \mathrm{diam}(\text{simplex in }\$d\sigma)
\leq \frac{n-1}{n}\mathrm{diam}(d\sigma)
\leq \frac{n}{n+1}\mathrm{diam}(\sigma)\,.
\]
\end{proof}

Now let's transfer this calculation to singular simplices in a space $X$ 
equipped with a cover $\mathscr{A}$. 
\begin{lemma}
For any singular chain $c$, some iterate of the subdivision operator sends $c$ to an $\mathscr{A}$-small chain.
\end{lemma}
\begin{proof} We may assume that $c$ is a single simplex $\sigma:\Delta^n\to X$, because in general you just take the largest of the iterates of $\$$ needed to send the simplices in $c$ to a $\mathscr{A}$-small chains.
We now encounter another of the great virtues of singular homology:
we pull $\mathscr{A}$ back to a cover of the standard simplex. 
Define an open cover of $\Delta^n$ by 
\[
\mathscr{U}:=\{\sigma^{-1}(\mathrm{Int}(A)):A\in\sca\}\,.
\]
The space $\Delta^n$ is a compact metric space, and so is subject to the
Lebesgue covering lemma, which we apply to the open cover $\mathscr{U}$.
\begin{lemma}[Lebesgue covering lemma]
Let $M$ be a compact metric space, and let $\mathscr{U}$ be an open cover. Then there is $\epsilon> 0$ such that for all $x\in M$, 
$B_\epsilon(x)\subseteq U$ for some $U\in \mathscr{U}$.
\end{lemma}
\end{proof}

To apply this, we will have to understand iterates of the subdivision operator.

\begin{lemma} For any $k\geq1$,
$\$^k\simeq 1:S_\ast(X)\to S_\ast(X)$.
\end{lemma}
\begin{proof}
We construct $T_k$ such that $dT_k+T_kd=\$^k-1$. To begin, we take $T_1=T$, since $dT+Td=\$-1$. Let's apply $\$$ to this equation. We get $\$dT+\$Td=\$^2-\$$. Sum up these two equations to get 
\[
dT+Td+\$dT+\$Td = \$^2-1\,,
\] 
which simplifies to 
\[
d(\$+1)T+(\$+1)Td=\$^2-1
\]
since $\$d=d\$$.

So define $T_2=(\$+1)T$. Continuing, you see that we can define
\[
T_k=(\$^{k-1}+\$^{k-2}+\cdots+1)T\,.
\]
\end{proof}

We are now in position to prove the Locality Principle, which we recall:
\begin{theorem}[The locality principle]
Let $\mathscr{A}$ be a cover of a space $X$. 
The inclusion $S^\mathscr{A}_\ast(X)\subseteq S_\ast(X)$ is a 
quasi-isomorphism; that is, $H^\mathscr{A}_\ast(X)\to H_\ast(X)$ 
is an isomorphism.
\end{theorem}
\begin{proof}
To prove surjectivity let $c$ be an $n$-cycle in $X$. We want to find an $\sca$-small $n$-cycle that is homologous to $c$. There's only one thing to do. Pick $k$ such that $\$^k c$ is $\sca$-small. This is a cycle because because $\$^k$ is a chain map. I want to compare this new cycle with $c$. That's what the chain homotopy $T_k$ is designed for: 
\[
\$^kc-c=dT_k c+T_kdc=dT_kc
\]
since $c$ is a cycle. So $\$^kc$ and $c$ are homologous.

Now for injectivity. Suppose $c$ is a cycle in $S^\sca_n(X)$ such that $c=db$ for some $b\in S_{n+1}(X)$. We want $c$ to be a boundary of an $\sca$-small chain. Use the chain homotopy $T_k$ again: Suppose that $k$ is such that $\$^kc$ is $\sca$-small. Compute:
\[
d\$^kb-c=d(\$^k-1)b=d(dT_k+T_kd)b=dT_kc
\]
so 
\[
c=d\$^kb-dT_kc=d(\$^kb-T_kc)\,.
\]
Now, $\$^kb$ is $\sca$-small, by choice of $k$. Is $T_kc$ also $\sca$-small? I claim that it is. Why? It is enough to show that $T_k\sigma$ is $\sca$-small if $\sigma$ is. We know that $\sigma=\sigma_\ast\iota_n$. Because $\sigma$ is $\sca$-small, we know that $\sigma:\Delta^n\to X$ is the composition $i_\ast\overline{\sigma}$ where $\overline{\sigma}:\Delta^n\to A$ and $i:A\to X$ is the inclusion of some $A\in\sca$. By naturality, then, $T_k\sigma=T_ki_\ast\overline{\sigma}=i_\ast T_k\overline{\sigma}$, which certainly is $\sca$-small.
\end{proof}

This completes the proof of the Eilenberg Steenrod axioms for singular homology. In the next chapter, we will develop a variety of practical tools, using these axioms to compute the singular homology of many spaces. 

\begin{landscape}
\begin{center}
\textbf{Lefschetz progeny}
\end{center}
\medskip
\noindent
According to the Mathematical Genealogy Project, Solomon Lefschetz had
9312 academic descendents as of March 2018. Here are just a few, with special 
attention to MIT faculty (marked with an asterisk).

\scriptsize
\[
\xymatrix{
&&& \mathrm{Solomon\,Lefschetz} \ar@{-}[dll] \ar@{-}[d] \ar@{-}[drr] \\
& \mathrm{Norman\,Steenrod} \ar@{-}[dl] \ar@{-}[d] \ar@{-}[dr] &&
\mathrm{Ralph\,Fox} \ar@{-}[d] \ar@{-}[dr] &&
\mathrm{Albert\,Tucker} \ar@{-}[d] \ar@{-}[dr] \\
\mathrm{Frank\,Peterson*} \ar@{-}[d] & 
\mathrm{George\,Whitehead*} \ar@{-}[d] &
\mathrm{Edgar\,Brown} \ar@{-}[d] & 
\mathrm{Barry\,Mazur} \ar@{-}[d] & 
\mathrm{John\,Milnor*} & 
\mathrm{Marvin\,Minsky*} \ar@{-}[dl] \ar@{-}[d] \ar@{-}[dr] &
\mathrm{John\,Nash*} \\
\mathrm{David\,Anick*} & 
\mathrm{John\,Moore} \ar@{-}[dl] \ar@{-}[d] \ar@{-}[dr] &
\mathrm{Ralph\,Cohen} &
\mathrm{Noam\,Elkes}  \ar@{-}[d] &
\mathrm{Silvio\,Micali*} \ar@{-}[d] &
\mathrm{Manuel\,Blum} \ar@{-}[d] \ar@{-}[dr] & 
\mathrm{Patrick\,Winston*} \\
\mathrm{Peter\,May} \ar@{-}[d] & 
\mathrm{William\,Browder} \ar@{-}[d] \ar@{-}[dr] & 
\mathrm{Haynes\,Miller*} &
{\txt{Henry Cohn*\\Abhinav Kumar*}} &
\mathrm{Bonnie\,Burger*} &
\mathrm{Gary\,Lee\,Miller} \ar@{-}[dl] & 
\mathrm{Mike\,Sipser*} \ar@{-}[dl] \ar@{-}[d] \\
\mathrm{Mark\,Behrens*} &  
\mathrm{Dennis\,Sullivan*} \ar@{-}[dl] \ar@{-}[dr] & 
\mathrm{George\,Lusztig*} &&
\mathrm{Tom\,Leighton*} \ar@{-}[dl] \ar@{-}[dr] &
\mathrm{Dan\,Spielman*} &
\mathrm{Andrew\,Sutherland*} \\
\mathrm{Curt\,McMullen} &&
\mathrm{Hal\,Abelson*} &
\mathrm{Peter\,Shor*} &&
\mathrm{Ankur\,Moitra*} 
}
\]
\normalsize
\end{landscape}


