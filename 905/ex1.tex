\begin{exercise}\label{simplicialidentities}
A proof of this theorem is included in the following exercise, which introduces
the important concept of a simplicial set.  
    \begin{enumerate}
	\item Any order-preserving map $\phi:[n]\to[m]$ extends to an affine 
map denoted by the same symbol:
	    $\phi:\Delta^n\to \Delta^m$. Give an explicit formula for
	    $\phi$ in terms of barycentric coordinates.
	\item Prove that any order-preserving map factors uniquely as the
	    composite of an order-preserving surjection followed by an
	    order-preserving injection.
	\item Let $d^j:[n-1]\to [n]$ be the order-preserving injection omitting
	    $j$ as a value. Prove that an order-preserving injection
	    $\phi:[n-k]\to[n]$ is uniquely a composition of the form $d^{j_k}
	    d^{j_{k-1}}\cdots d^{j_1}$ with $0\leq j_1<j_2<\cdots<j_k\leq n$.
	    Define the $j_i$'s in terms of $\phi$. Verify the straightening
	    rule
	    $$d^id^j = d^{j+1} d^i \text{ if }i\leq j.$$
	\item Let $s^i:[n+1]\to [n]$ be the order-preserving surjection
	    repeating the value $i$. Such maps are called ``degeneracies.''
Show that any order-preserving surjection
	    $\phi:[m]\to [n]$ is uniquely a composition of the form
	    $(s^n)^{i_n}(s^{n-1})^{i_{n-1}}\cdots(s^0)^{i_0}$ by describing the
	    $i_j$'s in terms of $\phi$. Verify a straightening rule for the
	    composite $s^i s^j$.
	\item Verify a straightening rule for $s^i d^j$.
	\item Write down the straightening rules for the induced maps $d_i =
	    (d^i)^\ast:\Sin_n(X)\to \Sin_{n-1}(X)$ and $s_i =
	    (s^i)^\ast:\Sin_m(X)\to \Sin_{m+1}(X)$. Use these to verify that
	    $\partial^2 = 0:S_n(X)\to S_{n-2}(X)$.
	\item Let $f:X\to Y$ be a continuous map. This induces a map
	    $f_\ast:\Sin_n(X)\to \Sin_n(Y)$. Show that the $f_\ast$ assemble to
	    give a map of simplicial sets, i.e., show that the maps $f_\ast$
	    commute with the maps induced by order-preserving maps $\phi:[m]\to
	    [n]$.
    \end{enumerate}
\end{exercise}
