\section{A plethora of products}

We are now heading towards a statement of the Poincar\'e duality theorem. 

Recall that we have the Kronecker pairing 
\[
\langle-,-\rangle: H^p(X:R)\otimes H_p(X:R)\to R\,.
\]
It's obviously not ``natural,'' because $ H^p$ is contravariant while homology is covariant. But given $f:X\to Y$, $b\in H^p(Y)$, and $x\in H_p(X)$, we can ask: How does $\langle f^\ast b,x\rangle$ relate to $\langle b,f_\ast x\rangle$?
\begin{claim}
$\langle f^\ast b,x\rangle=\langle b,f_\ast x\rangle$.
\end{claim}
\begin{proof}
This is easy! I find it useful to write out diagrams of where things are. We're going to work on the chain level.
\begin{equation*}
	\xymatrix{
\Hom(S_p(Y),R)\otimes S_p(X)\ar[r]^{1\otimes f_\ast}\ar[d]^{f^\ast\otimes 1} & 
\Hom(S_p(Y),R)\otimes S_p(Y)\ar[d]^{\langle-,-\rangle} \\
\Hom(S_p(X),R)\otimes S_p(X)\ar[r]^{\langle-,-\rangle} & R
	}
\end{equation*}
We want this diagram to commute. Suppose $[\beta]=b$ and $[\xi]=x$. Then from the to left, going to the right and then down gives 
\[
\beta\otimes\xi\mapsto\beta\otimes f_\ast(\xi)\mapsto\beta(f_\ast\xi)\,.
\]
The other way gives 
\[
\beta\otimes\xi\mapsto f^\ast(\beta)\otimes\xi=(\beta\circ f_*)\otimes\xi\mapsto(\beta\circ f_*)(\xi)\,.
\]
This is exactly $\beta(f_\ast\xi)$.
\end{proof}
There's actually another product around: 
\[
\mu:H(C_\ast)\otimes H(D_\ast)\to H(C_\ast\otimes D_\ast)
\]
given by $[c]\otimes [d]\mapsto[c\otimes d]$. I used it to pass from the 
chain level computation we did to the homology statement.

We also have the two cross products:
\[
\times: H_p(X)\otimes H_q(Y)\to H_{p+q}(X\times Y)
\]
and
\[\times: H^p(X)\otimes H^q(Y)\to H^{p+q}(X\times Y)\,.
\]
You should think of this as fishy because both maps are in the same direction.
This is OK because we used different things to make these constructions:
the chain-level cross product (or Eilenberg-Zilber map) 
for homology and the Alexander-Whitney map for
cohomology. Still, they're related:
\begin{lemma}
\label{lemma-cp-cross}
Let $a\in H^p(X),b\in H^q(Y),x\in H_p(X), y\in H_q(Y)$. Then:
\begin{equation*}
\langle a\times b,x\times y\rangle=(-1)^{|x|\cdot |b|}\langle a,x\rangle\langle b,y\rangle
\end{equation*}
\end{lemma}
\begin{proof}
Look at the chain-level cross product and the Alexander-Whitney map
\[
\times:S_*(X)\otimes S_*(Y)\leftrightarrows S_*(X\times Y):\alpha
\]
Both of them are the identity in dimension 0, 
and both sides are projective resolutions with respect to the models
$(\Delta^p,\Delta^q)$; so by acyclic models they are natural chain 
homotopy inverses. 

Say $[f]=a,[g]=b,[\xi]=x,[\eta]=y$. Write 
$fg$ for the composite 
\[
S_p(X)\otimes S_q(Y)\xrightarrow{\times}S_{p+q}(X\times Y)
\xrightarrow{f\otimes g}R\otimes R\to R\,.
\]
Then:
\[
(f\times g)(\xi\times\eta)=(fg)\alpha(\xi\times\eta)\sim(fg)(\xi\otimes\eta) 
=(-q)^{pq}f(\xi)g(\eta)\,.
\]
\end{proof}

We can use this to prove a restricted form of the K\"unneth theorem in 
cohomology.
\begin{theorem}
\label{thm-coh-kunneth}
Let $R$ be a PID. Assume that $H_p(X)$ is a finitely generated free $R$-module
for all $p$. Then 
\[
\times:H^\ast(X;R)\otimes_RH^\ast(Y;R)\to H^\ast(X\times Y;R)
\]
is an isomorphism. 
\end{theorem}
\begin{proof}
Write $M^\vee$ for the linear dual of an $R$-module $M$. 
By our assumption about $H_p(X)$, the map
\[
H_p(X)^\vee\otimes H_q(Y)^\vee\to\left(H_p(X)\otimes H_q(Y)\right)^\vee\,,
\]
sending $f\otimes g$ to $(x\otimes y\mapsto(-1)^{pq}f(x)g(y))$,
is an isomorphism. The homology K\"unneth theorem guarantees that the 
bottom map in the following diagram is an isomorphism.
\[
\xymatrix{
\bigoplus_{p+q=n}H^p(X)\otimes H^q(Y) \ar[rr]^\times\ar[d]^\cong && 
H^n(X\times Y) \ar[d]^\cong \\
\bigoplus_{p+q=n}H_p(X)^\vee\otimes H_q(Y)^\vee \ar[r]^\cong &
\left(\bigoplus_{p+q=n}H_p(X)\otimes H_q(Y)\right)^\vee &
H_n(X\times Y)^\vee \ar[l]_\cong
}\]
Commutativity of this diagram is exactly the content of 
Lemma \ref{lemma-cp-cross}. 
\end{proof}

We saw before that $\times$ is an algebra map, so under the conditions of the
theorem it is an isomorphism of algebras. You do need some finiteness 
assumption, even if you are working over a field. For example let $T$ be an
infinite set, regarded as a space with the discrete topology. Then 
$H^0(T;R)=\Map(T,R)$. But
\[
\Map(T,R)\otimes\Map(T,R)\to\Map(T\times T,R)
\]
sending $f\otimes g$ to $(s,t)\to f(s)g(t)$ is not surjective; the 
characteristic function of the diagonal is not in the image, for example. 

There are more products around. For example, there is a map 
\[
H^p(Y)\otimes H^q(X,A)\to H^{p+q}(Y\times X,Y\times A)\,.
\]
Constructing this is on your homework. Suppose $Y=X$. 
Then I get 
\[
\cup: H^\ast(X)\otimes H^\ast(X,A)\to H^\ast(X\times X,X\times A)\xrightarrow{\Delta^\ast} H^\ast(X,A)
\]
where $\Delta:(X,A)\to (X\times X,X\times A)$ is the ``relative diagonal.'' 
This ``relative cup product'' makes $ H^\ast(X,A)$ into a module over the graded algebra $ H^\ast(X)$. The relative cohomology is \emph{not} a ring -- it doesn't have a unit, for example -- but it is a module. And the long exact sequence of the pair is a sequence of $ H^\ast(X)$-modules. 

I want to introduce you to one more product, which will enter into our
expression of Poincar\'{e} duality. This is the {\em cap product}.
What can I do with $S^p(X)\otimes S_n(X)$? Well, I can form the composite:
\begin{equation*}
S^p(X)\otimes S_n(X)\xrightarrow{1\times (\alpha_{X,X}\circ \Delta_\ast)} S^p(X)\otimes S_p(X)\otimes S_{n-p}(X)\xrightarrow{\langle -,-\rangle\otimes 1}S_{n-p}(X)
\end{equation*}
Using our explicit formula for $\alpha$, we can write:
\begin{equation*}
\cap:\beta\otimes\sigma\mapsto\beta\otimes(\sigma\circ\alpha_p)\otimes(\sigma\circ\omega_q)\mapsto\left(\beta(\sigma\circ\alpha_p)\right) (\sigma\circ\omega_q)
\end{equation*}
We are evaluating the cochain on {\em part of} the chain, leaving a lower 
dimensional chain left over.

This composite is a chain map, and so induces a map in homology:
\[
\cap:H^p(X)\otimes H_n(X)\to H_{n-p}(X)\,.
\]
Here are some properties of the cap product.
\begin{lemma}
$(\alpha\cup\beta)\cap x=\alpha\cap(\beta\cap x)$ and $1\cap x=x$.
\end{lemma}
\begin{proof}
Easy to check from the definition.
\end{proof}
This makes $ H_\ast(X)$ into a module over $ H^\ast(X)$. These are not hard things to check. There's a lot of structure, and the fact that $ H^\ast(X)$ forms an algebra is a good thing. Notice how the dimensions work. Long ago a bad choice was made: we should index cohomology with negative numbers, so that the grading in $\cap: H^p(X)\otimes H_n(X)\to H_{n-p}(X)$ makes sense. A cochain complex with positive grading is the same as a chain complex with negative grading.

There are also two slant products. Maybe we won't talk about them. We will check a few things about cap products, and then we'll get into the machinery of Poincar\'{e} duality.
