\section{Universal coefficient theorem}

Suppose that we are given $ H_\ast(X;\Z)$. Can we compute $ H_\ast(X;\Z/2\Z)$? 
This is non-obvious. Consider the map $\RP^2\to S^2$ that pinches $\RP^1$ to a point. Now $H_2(\RP^2;\Z)=0$, so in $H_2$ this map is zero. But in $\Z/2\Z$-coefficients, in dimension $2$, this map gives an isomorphism. This shows that there's no {\em functorial} determination of $ H_\ast(X;\Z/2)$ in terms of $ H_\ast(X;\Z)$; the effect of a map in integral homology does not determine its effect in mod 2 homology. So how \emph{do} we go between different coefficients?

Let $R$ be a commutative ring and $M$ an $R$-module, and suppose we have a chain complex $C_\ast$ of $R$-modules. It could be the singular complex of a space, but it doesn't have to be. Let's compare $ H_n(C_\ast)\otimes M$ with $ H_n(C_\ast\otimes M)$. (Here and below we'll just write $\otimes$ for $\otimes_R$.) The latter thing gives homology with coefficients in $M$. How can we compare these two? Let's investigate, and build up conditions on $R$ and $C_\ast$ as we go along. 

First, there's a natural map 
\[
\alpha: H_n(C_\ast)\otimes M\to H_n(C_\ast\otimes M)\,,
\]
sending $[z]\otimes m$ to $[z\otimes m]$.
We propose to find conditions under which it is 
injective. The map $\alpha$ fits into a commutative 
diagram with exact columns like this:
\[
\xymatrix{
0 & 0 \\
H_n(C_*)\otimes M \ar[r]^\alpha \ar[u] & H_n(C_*\otimes M) \ar[u] \\
Z_n(C_*)\otimes M \ar[r] \ar[u] & Z_n(C_*\otimes M) \ar[u] \\
C_{n+1}\otimes M \ar[r]^= \ar[u] & C_{n+1}\otimes M \ar[u] \,.
}\]
Now, $Z_n(C_*\otimes M)$ is a submodule of $C_n\otimes M$, but the map
$Z_n(C)\otimes M\to C_n\otimes M$ need not be injective \ldots unless we impose more
restrictions. If we can guarantee that it is, then a diagram chase shows
that $\alpha$ is a monomorphism. 

So let's assume that $R$ is a PID and that $C_n$ is a free $R$-module for 
all $n$. Then the submodule $B_{n-1}(C_*)\subseteq C_{n-1}$ is again free, 
so the short exact sequence 
\[
\xymatrix{
0 \ar[r] & Z_n(C_*) \ar[r] & C_n \ar[r] \ar[dr]^d & 
B_{n-1}(C_*) \ar[r] \ar[d] & 0 \\
&&& C_{n-1} 
}\]
splits. So $Z_n(C_*)\to C_n$ is a split monomorphism, and hence 
$Z_n(C_*)\otimes M\to C_n\otimes M$ is too. 

In fact, a little thought shows that this argument produces a splitting of
the map $\alpha$. 

Now, $\alpha$ is not always an isomorphism. But it certainly is if $M=R$, 
and it's compatible with direct sums, so it certainly is if $M$ is free. 
The idea is now to resolve $M$ by frees, and see where that idea takes us.

So let 
\[
0\to F_1\to F_0\to M\to0
\]
be a free resolution of $M$. Again, we're using the assumption that $R$ is
a PID, to guarantee that $\ker(F_0\to M)$ is free. Again using the assumption
that each $C_n$ is free, we get a short exact sequence of chain complexes
\[
0\to C_*\otimes F_1\to C_*\otimes F_0\to C_*\otimes M\to0\,.
\]

In homology, this gives a long exact sequence. Unsplicing it gives the
left-hand column in the following diagram.
\begin{equation*}
\xymatrix{
0 \ar[d] & 0\ar[d] \\
\coker(H_n(C_\ast\otimes F_1)\to H_n(C_\ast\otimes F_0)) \ar[d] \ar[r]^\cong 
& \coker(H_n(C_\ast)\otimes F_1\to H_n(C_\ast)\otimes F_0)) \ar[d] \\
H_n(C_\ast\otimes M) \ar[d]^\partial \ar[r]^= & 
H_n(C_\ast\otimes M) \ar[d]\\
\ker( H_{n-1}(C_\ast\otimes F_1)\to H_{n-1}(C_\ast\otimes F_0))
\ar[r]^\cong\ar[d] & 
\ker(H_{n-1}(C_*)\otimes F_1\to H_{n-1}(C_*)\otimes F_0) \ar[d]\\
0 & 0}
\end{equation*}
The right hand column occurs because $\alpha$ is an isomorphism when the module
involved is free. But 
\[
\coker(H_n(C_\ast)\otimes F_1\to H_n(C_\ast)\otimes F_0))=H_n(C_*)\otimes M
\]
and
\[
\ker(H_{n-1}(C_\ast)\otimes F_1\to H_{n-1}(C_\ast)\otimes F_0)=
\Tor^R_1(H_{n-1}(C_\ast),M)\,.
\]
We have proved the following theorem.
\begin{theorem}[Universal Coefficient Theorem]
Let $R$ be a PID and $C_\ast$ a chain complex of $R$-modules such that $C_n$ 
is free for all $n$. Then there is a natural short 
exact sequence of $R$-modules
\begin{equation*}
0\to H_n(C_\ast)\otimes M\xrightarrow{\alpha} H_n(C_\ast\otimes M)\xrightarrow{\partial}\Tor^R_1( H_{n-1}(C_\ast),M)\to 0
\end{equation*}
that splits (but not naturally).
\end{theorem}
\begin{example}
The pinch map $\RP^2\to S^2$ induces the following map of universal
coefficient short exact sequences:
\[
\xymatrix{
0 \ar[r] & H_2(\RP^2)\otimes\Z/2\Z \ar[d]^0 \ar[r] & 
H_2(\RP^2;\Z/2\Z) \ar[d]^\cong \ar[r]^-\cong & 
\Tor_1(H_1(\RP^2),\Z/2\Z) \ar[d]^0 \ar[r] & 0 \\
0 \ar[r] & H_2(S^2)\otimes\Z/2\Z \ar[r]^-\cong & H_2(S^2;\Z/2\Z) \ar[r] &
\Tor_1(H_1(S^2),\Z/2\Z) \ar[r] & 0 
}\]
This shows that the splitting of the universal coefficient short exact 
sequence cannot be made natural, and it explains the mystery that we began 
with.
\end{example}
\begin{exercise}
The hypotheses are essential. Construct two counterexamples:
one with $R=\Z$ but in which the groups in the chain complex are not free,
and one in which $R=k[d]/d^2$ and the modules in $C_*$ are free over $R$.  
\end{exercise}

