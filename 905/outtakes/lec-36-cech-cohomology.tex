\section{$\cHH^\ast$ as a cohomology theory}

\subsection{Cofinality}
Let $\cI$ be a directed set. Let $A:\cI\to \mathbf{Ab}$ be a functor. If I have a functor $f:\cK\to\cI$, then I get $Af:\cK\to\mathbf{Ab}$, i.e., $(Af)_j=A_{f(j)}$.

I can form $\varinjlim_{\cK}Af$ and $\varinjlim_{I}A$. I claim you have a map $\varinjlim_{\cK}Af\to\varinjlim_{\cI}A$. All I have to do is the following:
\begin{equation*}
\xymatrix{
	\varinjlim_{J}Af\ar[r] & \varinjlim_{I}A\\
	A_{f(j)}\ar[u]^{\mathrm{in}_j}
}
\end{equation*}
So I have to give you maps $A_{f(j)}\to\varinjlim_{I}A$ for various $j$. I know what to do, because I have $\mathrm{in}_{f(j)}:A_{f(j)}\to\varinjlim_{I}A$. Are they compatible when I change $j$? Suppose I have $j^\prime\leq j$. Then I get a map $f(j^\prime)\to f(j)$, so I have a map $A_{f(j^\prime)}\to A_{f(j)}$, and thus the maps are compatible. Hence I get:
\begin{equation*}
\xymatrix{
	\varinjlim_{J}Af\ar[r] & \varinjlim_{I}A\\
	(Af)_j=A_{f(j)}\ar@{-->}[ur]^{\mathrm{in}_{f(j)}}\ar[u]^{\mathrm{in}_j}
}
\end{equation*}
\begin{example}
Suppose $K\supseteq L$ be closed, then I get a map $\cHH^\ast(K)\to\cHH^\ast(L)$. Is this a homomorphism? Well, $\cHH^\ast(K)=\varinjlim_{U\in\mathcal{U}_K}H^\ast(U)$ and $\cHH^\ast(L)=\varinjlim_{V\in\mathcal{U}_L}H^\ast(V)$. This is an example of a $\cI$ and $\cK$ that I care about. Well, $\mathcal{U}_K\subseteq\mathcal{U}_L$, and thus I get a map $\cHH^\ast(K)\to\cHH^\ast(L)$, which is what I wanted.

I can do something for relative cohomology. Suppose:
\begin{equation*}
\xymatrix{K\ar@{^(->}[d] & L\ar@{^(->}[d]\ar@{_(->}[l] \\ K^\prime & L^\prime\ar@{_(->}[l]}
\end{equation*}
I get a homomorphism $\cHH^\ast(K,L)\to \cHH^\ast(K^\prime,L^\prime)$ because I have $\mathcal{U}_{K,L}\to\mathcal{U}_{K^\prime,L^\prime}$.
\end{example}
This isn't exactly what we need:
\begin{question}
When does $f:\cK\to\cI$ induce an isomorphism $\varinjlim_{J}Af\to\varinjlim_{I}A$?
\end{question}
This is a lot like taking a sequence and a subsequence and asking when they have the same limit. There's a cofinality condition in analysis, that has a similar expression here.
\begin{definition}
$f:\cK\to\cI$ is cofinal if for all $i\in\cI$, there exists $j\in\cK$ such that $i\leq f(j)$.
\end{definition}
\begin{example}
If $f$ is surjective.
\end{example}
\begin{lemma}
If $f$ is cofinal, then $\varinjlim_{J}Af\to\varinjlim_{I}A$ is an isomorphism.
\end{lemma}
\begin{proof}
Check that $\{A_{f(j)}\to\varinjlim_{I}A\}$ satisfies the necessary and sufficient conditions:
\begin{enumerate}
\item For all $a\in\varinjlim_{I}A$, there exists $j$ and $a_j\in A_{f(j)}$ such that $a_j\mapsto a$. We know that there exists some $i$ and $a_i\in A$ such that $a_i\mapsto a$. Pick $j$ such that $f(j)\geq i$, so we get a map $a_i\to a_{f(j)}$, and by compatibility, we get $a_{f(j)}\mapsto a$.
\item The other condition is also just as easy.
\end{enumerate}
\end{proof}
This is a very convenient condition.
\begin{example}
I had a perverse way of constructing $\QQ$ by using the divisibility directed system. A much simpler (linear!) directed system is $\Z\xrightarrow{2}\Z\xrightarrow{3}\Z\xrightarrow{4}\Z\to\cdots$. This has the same colimit as the divisibility directed system because $n|n!$, so we have a cofinal map between directed systems.
\end{example}
How about the direct limits in the \v{C}ech cohomology case?
\begin{example}
Do I have a map $\cHH^\ast(K,L)\to\cHH^\ast(K)$? Suppose:
\begin{equation*}
\xymatrix{K\ar@{^(->}[d] & L\ar@{^(->}[d]\ar@{_(->}[l] \\ U & V\ar@{_(->}[l]}
\end{equation*}
Then $\cHH^p(K,L)=\varinjlim_{(U,V)\in\mathcal{U}_{K,L}}H^p(U,V)$ and $\cHH^p(K)=\varinjlim_{U\in\mathcal{U}_K}H^p(U)$. I have a map of directed sets $\mathcal{U}_{K,L}\to\mathcal{U}_K$ by sending $(U,V)\mapsto U$. I didn't have to use cofinality. I want a long exact sequence, though, and I'm going to do this by saying that it's a directed limit of a long exact sequence. I'm going to have to have all of these various \v{C}ech cohomologies as being the directed limit over the \emph{same} indexing set.

I'd really like to say that $\cHH^p(K)=\varinjlim_{U\in\mathcal{U}_K}H^p(U)\cong \varinjlim_{(U,V)\in\mathcal{U}_{K,L}}H^p(U)$. Thus I need to show that $\mathcal{U}_{K,L}\to\mathcal{U}_K$ where $(U,V)\mapsto U$ is cofinal. This is easy, because if $U\in\mathcal{U}_K$, just pick $(U,U)$, i.e., $\mathcal{U}_{K,L}\to\mathcal{U}_K$ is cofinal. How about $\mathcal{U}_{K,L}\to\mathcal{U}_L$ by $(U,V)\mapsto V$; is it cofinal? Yes! For $V\in\mathcal{U}_L$, pick $(X,V)$! This means that $\cdots\cHH^{p-1}(L)\to\cHH^p(K,L)\to\cHH^p(K)\to\cHH^p(L)\to\cHH^{p+1}(K,L)$ is $\varinjlim_{\mathcal{U}_{K,L}}\left(\cdots\to H^{p}(U,V)\to\cdots\right)$, and hence exact.
\end{example}
How about excision? I need this to get to Mayer-Vietoris!
\begin{lemma}
Assume $X$ is normal and $A,B$ are closed subsets. Then $\cHH^p(A\cup B,B)\to\cHH^p(A,A\cap B)$ is an isomorphism. 
\end{lemma}
\begin{proof}
Well, $\cHH^p(A\cup B,B)$ is $\varinjlim$ over $\mathcal{U}_{A\cup B,B}$ and $\cHH^p(A,A\cap B)$ is $\varinjlim$ over $\mathcal{U}_{A,A\cap B}$. Let $W\supseteq A$ and $Y\supseteq B$ are neighborhoods. I claim that $\mathcal{U}_A\times\mathcal{U}_B\to\mathcal{U}_{A\cup B,B}$ sending $(W,Y)\mapsto (W\cup Y,Y)$ and $\mathcal{U}_A\times\mathcal{U}_B\to\mathcal{U}_{A,A\cap B}$ sending $(W,Y)\mapsto (W,W\cap Y)$ are cofinal.

If I give you $(U,V)\in \mathcal{U}_{A\cup B,B}$, define $(W,V)\in\mathcal{U}_A\times\mathcal{U}_B$ where $W=U$ and $Y=V$, so $\mathcal{U}_A\times\mathcal{U}_B\to\mathcal{U}_{A\cup B,B}$ is surjective, hence cofinal. The latter is trickier. Let $U\supseteq A$ and $V\supseteq A\cap B$. Here's where normality comes into play. Separate $B-V$ from $A$. Let $T\supseteq B-V$. Shit. \emph{Shit!}

Maybe I'll leave this to you. I'll put this on the board on Wednesday. Anyway, I'll use normality to show that $\mathcal{U}_A\times\mathcal{U}_B\to\mathcal{U}_{A,A\cap B}$ is cofinal, and thus this verifies excision -- so you actually have excision.
\end{proof}



Let's finish off the proof from last time. Suppose $A,B$ are closed in normal $X$. \underline{Excision for $\cHH^p$}:
\begin{equation*}
\xymatrix{
	\varinjlim_{(W,Y)\in\mathcal{U}_A\times\mathcal{U}_B}H^p(W\cup Y,Y)\ar[rrr]^{\cong,\text{ ordinary excision}}\ar[d]^{\cong,\text{ cofinality/surjectivity}} & & & \varinjlim_{\mathcal{U}_A\times\mathcal{U}_B}H^p(W,W\cap Y)\ar[d]^{\cong,\text{ cofinality, see below}}\\
	\varinjlim_{(U,V)\in\mathcal{U}_{A\cup B,B}}H^p(U,V)\ar[rrr]\ar@{=}[d] & & & \varinjlim_{(U,V)\in\mathcal{U}_{A,A\cap B}}H^p(U,V)\ar@{=}[d]\\
	\cHH^p(A\cup B,B)\ar[rrr] & & & \cHH^p(A,A\cap B)
}
\end{equation*}

$\mathcal{U}_A\times\mathcal{U}_B\to\mathcal{U}_{A,A\cap B}$ is cofinal since: start with $(U,V)\supseteq(A,A\cap B)$. Using normality, separate $B\cap(X-V)\subseteq T$ and $A\subseteq S$. Take $W=U\cap S$ and $Y=V\cup T$. Then $A\subseteq W\subseteq U$ and $A\cap B\subseteq W\cap Y=S\cap V\subseteq V$.

This means that $\cHH^p$ satisfies excision, hence Mayer-Vietoris. Let's put this in the drawer for now.
