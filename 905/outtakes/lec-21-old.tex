\section{Tensor and Tor}

We continue to study properties of the tensor product. Recall that 
\[
A\otimes \Z/n\Z=A/nA\,.
\]
Consider the exact sequence 
\[
0\to \Z\xrightarrow{2}\Z\to \Z/2\Z\to 0\,.
\]
Let's tensor it with $\Z/2\Z$. We get
\[
0\to \Z/2\Z\to\Z/2\Z\to\Z/2\Z\to 0\,.
\]
This cannot be a short exact sequence! 
This is a major tragedy: tensoring doesn't preserve exact sequences; one says that $\Z/n\Z\otimes-$ is not ``exact.'' This is why we can't form homology with coefficients in $A$ by simply tensoring homology with $A$. 

Tensoring does respect certain exact sequences:
\begin{prop}
The functor $N\mapsto M\otimes_R N$ preserves cokernels; it is \emph{right exact}. 
\end{prop}
\begin{proof}
Suppose that $N\to N''$ is a surjection of $R$-modules, and $M$ is any $R$-module. Then 
\[
\xymatrix{
M\otimes_RN \ar[r] & M\otimes_RN'' \\
M\times N \ar[u] \ar@{->>}[r] & M\times N'' \ar[u]\,.
}\]
At least we know that $M\times-$ preserves surjections. But the image of 
$M\times N''$ generates $M\otimes_RN''$ as an $R$-module, so the image of
$M\times N$ generates it as well. This implies that 
$M\otimes_RN\to M\otimes_RN''$ is surjective.
\end{proof}

How about this failure of exactness? What can we do about that? Failure of exactness is bad, so let's try to repair it. Long exact sequences to the rescue!
Inspired by the example of singular homology, we have the following theorem.
\begin{theorem}
Fix a ring $R$ and an $R$-module $M$. 
There is are functors
\[
\Tor^R_n(M,-):\mathbf{Mod}_R\to\mathbf{Mod}_R
\]
for $n\geq 0$, natural isomorphisms 
\[
M\otimes_RN\xrightarrow{\cong}\Tor^R_0(M,N)
\]
and homomorphisms
\[
\partial:\Tor^R_n(M,N'')\to\Tor^R_{n-1}(M,N')
\]
natural in the short exact sequence
\[
0\to N'\to N\to N''\to0
\]
such that 
\[
\Tor^R_n(M,F)=0\,\,\text{for}\,\,n>0\,\,\text{ if $F$ is free}
\]
and 
\begin{equation*}
\xymatrix{\cdots\ar[r] & \Tor^R_n(M,N)\ar[r] & \Tor^R_n(M,N^{\prime\prime})\ar[dll]\\
\Tor^R_{n-1}(M,N^\prime)\ar[r] & \Tor^R_{n-1}(M,N)\ar[r] & \cdots}
\end{equation*}
is exact.
\end{theorem}
You can think of $\Tor^R_*(M,N)$ as the ``homology of $M$ with coefficients
in $N$.'' 

Let's explore what this gives us before we construct it.
\begin{example}
Let $R=\Z$, and consider the short exact sequence 
\[
0\to\Z\xrightarrow{n}\Z\to\Z/n\Z\to 0\,.
\] 
Tensor this with $A$. Because $\Z$ is free, we have an exact sequence:
\[
0\to\Tor_1^\Z(A,\Z/n\Z)\to A\to A\to A/nA\to0\,.
\]
That is to say, $\Tor_1^\Z(A,\Z/n\Z)$ is the $n$-{\em torsion} in $A$:
\[
\Tor_1^\Z(A,\Z/n\Z)=\ker(n|A)\,.
\]
This is the origin of the symbol $\Tor$. 
 
\end{example}
Take a general $R$-module $N$. You can always take a free module $F_0$ that surjects onto $N$, i.e., $F_0\to N\to 0$. For example, you can let $F_0$ be the free $R$-module on the underlying set of $N$. Form a short exact sequence
\[
0\to K_0\to F_0\to N\to 0\,.
\]. 
By our axioms, we find that
\begin{equation*}
\Tor^R_n(M,N)\xrightarrow{\partial}\Tor^R_{n-1}(M,K_0)
\end{equation*}
is an isomorphism for $n>1$, and
\begin{equation*}
0\rightarrow\Tor^R_1(M,N)\xrightarrow{\partial}
M\otimes_R K_0\rightarrow M\otimes_R F_0\rightarrow M\otimes_R N\rightarrow0
\end{equation*}
is exact. This is good, but not quite a construction yet because it appears
that these values probably depend upon the map $F_0\to N$. Moreover, when
I come to compute $\Tor_{n-1}^R(M,K_0)$, I'll need to pick a surjection
from a free module $F_1$ onto $K_0$. 

Let's pursue the second idea now, and come back to uniqueness later. 
By continuing to find surjections from frees, we get a diagram like this:
\begin{equation*}
\xymatrix{\cdots\ar[rr] & & F_2\ar[dr]\ar[rr]^d & & F_1\ar[dr]\ar[rr]^d & & F_0\ar[dr]\\
& K_2\ar[ur]\ar[dr] & & K_1\ar[ur]\ar[dr] & & K_0\ar[ur]\ar[dr] & & N\ar[dr]\\
0\ar[ur] & & 0\ar[ur] & & 0\ar[ur] & & 0\ar[ur] & & 0}
\end{equation*}
-- in which the upside-down V subdiagrams are short exact sequences.
and $F_s$ is free for all $s$. Splicing these exact sequences gives you a exact sequence in the top row. This is a \emph{free resolution of $N$}.
The top row, $F_*$, is a chain complex. It maps to the very short chain
complex with $N$ in degree 0 and 0 elsewhere, and this chain map is a 
homology isomorphism (or ``quasi-isomorphism''). We have in effect replaced
$N$ with this chain complex of free modules. The module $N$ may be very
complicated, with generators, relations, relations between relations \ldots.
All this is laid out in front of us by the free resolution. Generators of $F_0$
map to generators for $N$, and generators for $F_1$ map to relations among
those generators. 




