\section{Categories, functors, natural transformations}\label{categories}
%Office hours. Hood Chatham's are Mondays, 1:30 - 3:30, in 2-390A. Miller's is Tuesdays, 3:00-5:00 in 2-478. (commented out because not relevant to readers)
%Replaced \mathbf{C} with \cC as brought up in ``Stuff to fix.''
%Rearranged some stuff. Added a notational remark following the definition of a category.
From spaces and continuous maps, we constructed graded abelian groups and homomorphisms. We now cast this construction in the more general language of category theory.

Our discussion of category theory will be interspersed throughout the text, introducing new concepts as they are needed. Here we begin by introducing the basic definitions.

\begin{definition}
A \emph{category} $\cc$ consists of the following data.
\begin{itemize}
\item a class $\mathrm{ob}(\cc)$ of objects;
\item for every pair of objects $X$ and $Y$, a set of \emph{morphisms} 
$\cc(X,Y)$;
\item for every object $X$ a {\em identity morphism} $1_X\in\cc(X,X)$; and
\item for every triple of objects $X,Y,Z$, a {\em composition} map
$\cc(X,Y)\times\cc(Y,Z)\to\cc(X,Z)$, written $(f,g)\mapsto g \circ f$. 
\end{itemize}
These data are required to satisfy the following:
\begin{itemize}
\item $1_Y\circ f=f$, and $f\circ 1_X=f$.
\item Composition is associative: $(h\circ g)\circ f=h\circ(g\circ f)$.
\end{itemize}
\end{definition}
Note that we allow the collection of objects to be a class. This enables us to talk about a ``category of all sets'' for example. But we require each 
$\cc(X,Y)$ to be set, and not merely a class. Some interesting categories have
a {\em set} of objects; they are called {\em small categories}.

We will often write $X\in\cc$ to mean $X\in\mathrm{ob}(\cc)$, and $f\colon X\to Y$ to mean $f\in \cc(X,Y)$.
\begin{definition}
If $X,Y\in \cc$, then $f\colon X\to Y$ is an \emph{isomorphism} if there exists $g\colon Y\to X$ with $f \circ g=1_Y$ and $g\circ f=1_X$. We may write
\[
f:X\xrightarrow{\cong}Y
\]
to indicate that $f$ is an isomorphism. 
\end{definition}

\begin{example}
Many common mathematical structures can be arranged in categories.
\begin{itemize}
\item Sets and functions between them form a category $\set$.
\item Abelian groups and homomorphisms form a category $\mathbf{Ab}$.
\item Topological spaces and continuous maps form a category $\mathbf{Top}$.
\item Simplicial sets and their maps form a category $s\set$.
\item A monoid is the same as a category with one object, where the elements of the monoid are the morphisms in the category. It's a small category.
\item The sets $[n]=\{0,\ldots,n\}$ for $n\geq 0$ together with weakly order-preserving maps between them form the {\em simplex category} $\Deltab$, 
another small category. It contains as a subcategory the {\em semi-simplex
category} $\Deltab_{inj}$ with the same objects but only injective weakly order-preserving maps. 
\item A poset forms a category in which there is a morphism from $x$ to $y$ iff $x\leq y$. A small category with this property comes from a poset provided 
that the only isomorphisms are identities. 
\end{itemize}
\end{example}

Categories may be related to each other by rules describing effect on both
objects and morphisms. 
\begin{definition}
Let $\cc,\cd$ be categories. A \emph{functor} $F\colon\cc\to\cd$ consists
of the data of 
\begin{itemize}
\item an assignment  $F:\mathrm{ob}(\cc)\to\mathrm{ob}(\cd)$, and
\item for all $x,y\in\mathrm{ob}(\cc)$, map $F:\cc(x,y)\to\cc(F(x),F(y))$ \,.
\end{itemize}
These data are required to satisfy the following two properties:
\begin{itemize}
\item For all $X\in\mathrm{ob}\cc$, $F(1_X)=1_{F(X)}\in\cd(F(X),F(X))$, and
\item For all composable pairs of morphisms $f,g$ in $\cc$, 
$F(g\circ f)=F(g)\circ F(f)$.
\end{itemize}
\end{definition}

We have defined quite a few functors already:
\[
\Sin_n:\Top\to\mathbf{Set}\,,\quad S_n:\Top\to\Ab\,,\quad H_n:\Top\to\Ab\,,
\]
for example. We also have defined, for each $X$, 
a morphism $d:S_n(X)\to S_{n-1}(X)$. This is a ``morphism between 
functors.'' This property is captured by another definition.
\begin{definition}
Let $F,G\colon \cc\to\cd$. A \emph{natural transformation} 
or \emph{natural map} $\theta\colon F\to G$ consists of maps $\theta_X\colon F(X)\to G(X)$ for all $X\in\mathrm{ob}(\cc)$ such that for all $f\colon X\to Y$ the following diagram commutes. 
\begin{equation*}
\xymatrix{F(X)\ar[d]^{F(f)}\ar[r]^{\theta_X} & G(X)\ar[d]^{G(f)}\\
F(Y)\ar[r]^{\theta_Y} & G(Y)}
\end{equation*}
\end{definition}
So for example 
the boundary map $d\colon S_n\to S_{n-1}$ is a natural transformation.

Natural transformations are so \ldots well, so {\em natural} that their
occurance is indicated by a variety of terms: a {\em natural} or 
{\em canonical} map is precisely a natural transformation. 

\begin{example}
Suppose that $\cc$ and $\cd$ are two categories, and assume that $\cc$ is
small. We may then form the {\em category of functors} $\mathrm{Fun}(\cc,\cd)$.
Its objects are the functors fron $\cc$ to $\cd$, and given two functors
$F,G$, $\Fun(\cc,\cd)(F,G)$ is the set of natural transformations from $F$
to $G$. We let the reader define the rest of the structure of this category,
and check the axioms. We needed to assume that $\cc$ is small in order to
guarantee that there is no more than a set of natural transformations between
functors.

For example, let $G$ be a group (or a monoid) viewed as a one-object category. Any element $F\in\mathrm{Fun}(G,\mathbf{Ab})$ is simply a group action of $G$ on $F(\ast)=A$, i.e., a representation of $G$ in abelian groups. Given another $F^\prime\in\mathrm{Fun}(G,\mathbf{Ab})$ with $F^\prime(\ast)=A^\prime$, then a natural transformation from $F\to F^\prime$ is precisely a $G$-equivariant homomorphism $A\to A^\prime$.
\end{example}

