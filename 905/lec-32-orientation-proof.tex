\section{Proof of the orientation theorem}

We are studying the way in which local homological information gives rise
to global information, especially on an $n$-manifold $M$. The tool was the map
\[
j:H_n(M;R)\to\Gamma(M;o_M\otimes R)
\]
sending a class $c$ to the section of the orientation local coefficient
system given at $x\in M$ by the restriction $j_x(c)\in H_n(M,M-x)$. 
We asserted that if $M$ is compact then $j$ is an isomorphism and that
$H_q(M)=0$ for $q>n$. 

To make our induction go, we will need a refinement of this construction.
Let $A\subset M$ be a closed subset. A class in
$H_n(M,M-A)$ is represented by a cycle whose boundary lies outside of $A$. 
It may cover $A$ evenly. We can give meaning to this question as follows.
Let $x\in A$. Then $M-A\subseteq M-x$, so we have a map
\[
j_x:H_n(M,M-A)\to H_n(M,M-x)
\]
that tests whether the chain covers $x$. As $x$ ranges over $A$, 
these maps together give us a map to the group of compactly supported 
sections of $o_M$ over $A$,
\[
j:H_n(M,M-A)\to\Gamma(A;o_M)\,.
\]
Because $H_n(M,M-A)$ deals with homology classes that ``stretch over $A$,''
we will write
\[
H_n(M,M-A)=H_n(M|A)\,.
\]
\begin{theorem} Let $M$ be an $n$-manifold and let $A$ be a compact subset 
of $M$. Then $H_q(M|A;R)=0$ for $q>n$, and the map 
$j:H_n(M|A;R)\to\Gamma(A;o_M\otimes R)$ is an isomorphism.
\label{thm-fundamental-class}
\end{theorem}
Taking $A=M$ (assuming $M$ compact) we find that $H_q(M;R)=0$ for $q>n$ and 
\[
j:H_n(M;R)\xrightarrow{\cong}\Gamma(M;o_M\otimes R)\,.
\]
We follow \cite{bredon} in proving this, 
and refer you to this reference for the modifications appropriate for
the more general statement when $A$ is assumed merely closed rather than
compact.

First we establish two general results.
\begin{lemma} 
Let $A$ and $B$ be closed subspaces of $M$, and suppose the result
holds for $A$, $B$, and $A\cap B$. Then it holds for $A\cup B$.
\label{lem-mv}
\end{lemma}
\begin{proof}
The relative Mayer-Vietoris theorem and the inductive hypothesis that 
$H_{n+1}(M|A\cap B)=0$ gives us exactness of the top row in the ladder
\[
\xymatrix{
0 \ar[r] & H_n(M|A\cup B) \ar[d]^j \ar[r] & 
H_n(M|A)\oplus H_n(M|B) \ar[d]^j \ar[r] &
H_n(M|A\cap B) \ar[d]^j \\
0 \ar[r] & \Gamma(A\cup B;o_M) \ar[r] & 
\Gamma(A;o_M)\oplus\Gamma(B;o_M) \ar[r] &
\Gamma(A\cap B;o_M)\,.
}\]
Exactness of the bottom row is clear: A section over $A\cup B$ is precisely
a section over $A$ and a section over $B$ that agree on the intersection.
So the five-lemma shows that $j_{A\cup B}$ is an isomorphism. 
Looking further back in the Mayer-Vietoris sequence gives the vanishing of
$H_q(M|A)$ for $q>n$.  
\end{proof}
\begin{lemma} Let $A_1\supset A_2\supset\cdots$ be a decreasing sequence
of compact subsets of $M$, and assume that the theorem holds for each $A_n$.
Then it holds for the intersection $A=\cup A_i$. 
\label{lem-lim}
\end{lemma}
\begin{proof}
We claim first that 
\[
\varinjlim_iH_q(M|A_i)\xrightarrow{\cong}H_q(M|A)\,.
\]
Let $\sigma:\Delta^1\to M$ be any simplex in $M-A$. The subsets $M-A_i$
form an open cover of $\img(\sigma)$, so by compactness it lies in some
single $M-A_i$. This shows that 
\[
\varinjlim_iS_q(M-A_i)\xrightarrow{\cong}S_q(M-A)\,.
\]
Thus
\[
\varinjlim_iS_q(M|A_i)\xrightarrow{\cong}S_q(M|A_i)
\]
by exactness of direct limit, and the claim then follows for the same reason. 

This verifies that $H_q(M|A)=0$ for $q>n$. In dimension $n$ we contemplate
the commutative diagram 
\[
\xymatrix{
\varinjlim_iH_n(M|A_i) \ar[r]^\cong \ar[d]^\cong & H_n(M|A) \ar[d] \\
\varinjlim_i\Gamma(A_i;o_M) \ar[r]^\cong & \Gamma(A;o_M) \,.
}\]
\end{proof}

\begin{proof}[Proof of Theorem \ref{thm-fundamental-class}] 
There are five steps. In describing them, we will call a subset of $M$
``Euclidean'' if it lies inside some open set homeomorphic to $\RR^n$.

\noindent
(1) $M=\RR^n$, $A$ a compact convex subset.

\noindent
(2) $M=\RR^n$, $A$ a finite union of compact convex subsets.

\noindent
(3) $M=\RR^n$, $A$ any compact subset.

\noindent
(4) $M$ arbitrary, $A$ a finite union of compact subsets of Euclidean opens.

\noindent
(5) $M$ arbitrary, $A$ an arbitrary compact subset. 

Notes on the proofs: (1) To be clear, ``convex'' implies nonempty. 
By translating $A$, we may assume that $0\in A$. The compact subset $A$
lies in some disk, and by a homothety we may assume that the disk is the
unit disk $D^n$. Then we claim that the inclusion $i:S^{n-1}\to\RR^n-A$
is a deformation retract. A retraction is given by $r(x)=x/||x||$, 
and a homotopy from $ir$ to the identity is given by 
\[
h(x,t)=\left(t+\frac{1-t}{||x||}\right)x\,.
\]

It follows that $H_q(\RR^n,\RR^n-A)\cong H_q(\RR^n,\RR^n-D^n)$ for all
$q$. This group is zero for $q>n$. In dimension $n$, note that restricting
to the origin gives an isomorphism 
$H_n(\RR^n,\RR^n-D^n)\to H_n(\RR^n,\RR^n-0)$ since 
$\RR^n-D$ is a deformation retract of $\RR^n-0$. The local system $o_{\RR^n}$ 
is trivial, since $\RR^n$ is simply connected, so restricting to the origin
gives an isomorphism $\Gamma(D^n,o_{\RR^n})\to H_n(\RR^n,\RR^n-0)$. 
This implies that $j:H_n(\RR^n,\RR^n-D^n)\to\Gamma(D^n,o_{\RR^n})$
is an isomorphism. The restriction 
$\Gamma(D^n,o_{\RR^n})\to\Gamma(A,o_{\RR^n})$ is also an isomorphism, since
$A\to D^n$ is a deformation retract. So by the commutative diagram
\[
\xymatrix{
H_n(\RR^n,\RR^n-D^n) \ar[r]^\cong \ar[d]^j & H_n(\RR^n,\RR^n-A) \ar[d]^j \\
\Gamma(D^n,o_{\RR^n}) \ar[r] & \Gamma(A,o_{\RR^n}) 
}\]
we find that $j:H_n(\RR^n,\RR^n-A)\to\Gamma(A;o_{\RR^n})$ is an isomorphism.

(2) by Lemma \ref{lem-mv}. 

(3) For each $j\geq1$, let $C_j$ be a finite subset of $A$ such that 
\[
A\subseteq \cup_{x\in C_j}B_{1/j}(x)\,.
\]
Since any intersection of convex sets is either empty or convex, 
\[
A_k=\cap_{j=1}^k\cup_{x\in C_j}B_{1/j}(x)
\]
is a union of finitely many convex sets, and since $A$ is closed
it is the intersection of this decreasing family. So the result
follows from (1), (2), and Lemma \ref{lem-lim}.

(4) by (3) and (2). 

(5) Cover $A$ by finitely many open subsets that embed in Euclidean
opens as open unit disks. Their closures then form a finite cover by closed
Euclidean disks $D_i$ in Euclidean opens $U_i$. For each $i$, 
excise the closed subset $M-U_i$ to see that
\[
H_q(M,M-A\cap D_i)\cong H_q(U_i,U_i-A\cap D_i)\cong 
H_q(\RR^n,\RR^n-A\cap D_i)\,.
\]
By (4), the theorem holds for each of these. Each intersection 
$(A\cap D_i)\cap(A\cap D_j)$ is again a compact Euclidean subset, 
so the result holds for them by excision as well. The result then
follows by (1).
\end{proof}
 







