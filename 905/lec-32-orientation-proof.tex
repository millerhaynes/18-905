\section{Proof of the orientation theorem}

Recall the theorem:

\begin{theorem}
If $M$ is a compact $n$-manifold, $H_q(M;R)=0$ for $q>n$, and
the map $j:H_n(M;R)\to\Gamma(M;o_M\otimes R)$ is an isomorphism. 
\end{theorem}

The map $j$ assigns to a homology class $c\in H_n(M)$ the section
of the orientation local system that takes on the value 
$j_x(c)\in H_n(M,M-x)$ at $x\in M$ given by restriction. 

The proof will proceed by an induction on subspaces of $M$. To make 
this induction go, I need to make two refinements. 

First, you can't expect $j$ to be surjective if $M$ isn't compact. 
Here's why. Any homology class in $M$ is represented by a chain, and
the union of images of the simplices making up that chain is a compact
subset $K$ of $M$. If $x$ is outside of $K$, then $K\subseteq M-x$ and
the restriction map factors as
\[
H_n(M)\to H_n(M,K)\to H_n(M,M-x)
\]
But since $c$ is in the image of $K$, it maps to zero in $H_n(M,K)$. 

What this shows is that $j$ lands in the module of sections with ``compact
support.''

\begin{definition} Let $E$ be a local coefficient system over a space $X$. 
The {\em support} of a section $\sigma$ is 
\[
\mathrm{supp}(\sigma)=\overline{\{x\in X:\sigma(x)\neq0\}}\,.
\]
The set of sections with compact support is a submodule 
\[
\Gamma_c(X;E)\subseteq\Gamma(X;E)\,.
\]
\end{definition}
So $j:H_n(M;R)\to\Gamma_c(M;o_M\otimes R)$.

The second refinement seems a little artificial, but it's part of the 
inductive process. Let $A\subset M$ be a closed subset. If $x\in A$
then $M-A\subseteq M-x$, and the inclusion inducing $j_x$ factors as
\[
(M,\varnothing)\hookrightarrow(M,M-A)\hookrightarrow(M,M-x)\,.
\]
Let $\Gamma(A;o_M)$ denote the module of sections of the restriction
of $o_M$ to $A$, and $\Gamma_c(A;o_M)$ the submodule of such sections
with compact support.  Then 
\[
j:H_n(M,M-A)\to\Gamma_c(A;o_M)\,.
\]
\begin{theorem} Let $M$ be any $n$-manifold and $A$ a closed subset of $M$.
Then $H_q(M,M-A;R)=0$ for $q>n$, and the map 
$j:H_n(M,M-A;R)\to\Gamma_c(A;o_M\otimes R)$ is an isomorphism.
\end{theorem}
Taking $A=M$, we find that $H_q(M;R)=0$ for $q>n$ and 
\[
j:H_n(M;R)\xrightarrow{\cong}\Gamma_c(M;o_M\otimes R)\,.
\]
This includes the theorem we stated at the outset. But it says something
if $M$ is not compact, as well. If $M$ is connected and not compact, then 
any compactly supported section must be zero somewhere, but then it is
zero everywhere by unique path lifting. So in that case $H_n(M;R)=0$. 

We will not prove this theorem for general closed subsets. 







