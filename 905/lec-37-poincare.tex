\section{Poincar\'{e} duality}

Let  $M$ be a $n$-manifold and $K$ a compact subset. 
Then by Theorem \ref{thm-fundamental-class}
$H_n(M,M-K)\xrightarrow{\cong}\Gamma(K;o_M\otimes R)$. 
An {\em orientation along $K$} is a section of $o_M\otimes R$ over $K$ that 
restricts to a generator 
of $H_n(M,M-x;R)$ for every $x\in K$. The corresponding class in 
$H_n(M,M-K;R)$ is a
{\em fundamental class along $K$}, $[M]_K$. 
We recall also the fully relative cap product pairing (in which $p+q=n$
and $L$ is a closed subset of $K$)
\[
\cap:\cHH^p(K,L;R)\otimes_R H_n(M,M-K;R)\to H_q(M-L,M-K;R)\,.
\]
We now combine all of this in the following climactic result.
\begin{theorem}[Fully relative Poincar\'e duality]
Let $M$ be an $n$-manifold and $K\supseteq L$ a pair of 
compact subsets. Assume given an $R$-orientation along $K$, 
with corresponding fundamental class $[M]_K$. With $p+q=n$, the map
\[
\cap[M]_K:\cHH^p(K,L;R)\to H_q(M-L,M-K;R)\,.
\]
is an isomorphism.
\end{theorem}
We have seen that these isomorphisms are compatible; they form the 
rungs of the commuting ladder
\[
\xymatrix{
\cdots \ar[r] & 
\cHH^{p-1}(L) \ar[r] \ar[d]^{\cap[M]_L} &
\cHH^p(K,L) \ar[r] \ar[d]^{\cap[M]_K} &
\cHH^p(K) \ar[r] \ar[d]^{\cap[M]_K} &
\cHH^p(L) \ar[r] \ar[d]^{\cap[M]_L} & \cdots \\
\cdots \ar[r] & H_{q+1}(M,M-L) \ar[r] & 
H_q(M-L,M-K) \ar[r] & H_q(M,M-K) \ar[r] &
H_q(M,M-L) \ar[r] & \cdots 
}\]
and if $M$ is compact oriented with fundamental class $[M]$,
\[
\xymatrix{
\cdots \ar[r] & \cHH^p(M,L) \ar[r] \ar[d]^{\cap[M]} &
\cHH^p(K,L) \ar[r] \ar[d]^{\cap[M]_K} &
\cHH^p(M,K) \ar[r] \ar[d]^{\cap[M]} & 
\cHH^{p+1}(M,L) \ar[r] \ar[d]^{\cap[M]} & \cdots \\
\cdots \ar[r] & H_q(M-L) \ar[r] & 
H_q(M-L,M-K) \ar[r] & H_{q-1}(M-K) \ar[r] & 
H_{q-1}(M-L) \ar[r] & \cdots
}\]
\begin{proof}
By the top ladder and the five-lemma, we may assume $L=\varnothing$; so we 
want to prove that 
\[
\cap[M]_K:\cHH^p(K;R)\to H_q(M,M-K;R)
\]
is an isomorphism. 

We will follow the same five-step process we used to prove the orientation
theorem Theorem \ref{thm-fundamental-class}.

(1) $M=\RR^n$, $K$ a convex set. 



\end{proof}
Let's point out some special cases. With $K=M$, we get: 
\begin{corollary}
Suppose that $M$ is a compact $R$-oriented $n$-manifold, and let $L$ be a 
closed subset. Then (with $p+q=n$) we have the commuting ladder whose rungs
are isomorphisms:
\[
\xymatrix{
\cdots \ar[r] & \cHH^{p-1}(L) \ar[d]^{\cap[M]_L} \ar[r] & 
\cHH^p(M,L) \ar[d]^{\cap[M]} \ar[r] & 
H^p(M) \ar[d]^{\cap[M]} \ar[r]^{\cap[M]} & 
\cHH^p(L) \ar[d]^{\cap[M]_L} \ar[r] & \cdots \\
\cdots \ar[r] & H_{q+1}(M,M-L) \ar[r] & H_q(M-L) \ar[r] & 
H_q(M) \ar[r] & H_q(M,M-L) \ar[r] & \cdots 
}\]
\end{corollary}
With $L=\varnothing$, we get:
\begin{corollary} 
Suppose that $M$ is a compact $R$-oriented $n$-manifold, and let $K$ be a 
closed subset. Then (with $p+q=n$) we have the commuting ladder whose rungs
are isomorphisms:
\[
\xymatrix{
\cdots \ar[r] & H^p(M) \ar[d]^{\cap[M]} \ar[r] & 
\cHH^p(K) \ar[d]^{\cap[M]_K} \ar[r] & \cHH^{p+1}(M,K) \ar[d]^{\cap[M]} \ar[r] &
H^{p+1}(M) \ar[d]^{\cap[M]} \ar[r] & \cdots \\
\cdots \ar[r] & H_q(M) \ar[r] & H_q(M,M-K) \ar[r] & H_{q-1}(M-K) \ar[r] & 
H_{q-1}(M) \ar[r] & \cdots
}\]
\end{corollary}
The intersection of these two special cases is:
\begin{corollary}[Poincar\'e duality] 
Let $M$ be a compact $R$-oriented $n$-manifold. Then 
\[
\cap[M]:H^p(M;R)\to H_{n-p}(M;R)
\]
is an isomorphism.
\end{corollary}

