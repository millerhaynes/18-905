\section{Poincar\'{e} duality}

Let  $M$ be a $n$-manifold and $K$ a compact subset. 
By Theorem \ref{thm-fundamental-class}
\[
H_n(M,M-K;R)\xrightarrow{\cong}\Gamma(K;o_M\otimes R)\,.
\]
An {\em orientation along $K$} is a section of $o_M\otimes R$ over $K$ that 
restricts to a generator 
of $H_n(M,M-x;R)$ for every $x\in K$. The corresponding class in 
$H_n(M,M-K;R)$ is a
{\em fundamental class along $K$}, $[M]_K$. 
We recall also the fully relative cap product pairing (in which $p+q=n$
and $L$ is a closed subset of $K$)
\[
\cap:\cHH^p(K,L;R)\otimes_R H_n(M,M-K;R)\to H_q(M-L,M-K;R)\,.
\]
We now combine all of this in the following climactic result.
\begin{theorem}[Fully relative Poincar\'e duality]
\label{thm-big-poincare}
Let $M$ be an $n$-manifold and $K\supseteq L$ a pair of 
compact subsets. Assume given an $R$-orientation along $K$, 
with corresponding fundamental class $[M]_K$. With $p+q=n$, the map
\[
\cap[M]_K:\cHH^p(K,L;R)\to H_q(M-L,M-K;R)\,.
\]
is an isomorphism.
\end{theorem}
We have seen that these isomorphisms are compatible; they form the 
rungs of the commuting ladder
\[
\xymatrix{
\cdots \ar[r] & 
\cHH^{p-1}(L) \ar[r] \ar[d]^{\cap[M]_L} &
\cHH^p(K,L) \ar[r] \ar[d]^{\cap[M]_K} &
\cHH^p(K) \ar[r] \ar[d]^{\cap[M]_K} &
\cHH^p(L) \ar[r] \ar[d]^{\cap[M]_L} & \cdots \\
\cdots \ar[r] & H_{q+1}(M,M-L) \ar[r] & 
H_q(M-L,M-K) \ar[r] & H_q(M,M-K) \ar[r] &
H_q(M,M-L) \ar[r] & \cdots 
}\]
Also, if $M$ is compact and $R$-oriented with fundamental class $[M]$ 
restricting along $K$ to $[M]_K$, we have the ladder of isomorphisms
\[
\xymatrix{
\cdots \ar[r] & \cHH^p(M,L) \ar[r] \ar[d]^{\cap[M]} &
\cHH^p(K,L) \ar[r] \ar[d]^{\cap[M]_K} &
\cHH^{p+1}(M,K) \ar[r] \ar[d]^{\cap[M]} & 
\cHH^{p+1}(M,L) \ar[r] \ar[d]^{\cap[M]} & \cdots \\
\cdots \ar[r] & H_q(M-L) \ar[r] & 
H_q(M-L,M-K) \ar[r] & H_{q-1}(M-K) \ar[r] & 
H_{q-1}(M-L) \ar[r] & \cdots
}\]

To prove this theorem, 
we will follow the same five-step process we used to prove the Orientation
Theorem \ref{thm-fundamental-class}. We have already prepared the
Mayer-Vietoris ladder for this purpose. We will also need:

\begin{lemma}
\label{lem-cech-limit}
Let $A_1\supseteq A_2\supseteq \cdots$ be a decreasing sequence of 
compact subspaces of $M$. Then 
\[
\cHH^p(A_k)\to\cHH^p(A)
\]
is an isomorphism. 
\end{lemma}
\begin{proof}
This follows from the observation that 
a direct limit of direct limits is a direct limit.
\end{proof}


\begin{proof}[Proof of Theorem \ref{thm-big-poincare}]
By the top ladder and the five-lemma, we may assume $L=\varnothing$; so we 
want to prove that 
\[
\cap[M]_K:\cHH^p(K;R)\to H_q(M,M-K;R)
\]
is an isomorphism. 

(1) $M=\RR^n$, $K$ a compact convex set. We claim that 
\[
\cHH^*(K)\xrightarrow{\cong} H^*(K)\,.
\]
For any $\epsilon>0$, let $U_\epsilon$
denote the $\epsilon$-neighborhood of $K$, 
\[
U_\epsilon=\bigcup_{x\in K}B_\epsilon(x)\,.
\]
For any $y\in U_\epsilon$ there is a closest point in $K$, since the
distance function to $y$ is continuous and bounded below on the compact set 
$K$ and so achieves its infimum. If $x',x''\in K$ are the same distance 
from $y$, then the midpoint of the segment joining $x'$ and $x''$ is closer,
but lies in $K$ since $K$ is convex. So there is a unique closest point,
$f(y)$. We let the listener check that $f:U_\epsilon\to K$ is continuous. 
It is also clear that if $i:K\to U_\epsilon$ is the inclusion then 
$i\circ f$ is homotopic to the identity on $Y$, by an affine homotopy. 

Now let $D^n$ be a disk centered at the origin and containing the compact
set $K$, and consider the commutative diagram
\[
\xymatrix{
H^p(K) \ar[rr]^{\cap[\RR^n]_K} && H_q(\RR^n,\RR^n-K) \\
H^p(D^n) \ar[u]_\cong \ar[rr] \ar[d]^\cong && 
H_q(\RR^n,\RR^n-D^n) \ar[u]_\cong \ar[d]^\cong \\
H^p(\ast) \ar[rr] && H_q(\RR^n,\RR^n-\ast) \,.
}\]
The groups are zero unless $p=0,q=n$. 
By naturality of the cap product, the bottom map is given by 
$1\mapsto1\cap[\RR^n]_\ast$, and this is $[\RR^n]_\ast$ since capping with 
1 is the identity, and this fundamental class is a generator of
$H_n(\RR^n,\RR^n-\ast)$. 

(2) $K$ a finite union of compact convex subsets of $\RR^n$. 
This follows by induction and the five lemma applied to the Mayer-Vietoris
ladder \ref{thm-mayer-vietoris-ladder}.

(3) $K$ is any compact subset of $\RR^n$. This follows as before by a limit
argument, using Lemmas \ref{lem-lim-cpt} and \ref{lem-cech-limit}.

(4) $M$ arbitrary, $K$ is a finite union of compact Euclidean subsets of $M$.
This follows from (3) and Theorem \ref{thm-mayer-vietoris-ladder}.

(5) $M$ arbitrary, $K$ an arbitrary compact subset. This follows just as in
the proof of Theorem \ref{thm-fundamental-class}. 
\end{proof}
Let's point out some special cases. With $K=M$, we get: 
\begin{corollary}
Suppose that $M$ is a compact $R$-oriented $n$-manifold, and let $L$ be a 
closed subset. Then (with $p+q=n$) we have the commuting ladder whose rungs
are isomorphisms:
\[
\xymatrix{
\cdots \ar[r] & \cHH^{p-1}(L) \ar[d]^{\cap[M]_L} \ar[r] & 
\cHH^p(M,L) \ar[d]^{\cap[M]} \ar[r] & 
H^p(M) \ar[d]^{\cap[M]} \ar[r] & 
\cHH^p(L) \ar[d]^{\cap[M]_L} \ar[r] & \cdots \\
\cdots \ar[r] & H_{q+1}(M,M-L) \ar[r] & H_q(M-L) \ar[r] & 
H_q(M) \ar[r] & H_q(M,M-L) \ar[r] & \cdots 
}\]
\end{corollary}
With $L=\varnothing$, we get:
\begin{corollary} 
Suppose that $M$ is an $n$-manifold, and let $K$ be a compact subset. 
An $R$-orientation along $K$ determines (with $p+q=n$) an isomorphism
\[
\cap[M]_K:\cHH^p(K;R)\to H_q(M,M-K;R)\,.
\]
\end{corollary}
The intersection of these two special cases is:
\begin{corollary}[Poincar\'e duality] 
Let $M$ be a compact $R$-oriented $n$-manifold. Then 
\[
\cap[M]:H^p(M;R)\to H_{n-p}(M;R)
\]
is an isomorphism.
\end{corollary}

