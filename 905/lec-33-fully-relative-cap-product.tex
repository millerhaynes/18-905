\section{$\cHH^\ast$ as a cohomology theory, and the fully relative $\cap$ product}

Let $X$ be any space, and let $K\subseteq X$ be a closed subspace.
We've defined the \v{C}ech cohomology of $K$ as the direct limit of 
$H^*(U)$ as $U$ ranges over the poset $\mathcal{U}_K$ of open neighborhoods
of $K$. This often coincides with $H^*(K)$ but will not be the same in
general. Nevertheless it behaves like a cohomology theory. To expand on
this claim, we should begin by defining a relative version. 

Suppose $L\subseteq K$ is a pair of  closed subsets of a space $X$.  Let 
$(U,V)$ be a ``neighborhood pair'' for $(K,L)$: 
\[
\begin{array}{ccc} L & \subseteq & K \\
\downsubseteq & & \downsubseteq \\
V & \subseteq & U
\end{array}
\]
These again form a directed set $\mathcal{U}_{K,L}$,
with partial order given by reverse inclusion of pairs. Then define
\[
\cHH^p(K,L)=\varinjlim_{(U,V)\in\mathcal{U}_{K,L}} H^p(U,V)\,.
\]
\begin{theorem}
Let $(K,L)$ be a closed pair in $X$. There is a long exact sequence
\begin{equation*}
\cdots\to\cHH^p(K,L)\to\cHH^p(K)\to\cHH^p(L)\xrightarrow{\delta}\cHH^{p+1}(K,L)\to\cdots
\end{equation*}
that is natural in the pair. 
\end{theorem}

Also, a form of excision holds:
\begin{theorem}[Excision]
Suppose $A,B\subseteq X$ are closed. Then the inclusion induces isomorphisms
\[
\cHH^p(A\cup B,A)\xrightarrow{\cong}\cHH^p(B,A\cap B)\,.
\] 
\end{theorem}
So \v{C}ech cohomology is better suited to closed subsets than singular
cohomology is. 

\v{C}ech cohomology appeared as the natural algebra acting on $H^*(X,X-K)$,
where $K$ is a closed subspace of $X$: 
\[
\cap:\cHH^p(K)\otimes H_n(X,X-K)\to H_q(X,X-K)\,,\quad p+q=n\,.
\]
If we fix $x_K\in H_n(X,X-K)$, then capping with $x_K$ gives a map 
\[
\cap x_K:\cHH^p(K)\to H_q(X,X-K)\,,\quad p+q=n\,. 
\]
We will be very interested in showing that this map is an isomorphism 
under certain conditions. This is a kind of duality result, comparing
cohomology and relative homology and reversing the dimensions. 
We'll try to show that such a map is an isomorphism by embedding it 
in a map of long exact sequences and using the five-lemma. 

For a start, let's think about how these maps vary as I change $K$.
So let $L$ be a closed subset of $K$, so $X-K\subseteq X-L$ and I get 
a ``restriction map''
\[
i_\ast:H_n(X,X-K)\to H_n(X,X-L)\,.
\]
Define $x_L$ as the image of $x_K$. The diagram
\begin{equation*}
\xymatrix{
\cHH^p(K)\ar[r]\ar[d]_{-\cap x_K} & \cHH^p(L) \ar[d]_{-\cap x_L} \\
H_q(X,X-K)\ar[r] & H_q(X,X-L)
}
\end{equation*}
commutes by the projection formula. This embeds in a bigger diagram:
\begin{theorem}
There is a ``fully relative'' cap product
\[
\cap:\cHH^p(K,L)\otimes H_n(X,X-K)\to H_q(X-L,X-K)\,,\quad p+q=n\,,
\]
such that for any $x_K\in H_n(X,X-K)$ the diagram
\begin{equation*}
\xymatrix{
\cdots \ar[r] & \cHH^p(K,L)\ar[r]\ar[d]_{-\cap x_{K}} & \cHH^p(K)\ar[r]\ar[d]_{-\cap x_K} & \cHH^p(L)\ar[r]^\delta\ar[d]_{-\cap x_L} & \cHH^{p+1}(K,L)\ar[r]\ar[d]_{-\cap x_{K}} & \cdots\\
	\cdots\ar[r] & H_q(X-L,X-K)\ar[r]& H_q(X,X-K)\ar[r] & H_q(X,X-L)\ar[r]^\partial & H_{q-1}(X-L,X-K)\ar[r] & \cdots
}
\end{equation*}
commutes. Here $x_L$ is $x_K$ restricted to $H_n(X,X-L)$. 
\end{theorem}
What I have to do is define a cap product of the following form (bottom row):
\begin{equation*}
\xymatrix{
	\cHH^p(K)\otimes H_n(X,X-K)\ar[r]^{\cap} & H_q(X,X-K)\\
	\cHH^p(K,L)\otimes H_n(X,X-K)\ar[u]\ar[r]^{\cap} & H_q(X-L,X-K)
}
\end{equation*}
(where $p+q=n$)



Our map $\cHH^p(K)\otimes H_n(X,X-K)\to H_q(X,X-K)$ came from $S^p(U)\otimes S_n(U,U-K)\to S_q(U,U-K)$ where $U\supseteq K$, defined via $\beta\otimes\sigma\mapsto\beta(\sigma\circ\alpha_p)\cdot(\sigma\circ\omega_q)$. I'm hoping to get:
\begin{equation*}
\xymatrix{
	S^p(U)\otimes S_n(U,U-K)\ar[r] & S_q(U,U-K)\\
	S^p(U,V)\otimes S_n(U-L)/S_n(U-K)\ar[r]\ar[u] & S_q(U-L)/S_q(U-K)\ar[u]
}
\end{equation*}
where again we have inclusions ($U,V$ open and $K,L$ closed):
\begin{equation*}
\xymatrix{K\ar@{^(->}[d] & L\ar@{^(->}[d]\ar@{^(->}[l] \\ U & V\ar@{^(->}[l]}
\end{equation*}
The bottom map $S^p(U,V)\otimes S_n(U-L)/S_n(U-K)\to S_q(U-L)/S_q(U-K)$ makes sense. We can evaluate a cochain that kills everything on $V$. This means that we can add in $S_n(V)$ to get $S^p(U,V)\otimes (S_n(U-L)+S_n(V))/S_n(U-K)\to S_q(U-L)/S_q(U-K)$ by sending $\beta\otimes\tau\mapsto 0$ where $\tau:\Delta^n\to V$. This means that the diagram:
\begin{equation*}
\xymatrix{
	S^p(U)\otimes S_n(U,U-K)\ar[r] & S_q(U,U-K)\\
	S^p(U,V)\otimes (S_n(U-L)+S_n(V))/S_n(U-K)\ar[r]\ar[u] & S_q(U-L)/S_q(U-K)\ar[u]
}
\end{equation*}
commutes. It's not that far off from where we want to go.

Now, $(U-L)\cup V=U$. I have this covering of $U$ by two open sets. In $S_n(U-L)+S_n(V)$ we're taking the sum of $n$-chains. We have a map $S_\ast(U-L)+S_\ast(V)\to S_\ast(U)$. We have already worked through this -- the locality principle! This tells us that $S_\ast(U-L)+S_\ast(V)\to S_\ast(U)$ is a homotopy equivalence. Hence we can extend our diagram:
\begin{equation*}
\xymatrix{
	S^p(U)\otimes S_n(U,U-K)\ar[r] & S_q(U,U-K)\\
	S^p(U,V)\otimes (S_n(U-L)+S_n(V))/S_n(U-K)\ar[r]\ar[d]^{\simeq}\ar[u] & S_q(U-L)/S_q(U-K)\ar[u]\\
	S^p(U,V)\otimes S_n(U)/S_n(U-K) & 
}
\end{equation*}
We want the homology of $S_n(U)/S_n(U-K)$ to approximate $H_n(X,X-K)$.
\begin{claim}
There is an isomorphism $H_n(S_\ast(U)/S_\ast(U-K))=H_n(U,U-K)\to H_n(X,X-K)$.
\end{claim}
\begin{proof}
This is exactly excision! Remember our recasting of excision in the previous lecture.
\end{proof}
This means that what we've constructed really \emph{is} what we want! We now have our large lexseq:
\begin{equation*}
\xymatrix{
	\cdots\ar[r] & \cHH^p(K,L)\ar[r]\ar@{-->}[d]_{-\cap x_{K}} & \cHH^p(K)\ar[r]\ar[d]_{-\cap x_k} & \cHH^p(L)\ar[r]^\delta\ar[d]_{-\cap x_L} & \cHH^{p+1}(K,L)\ar[r]\ar[d]_{-\cap x_{K}} & \cdots\\
	\cdots\ar[r] & H_q(X-L,X-K)\ar[r]& H_q(X,X-K)\ar[r] & H_q(X,X-L)\ar[r]^\partial & H_{q-1}(X-L,X-K)\ar[r] & \cdots\\
	\cdots\ar[r] & H_q(U-L,U-K)\ar[r]\ar[u]^{\cong,\text{ five-lemma}} & H_q(U,U-K)\ar[r]\ar[u]^{\cong,\text{ locality}} & H_q(U,U-L)\ar[r]\ar[u]^{\cong,\text{ locality}} & \cdots
}
\end{equation*}
As desired.

The diagram:
\begin{equation*}
\xymatrix{
	\cHH^p(L)\ar[r]^\delta\ar[d]^{-\cap x_L} & \cHH^{p+q}(K,L)\ar[d]^{-\cap x_K}\\
	H_q(X,X-L)\ar[r]^{\partial} & H_{q-1}(X-L,X-K)
}
\end{equation*}
says that:
\begin{equation*}
(\delta b)\cap x_k=\partial(b\cap x_L)
\end{equation*}
It's rather wonderful! You have a decreasing sequence below and an increasing one above.

I want to reformulate all of this in a more useful fashion, from Mayer-Vietoris. We had two different proofs, one from locality, and another one that we'll remind you of:
\begin{equation*}
\xymatrix{
	\cdots\ar[r] & A_n\ar[r]\ar[d] & B_n\ar[r]\ar[d]^\cong & C_n\ar[r]\ar[d] & A_{n-1}\ar[r]\ar[d] & \cdots\\
	\cdots\ar[r] & A_n^\prime\ar[r] & B_n^\prime\ar[r] & C^\prime_n\ar[r] & A^\prime_{n-1}\ar[r] & \cdots
}
\end{equation*}
then you get a lexseq:
\begin{equation*}
\cdots\to C_{n+1}\to C^\prime_{n+1}\oplus A_n\to A^\prime_n\xrightarrow{\partial} C_n\to\cdots
\end{equation*}
You can use this to prove Mayer-Vietoris -- I will do this in a special case. (This is exactly what I did in a homework assignment\footnote{Suppose $A\subseteq X$ is a subspace of $X$. Then there is a lexseq in reduced homology $\cdots\to \widetilde{ H}_n(A)\to \widetilde{ H}_n(X)\to H_n(X,A)\to\widetilde{ H}_{n-1}(A)\to\cdots$ that can be obtained by using the lexseq in homology of the sexseq $0\to\widetilde{S}_\ast(A)\to\widetilde{S}_\ast(X)\to S_\ast(X,A)\to 0$.

Now suppose $X=A\cup B$. Consider the ladder:
\begin{equation*}
\xymatrix@C=10pt{\cdots\ar[r] & H_{n+1}(A,A\cap B)\ar[r]\ar[d] & \widetilde{ H}_n(A\cap B)\ar[r]\ar[d] & \widetilde{ H}_n(A)\ar[r]\ar[d] & H_n(A,A\cap B)\ar[r]\ar[d] & \cdots\\
\cdots\ar[r] & H_{n+1}(X,B)\ar[r] & \widetilde{ H}_n(B)\ar[r] & \widetilde{ H}_n(X)\ar[r] & H_n(X,B)\ar[r] & \cdots}
\end{equation*}
The first and fourth maps as shown are isomorphisms because of excision. The lexseq from the ladder (see above) therefore yields the Mayer-Vietoris sequence $\cdots\to \widetilde{ H}_n(A\cap B)\to \widetilde{ H}_n(B)\oplus \widetilde{ H}_n(A)\to \widetilde{ H}_n(X)\to \widetilde{ H}_{n-1}(A\cap B)\to\cdots$.}!) We have a ladder of lexseqs:
\begin{equation*}
\xymatrix{
	\cdots\ar[r] & H_q(X,X-A\cup B)\ar[r]\ar[d] & H_q(X,X-A)\ar[r]\ar[d] & H_{q-1}(X-A,X-A\cup B)\ar[r]\ar[d]^{\cong,\text{ excision}} & \cdots\\
	\cdots\ar[r] & H_q(X,X-B)\ar[r] & H_q(X,X-A\cap B)\ar[r] & H_{q-1}(X-A\cap B,X-B)\ar[r] & \cdots
}
\end{equation*}
This means that (using the lexseq of the ladder) you have a lexseq:
\begin{equation*}
\cdots\to H_q(X,X-A\cup B)\to H_q(X,X-A)\oplus H_q(X,X-B)\to H_q(X,X-A\cap B)\to H_{q-1}(X,X-A\cup B)\to\cdots
\end{equation*}
This can be used to give a lexseq for \v{C}ech cohomology:
\begin{equation*}
\cdots\to \cHH^p(A\cup B)\to \cHH^p(A)\oplus \cHH^p(B)\to \cHH^p(A\cap B)\to \cHH^{p+q}(A\cup B)\to\cdots
\end{equation*}
so that we're going to get a commutative Mayer-Vietoris ladder:
\begin{theorem}
There's a ``Mayer-Vietoris'' ladder:
\begin{equation*}
\xymatrix{
	\to\cHH^p(A\cup B)\ar[r]\ar[d]^{-\cap x_{A\cup B}} & \cHH^p(A)\oplus \cHH^p(B)\ar[r]\ar[d]^{(-\cap x_A)\oplus -\cap x_B} & \cHH^p(A\cap B)\ar[r]\ar[d] & \cHH^{p+q}(A\cup B)\ar[r]\ar[d] & \cdots\\
	\to H_q(X,X-A\cup B)\ar[r]& H_q(X,X-A)\oplus H_q(X,X-B)\ar[r]& H_q(X,X-A\cap B)\ar[r]& H_{q-1}(X,X-A\cup B)\ar[r]&\cdots
}
\end{equation*}
where I have four cohomology classes $x_{A \cup B},x_A,x_B,x_{A\cap B}$ that commute in:
\begin{equation*}
\xymatrix{
	 & H_n(X,X-A)\ar[dr] & \\
	H_n(X,X-A\cap B)\ar[ur]\ar[dr] & & H_n(X,X-A\cap B)\\
	 & H_n(X,X-B)\ar[ur]
}
\end{equation*}
\end{theorem}
This is the most complicated blackboard for the rest of the course. Also xymatrix is not compiling properly because the diagram is too big!
