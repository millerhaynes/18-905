\section{Cup product, continued}
We have constructed an explicit map $S^p(X)\otimes S^q(Y)\xrightarrow{\times} S^{p+q}(Y)$ via:
\begin{equation*}
(f\times g)(\sigma)=(-1)^{pq}f(\sigma_1\circ\alpha_p)g(\sigma_2\circ\omega_q)
\end{equation*}
where $\alpha_p:\Delta^p\to\Delta^{p+q}$ sends $i$ to $i$ for 
$0\leq i\leq p$ and $\omega_q:\Delta^q\to\Delta^{p+q}$ sends $j$ to $j+p$ 
for  $0\leq j\leq q$. This is a chain map; it induces a ``cross product''
$ H^p(X)\otimes H^q(Y)\to H_{p+q}(X\times Y)$, and, by composing with the
map induced by the diagonal embedding, the ``cup product''
\[
\cup:H^p(X)\otimes H^q(X)\to H^{p+q}(X)\,.
\]
We formalize the structure that this product imposes on cohomology.
\begin{definition}
Let $R$ be a commutative ring. A \emph{graded $R$-algebra} is a graded $R$-module $\ldots,A_{-1},A_0, A_1,A_2,\ldots$ equipped with maps $A_p\otimes_R A_q\to A_{p+q}$ and a map $\eta:R\to A_0$ that make the following diagram commute.
\[
\xymatrix{
A_p\otimes_R R \ar[r]^{1\otimes\eta} \ar[dr]^= &A_p\otimes_R A_0 \ar[d] &
A_0\otimes_R A_q \ar[d] & R\otimes_R A_q \ar[l]_{\eta\otimes1} \ar[dl]_= \\
& A_p & A_q 
}\]
\begin{equation*}
\xymatrix{
A_p\otimes_R (A_q\otimes_R A_r)\ar[r]\ar[d] & A_p\otimes_R A_{q+r}\ar[d]\\
A_{p+q}\otimes_R A_r \ar[r] & A_{p+q+r}
}
\end{equation*}
A graded $R$-algebra $A$ is {\em commutative} if the following diagram commutes:
\begin{equation*}
\xymatrix{
	A_p\otimes_R A_q\ar[rr]^{\tau}\ar[dr] & & A_q\otimes_R A_p \ar[dl]\\
	 & A_{p+q} & 
}
\end{equation*}
where $\tau(x\otimes y)=(-1)^{pq}y\otimes x$. 
\end{definition}
We claim that $ H^\ast(X;R)$ forms a commutative graded $R$-algebra under the cup product. This is nontrivial. On the cochain level, this is clearly not graded commutative. We're going to have to work hard -- in fact, so hard that you're going to do some of it for homework. What needs to be checked is that the following 
diagram commutes up to natural chain homotopy.
\begin{equation*}
\xymatrix{S_\ast(X\times Y)\ar[r]^{T_\ast}\ar[d]_{\alpha_{X,Y}} & S_\ast(Y\times X)\ar[d]^{\alpha_{Y,X}}\\
S_\ast(X)\otimes_R S_\ast(Y)\ar[r]^{\tau} & S_\ast(Y)\otimes_R S_\ast(X)}
\end{equation*}
Acyclic models helps us prove things like this.

You might hope that there is some way to produce a commutative product on 
a chain complex modeling $H^\ast(X)$. With coefficients in $\QQ$, this is 
possible, by a construction due to Dennis Sullivan. 
With coefficients in a field of nonzero characteristic, it is not possible.
Steenrod operations provide the obstruction.

My goal now is to compute the cohomology algebras of some spaces. 
Some spaces are
easy! There is no choice for the product structure on $H^*(S^n)$, for example.
(When $n=0$, we get a free module of rank 2 in dimension 0. This admits
a variety of commutative algebra structures; but we have already seen that
$H^0(S_0)=\Z\times\Z$ as an algebra.) Maybe the next thing to try is a 
product of spheres. More generally, we should ask whether there is an 
algebra structure on $H^\ast(X)\otimes H^\ast(Y)$ making the cross product an
algebra map. If $A$ and $B$ are two graded algebras, there {\em is} a natural
algebra structure on $A\otimes B$, given by $1=1\otimes1$ and
\[
(a'\otimes b')(a\otimes b)=(-1)^{|b^\prime|\cdot|a|}a'a\otimes b'b\,.
\]
If $A$ and $B$ are commutative, then so is $A\otimes B$ with this algebra
structure. 
\begin{prop} The cohomology cross product
\[
\times:H^\ast(X)\otimes H^\ast(Y)\rightarrow H^\ast(X\times Y)
\]
is an $R$-algebra homomorphism.
\end{prop}
\begin{proof}
I have diagonal maps $\Delta_X:X\to X\times X$ and $\Delta_Y:Y\to Y\times Y$. 
The diagonal on $X\times Y$ factors as
\[
\xymatrix{
X\times Y \ar[rr]^{\Delta_{X\times Y}} \ar[dr]^{\Delta_X\times\Delta_Y} &&
X\times Y\times X\times Y \\
& X\times X\times Y\times Y \ar[ur]^{1\times T\times1}\,.
}\]
Let $\alpha_1,\alpha_2\in H^\ast(X)$ and $\beta_1,\beta_2\in H^\ast(Y)$. Then $\alpha_1\times \beta_1,\alpha_2\times\beta_2\in H^\ast(X\times Y)$, and I want to calculate $(\alpha_1\times\beta_1)\cup(\alpha_2\times\beta_2)$. Let's see:
\begin{align*}
(\alpha_1\times\beta_1)\cup(\alpha_2\times\beta_2) & = \Delta_{X\times Y}^\ast(\alpha_1\times\beta_1\times\alpha_2\times\beta_2)\\
& = (\Delta_X\times\Delta_Y)^\ast(1\times T\times 1)^\ast(\alpha_1\times\beta_1\times\alpha_2\times\beta_2)\\
& = (\Delta_X\times\Delta_Y)^\ast(\alpha_1\times T^\ast(\beta_1\times\alpha_2)\times\beta_2)\\
& = (-1)^{|\alpha_2|\cdot|\beta_1|}(\Delta_X\times\Delta_Y)^\ast(\alpha_1\times\alpha_2\times\beta_1\times\beta_2)
\end{align*}
Now, I have a diagram:
\begin{equation*}
\xymatrix{
	 H^\ast(X\times Y) & \ar[l]^{\times_{X\times Y}} H^\ast(X)\otimes_R H^\ast(Y)\\
	 H^\ast(X\times X\times Y\times Y)\ar[u]^{(\Delta_X\times\Delta_Y)^\ast} & H^\ast(X\times X)\otimes H^\ast(Y\times Y)\ar[l]_{\times_{X\times X,Y\times Y}}\ar[u]^{\Delta_X^\times\otimes\Delta_Y^\ast}
}
\end{equation*}
This diagram commutes because the cross product is natural. We learn:
\begin{align*}
(\alpha_1\times\beta_1)\cup(\alpha_2\times\beta_2) & = (-1)^{|\alpha_2|\cdot|\beta_1|}(\Delta_X\times\Delta_Y)^\ast(\alpha_1\times\alpha_2\times\beta_1\times\beta_2)\\
& = (-1)^{|\alpha_2|\cdot|\beta_1|}(\alpha_1\cup\alpha_2)\times(\beta_1\cup\beta_2)\,.
\end{align*}
That's exactly what we wanted.
\end{proof}
We will see later, in Theorem \ref{thm-coh-kunneth}, that the cross product
map is often an isomorphism. 

\begin{example}
How about $H^\ast(S^p\times S^q)$? I'll assume that $p$ and $q$ are both
positive, and leave the other cases to you. The K\"unneth theorem guarantees
that $\times:H^*(S^p)\otimes H^*(S^q)\to H^*(S^p\times S^q)$ is an isomorphism.
Write $\alpha$ for a generator of $S^p$ and $\beta$ for a generator of 
$S^q$; and use the same notations for the pullbacks of these elements to
$S^p\times S^q$ under the projections. Then 
\[
H^*(S^p\times S^q)=\Z\langle1,\alpha,\beta,\alpha\cup\beta\rangle\,,
\]
and
\[
\alpha^2=0\,,\quad\beta^2=0\,,\quad\alpha\beta=(-1)^{pq}\beta\alpha\,.
\]

This calculation is useful!
\begin{corollary} Let $p,q>0$.
Any map $S^{p+q}\to S^p\times S^q$ induces the zero map in $H^{p+q}(-)$. 
\end{corollary}
\begin{proof}
Let $f:S^{p+q}\to S^p\times S^q$ be such a map. It induces an algebra map
$f^*:H^*(S^p\times S^q)\to H^*(S^{p+q})$. This map must kill $\alpha$ and
$\beta$, for degree reasons. But then it also kills their product, since
$f^*$ is multiplicative. 
\end{proof}
The space $S^p\vee S^q\vee S^{p+q}$ has the same cohomology groups as
$S^p\times S^q$. Both are built as CW complexes with cells in dimensions
$0, p, q$, and $p+q$. But they are not homotopy equivalent. We can see this
now because there {\em is} a map $S^p\vee S^q\vee S^{p+q}\to S^{p+q}$
inducing an {\em isomorphism} in $H^{p+q}(-)$, namely, the map that pinches
the other two factors to the basepoint. 
\end{example}
