\section{Products in cohomology}

We'll talk about the cohomology cross product first. Actually, the first
step is to produce a map on chains that goes in the reverse direction from
the cross product we constructed in Lecture 7. 

\begin{construction}
For each pair of natural numbers $p,q$, we will define a natural homomorphism 
\[
\alpha: S_{p+q}(X\times Y)\to S_p(X)\otimes S_q(Y)\,.
\]
It suffices to define this on simplices, so let 
$\sigma:\Delta^{p+q}\to X\times Y$ be a singular $(p+q)$-simplex in the product. I can write $\sigma=\begin{pmatrix}\sigma_1 \\ \sigma_2\end{pmatrix}$ where $\sigma_1:\Delta^{p+q}\to X$ and $\sigma_2:\Delta^{p+q}\to Y$. I have to produce
a $p$-simplex in $X$ and a $q$-simplex in $Y$. 
First define two maps in the simplex category:
 the ``front face''  $\alpha_p:[p]\to[p+q]$, sending $i$ to $i$ for $0\leq i\leq p$, and the ``back face'' $\omega_q:[q]\to[p+q]$, sending $j$ to $j+p$ for $0\leq j\leq q$. Use the same symbols for the affine 
extensions to maps $\Delta^p\to \Delta^{p+q}$ and $\Delta^q\to\Delta^{p+q}$. 
Now let 
\[
\alpha(\sigma)=\sigma_1\circ\alpha_p\otimes\sigma_2\circ\omega_q\,.
\]
\end{construction}
This seems like a very random construction; but it works! It's named after two
great early algebraic topologists, Alexander and Whitney.
For homework, you will show that these maps assemble into a chain map
\[
\alpha:S_\ast(X\times Y)\to S_\ast(X)\otimes S_\ast(Y)\,.
\]

This works over any ring $R$. To get a map in cohomology, we should form 
\[
S^p(X;R)\otimes_RS^q(Y;R)\to
\Hom_R(S_p(X;R)\otimes_RS_q(Y;R),R)\xrightarrow{\alpha^*}
\Hom_R(S_{p+q}(X\times Y;R),R)=S^{p+q}(X\times Y)\,.
\]
The first map goes like this: Given chain complexes $C_\ast$ and $D_\ast$, 
we can consider the dual cochain complexes $\Hom_R(C_\ast,R)$ and
$\Hom_R(D_\ast,R)$, and construct a chain map 
\[
\Hom_R(C_\ast,R)\otimes_R\Hom_R(D_\ast,R)\to\Hom_R(C_\ast\otimes_R D_\ast,R)
\]
by 
\begin{equation*}
f\otimes g\mapsto\begin{cases}
(x\otimes y\mapsto (-1)^{pq}f(x)g(y)) & |x|=|f|=p, |y|=|g|=q\\
0 & \text{otherwise}.
\end{cases}
\end{equation*}
Again, I leave it to you to check that this is a chain map. 

Altogether, we have constructed a natural chain map
\[
\times:S^p(X)\otimes S^q(Y)\to S^{p+q}(X\times Y)
\]
From this, we get a homomorphism
\[
H^\ast(S^\ast(X)\otimes S^\ast(Y))\to H^\ast(X\times Y)\,.
\]
I'm not quite done! As in the K\"unneth theorem, there is an evident natural 
map 
\[
 H^\ast(X)\otimes H^\ast(Y)\to H^\ast(S^\ast(X)\otimes S^\ast(Y))\,.
\] 
The composite
\[
\times:H^\ast(X)\otimes H^\ast(Y)\to H^\ast(S^\ast(X)\otimes S^\ast(Y))\to H^\ast(X\times Y)
\]
 is the {\em cohomology cross product}.

It's not very easy to do computations with this, directly. We'll find indirect means. Let me make some points about this construction, though.
\begin{definition}
The {\em cup product} is the map obtained by taking $X=Y$ and composing with
the map induced by the diagonal $\Delta:X\to X\times X$:
\[
\cup:H^p(X)\otimes H^q(X)\xrightarrow{\times} H^{p+q}(X\times X)\xrightarrow{\Delta^\ast} H^{p+q}(X),.
\]
\end{definition}
These definitions make good sense with any ring for coefficients.

Let's explore this definition in dimension zero. 
I claim that $ H^0(X;R)\cong\Map(\pi_0(X),R)$ as rings. When $p=q=0$, both $\alpha_0$ and $\omega_0$ are the identity maps, so we are just forming the 
pointwise product of functions. 

There's a distinguished element in $H^0(X)$, namely the the function $\pi_0(X)\to R$ that takes on the value 1 on every path component. 
This is the identity for the cup product. This comes out because when $p=0$ in our above story, then $\alpha_0$ is just including the $0$-simplex, and $\omega_q$ is the identity. 

The cross product is also associative, even on the chain level. 
\begin{prop}
Let $f\in S^p(X)$, $g\in S^q(Y)$, and $h\in S^r(Z)$, and let 
$\sigma:\Delta^{p+q+r}\to X\times Y\times Z$ be any simplex. Then
\[
((f\times g)\times h)(\sigma)=(f\times(g\times h))(\sigma)\,.
\]
\end{prop}
\begin{proof}
Write $\sigma_{12}$ for the composite of $\sigma$ with the projection map
$X\times Y\times Z\to X\times Y$, and so on. Then
\[
((f\times g)\times h)(\sigma)=(-1)^{(p+q)r}
(f\times g)(\sigma_{12}\circ\alpha_{p+q})h(\sigma_3\circ\omega_r)\,.
\]
But 
\[(f\times g)(\sigma_{12}\circ\alpha_{p+q})=(-1)^{pq}
f(\sigma_1\circ\alpha_p)g(\sigma_2\circ\mu_q)\,,
\] 
where $\mu_q$ is the ``middle face,'' sending $\ell$ to $\ell+p$ for
$0\leq\ell\leq q$. In other words, 
\[
((f\times g)\times h)(\sigma)=(-1)^{pq+qr+rp}
f(\sigma_1\circ\alpha_p)g(\sigma_2\circ\mu_q)h(\sigma_3\circ\omega_r)\,.
\]
I've used associativity of the ring. But you get exactly the same thing when
you expand $(f\times(g\times h))(\sigma)$, so the cross product is associative.
\end{proof}

Of course the diagonal map is ``associative,'' too, and we find that the
cup product is associative:
\[
(\alpha\cup\beta)\cup\gamma=\alpha\cup(\beta\cup\gamma)\,.
\]

But this product is obviously not commutative on the level of cochains. 
It treats the two maps completely differently. But we have ways of 
dealing with this. You will show for homework that the 
method of acyclic models shows that 
\[
\alpha\cup\beta=(-1)^{|\alpha|\cdot|\beta|}\beta\cup\alpha\,.
\]

So $ H^\ast(X;R)$ forms a \emph{commutative graded $R$-algebra}.
