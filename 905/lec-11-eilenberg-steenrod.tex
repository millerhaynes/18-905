\section{The Eilenberg Steenrod axioms and the locality principle}


Before we proceed to prove the excision theorem, let's review the properties ofsingular homology as we have developed them. They are captured by a set of axioms, due to Sammy Eilenberg and Norman Steenrod \cite{eilenberg-steenrod}. 
\begin{definition}
A {\em homology theory} (on $\mathbf{Top}$) is:
\begin{itemize}
\item a sequence of functors $h_n:\mathbf{Top}_2\to\mathbf{Ab}$ for all $n\in\ZZ$ and
\item a sequence of natural transformations $\partial:h_n(X,A)\to h_{n-1}(A,\varnothing)$
\end{itemize}
such that:
\begin{itemize}
\item If $f_0,f_1:(X,A)\to (Y,B)$ are homotopic, then $f_{0\ast}=f_{1\ast}:h_n(X,A)\to h_n(Y,B)$.
\item Excisions induce isomorphisms.
\item For any pair $(X,A)$, the sequence 
\begin{equation*}
\cdots\to h_{q+1}(X,A)\xrightarrow{\partial}h_q(A)\to h_q(X)\to h_q(X,A)\xrightarrow{\partial}\cdots
\end{equation*}
is exact, where we have written $h_q(X)$ for $h_q(X,\varnothing)$.
\item (The dimension axiom): The group $h_n(\ast)$ is nonzero only for $n=0$. 
\end{itemize}
\end{definition}
We add the following ``Milnor axiom'' \cite{milnor}
to our definition. To state it,
let $I$ be a set and suppose that for each $i\in I$ we have a space $X_i$. We can form their disjoint union or {\em coproduct} $\coprod X_i$. The inclusion maps $X_i\to\coprod X_i$ induce maps $h_n(X_i)\to h_n(\coprod X_i)$, and these in turn induce a map from the direct sum, or coproduct in $\Ab$:
\[
\alpha:\bigoplus_{i\in I} h_n(X_i)\to h_n\left(\coprod_{i\in I} X_i\right)\,.
\]
Then:
\begin{itemize}
\item The map $\alpha$ is an isomorphism for all $n$.
\end{itemize}

Ordinary singular homology satisfies these, with $h_0(\ast)=\ZZ$. We will soon add ``coefficents'' to homology, producing a homology theory whose value on a point is any prescribed abelian group. In later developments, it 
emerges that the dimension axiom is rather like the parallel postulate
in Euclidean geometry: it's ``obvious,'' but, as it turns out, the remaining
axioms accomodate extremely interesting alternatives, in which
$h_n(*)$ is nonzero for infinitely many values of $n$ (both positive and negative). 

\bigskip
Excision is a statement that homology is ``localizable.'' To make this precise, we need some definitions. 

\begin{definition}
Let $X$ be a topological space. A family $\mathscr{A}$ of subsets of $X$ is a
{\em cover} if $X$ is the union of the interiors of elements of $\mathscr{A}$. 
\end{definition}
\begin{definition}
Let ${\mathscr{A}}$ be a cover of $X$. An $n$-simplex $\sigma$ is ${\mathscr{A}}$-{\em small} if there is $A\in \mathscr{A}$ such that the image of $\sigma$ is entirely in $A$.
\end{definition}
Notice that if $\sigma:\Delta^n\to X$ is ${\mathscr{A}}$-small, then so is $d_i\sigma$; in fact, for any simplicial operator $\phi$, $\phi^*\sigma$ is again $\mathscr{A}$-small. Let's denote by $\Sin^{\mathscr{A}}_*(X)$ the graded set of ${\mathscr{A}}$-small simplices. This us a sub-simplicial set of $\Sin_*(X)$.
Applying the free abelian group functor, we get the subcomplex 
\[
S^{\mathscr{A}}_\ast(X)
\]
of $\mathscr{A}$-{\em small singular chains}. Write $H^{\mathscr{A}}_*(X)$ for its
homology.
\begin{theorem}[The locality principle]
The inclusion $S^\mathscr{A}_\ast(X)\subseteq S_\ast(X)$ induces an isomorphism in homology, $H^\mathscr{A}_\ast(X)\xrightarrow{\cong}H_\ast(X)$. 
\end{theorem}
This will take a little time to prove. Let's see right now how it implies excision.

Suppose $X\supset A\supset U$ is excisive, so that 
$\overline U\subseteq\mathrm{Int}A$, or $\mathrm{Int}(X-U)\cup\mathrm{Int}A=X$.
This if we let $B=X-U$, then $\mathscr{A}=\{A,B\}$ is a cover of $X$. 
Rewriting in terms of $B$,
\[
(X-U,A-U)=(B,A\cap B)\,,
\]
so we aim to show that 
\[
S_*(B,A\cap B)\rightarrow S_*(X,A)
\]
induces an isomorphism in homology. We have the following diagram of chain complexes with exact rows:
\[
\xymatrix{
0 \ar[r] & S_*(A) \ar[d]^= \ar[r] & S^{\mathscr{A}}_*(X) \ar[d]\ar[r] &
S^{\mathscr{A}}_*(X)/S_*(A) \ar[d]\ar[r] & 0 \\
0 \ar[r] & S_*(A) \ar[r] & S_*(X) \ar[r] & S_*(X,A) \ar[r] & 0 
}\]
The middle vertical induces an isomorphism in homology by the locality principle, so the homology long exact sequences combine with the five-lemma to show that the right hand vertical is also a homology isomorphism. But 
\[
S^{\mathscr{A}}_n(X)=S_n(A)+S_n(B)\subseteq S_n(X)\,
\]
and a simple result about abelian groups provides an isomorphism
\[
\frac{S_n(B)}{S_n(A\cap B)}=
\frac{S_n(B)}{S_n(A)\cap S_n(B)}\xrightarrow{\cong}
\frac{S_n(A)+S_n(B)}{S_n(A)}
=\frac{S_n^{\mathscr{A}}(X)}{S_n(A)}\,,
\]
so excision follows.

This case of a cover with two elements leads to another expression of 
excision, known as the ``Mayer-Vietoris sequence.'' In describing it we will
use the following notation for the various inclusion.
\begin{equation*}
\xymatrix{A\cap B\ar[r]^{j_1}\ar[d]^{j_2} & A\ar[d]^{i_1}\\
B\ar[r]_{i_2} & X}
\end{equation*}
\begin{theorem}[Mayer-Vietoris] 
Assume that $\mathscr{A}=\{A,B\}$ is a cover of $X$. There are natural
maps $\partial:H_n(X)\to H_{n-1}(A\cap B)$ such that the sequence
\[
\xymatrix{
& \cdots \ar[r]^\beta & H_{n+1}(X) \ar[dll]_\partial \\
H_n(A\cap B) \ar[r]^\alpha &
H_n(A)\oplus H_n(B) \ar[r]^\beta & H_n(X) \ar[dll]_\partial \\
H_{n-1}(A\cap B) \ar[r]^\alpha & \cdots 
}\]
is exact, where 
\[ 
\alpha=\left[\begin{array}{c}j_{1*}\\-j_{2*}\end{array}\right]\,,\quad
\beta=[\,i_{1*}\quad i_{2*}\,]\,.
\]
\end{theorem}
\begin{proof}
This is the homology long exact sequence associated to the short exact sequence
of chain complexes
\[
0\to S_*(A\cap B)\xrightarrow{\alpha}S_*(A)\oplus S_*(B)\xrightarrow{\beta}
S^{\mathscr{A}}_*(X)\to0\,,
\]
combined with the locality principle.
\end{proof}

The Mayer-Vietoris theorem follows from excision as well, via
the following simple observation.
Suppose we have a map of long exact sequences
\begin{equation*}
\xymatrix{
\cdots\ar[r] & C'_{n+1} \ar[r]^k\ar[d]^h & A'_n\ar[r]\ar[d]^f & 
B'_n\ar[r]\ar[d]^\cong & C'_n\ar[r]\ar[d]^h &  \cdots\\
\cdots\ar[r] & C_{n+1} \ar[r]^k & A_n \ar[r] & B_n \ar[r] & C_n \ar[r] & \cdots
}
\end{equation*}
in which every third arrow is an isomorphism as indicated.
Define a map
\[
\partial:A_n\to B_n\xleftarrow{\cong} B'_n\to C'_n\,.
\]
An easy diagram chase shows:
\begin{lemma} 
\label{lem-ladder}
The sequence 
\begin{equation*}
\cdots\longrightarrow 
C'_{n+1}\xrightarrow{\left[\begin{array}{c}h\\-k\end{array}\right]}
C_{n+1}\oplus A'_n\xrightarrow{\left[\begin{array}{cc}k&f\end{array}\right]}
A_n\xrightarrow{\,\,\partial\,\,} C'_n\longrightarrow\cdots
\end{equation*}
is exact.
\end{lemma}
To get the Mayer-Vietoris sequence, let $\{A,B\}$ be a cover of $X$ 
and apply the lemma to 
\[
\xymatrix{
\cdots \ar[r] & H_n(A\cap B) \ar[d] \ar[r] & H_n(B)  \ar[d] \ar[r] &
H_n(B,A\cap B) \ar[d]^\cong \ar[r] & H_{n-1}(A\cap B) \ar[d] \ar[r] &
H_{n-1}(B) \ar[d] \ar[r] & \cdots \\
\cdots \ar[r] & H_n(A) \ar[r] & H_n(X) \ar[r] & H_n(X,A) \ar[r] & 
H_{n-1}(A) \ar[r] & H_{n-1}(X) \ar[r] & \cdots\,.
}\]

