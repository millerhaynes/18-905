\section{Local coefficients and orientations}

The fact that a manifold is locally Euclidean puts surprising constraints on
its cohomology, captured in the statement of Poincar\'e duality.
To understand how this comes about, we have to find ways to promote
{\em local information} -- like the existence of Euclidean neighborhoods --
to {\em global information} -- like restrictions on the structure of the
cohomology. Today we'll study the notion of an orientation, which is the first 
link between local and global.

The local-to-global device relevant to this is the notion of a ``local
coefficient system,'' which is based on the more primitive notion of a 
covering space. We summarize that theory, since it is a prerequisite 
of this course.

\begin{definition}
A continuous map $p:E\to B$ is a {\em covering space} if \\
(1) every point pre-image is a discrete subspace of $E$, and \\
(2) every $b\in B$ has a neighborhood $V$ admitting a map $p^{-1}(V)\to p^{-1}(b)$ such that the induced map
\[
\xymatrix{
p^{-1}(V) \ar[rr]^\cong \ar[dr]^p && V\times p^{-1}(b) \ar[dl]_{\pr_1} \\
& B
}\]
is a homeomorphism.
\end{definition}
The space $B$ is the ``base,'' $E$ the ``total space.''

\begin{example} 
A trivial example is given by the projection map $\pr_1:B\times F\to B$ where
$F$ is discrete. A covering space of this form is said to be {\em trivial},
so the covering space condition can be rephrased as ``locally trivial.'' 

The first interesting example is the projection map
$S^n\to \RP^n$ obtained by identifying antipodal maps on the sphere. 
\end{example}

This example generalizes in the following way. 
\begin{definition} An action of a group $\pi$ on a space $X$ is
{\em principal} or {\em totally discontinuous} (terrible language,
since we are certainly assuming that every group element acts by
homeomorphisms) provided every element $x\in X$ has a neighborhood
$U$ such that the only time $U$ and $gU$ intersect is when $g=1$. 
\end{definition}
This is a strong form of ``freeness'' of the action. It is precisely
what is needed to guarantee:
\begin{lemma}
If $\pi$ acts principally on $X$ then the orbit projection map
$X\to \pi\backslash X$ is a covering space.
\end{lemma} 

It is not hard to use local triviality to prove the following:
\begin{theorem}[Unique path lifting]
Let $p:E\to B$ be a covering space, and $\omega:I\to B$ a path in the base.
For any $e\in E$ such that $p(e)=\omega(0)$, there is a unique path 
$\widetilde\omega:I\to E$ in $E$ such that $p\widetilde\omega=\omega$ and
$\\widetilde\omega(0)=e$. 
\end{theorem}

This leads to a right action of $\pi_1(B,b)$ on $F=p^{-1}(b)$: Represent an 
element of $\pi_1(B)$ by a path $\omega$; for an element $e\in p^{-1}(b)$
let $\widetilde\omega$ be the lift of $\omega$ with $\widetilde\omega(0)=e$;
and define 
\[
e\cdot[\omega]=\widetilde\omega(1)\in E\,.
\]
This element lies in $F$ because $\omega$ was a {\em loop}, ending at $b$. 
One must check that this action by $[\omega]\in\pi_1(B,b)$ does not depend 
upon the choice of representative $\omega$, and that we do indeed get an 
{\em action}:
\[
e\cdot(ab)=(e\cdot a)\cdot b\,,\quad e\cdot1=e\,.
\]

Given a principal $\pi$-action on $X$, with orbit space $B$, 
we can more than just form the orbit space!
If we also have a right action of $\pi$ on a set $F$, we can form a new
covering space over $B$ with $F$ as ``generic'' fiber. 
Write $F\times_\pi X$ for the quotient of the product space $F\times X$ by 
the equivalence relation
\[
(s,gx)\sim(sg,x)\,,\quad g\in\pi\,.
\]
The composite projection $F\times X\to X\to B$ factors through a map
$F\times_\pi X\to B$, which is easily seen to be a covering space. 
Any element $x\in X$ determines a homeomorphism 
\[
F\to p^{-1}p(x)\,\,\hbox{by}\,\, s\mapsto[s,x]\,.
\]
Of course $\ast\times_\pi X=B$, and if we let $\pi$ act on itself by
right translation, $\pi\times_\pi X=X$. 

Covering spaces of a fixed space $B$ form a category $\mathbf{Cov}_B$,
in which a morphism $E'\to E$ is ``covering transformation,'' that is,
a map $f:E'\to E$ making
\[
\xymatrix{
E' \ar[rr]^f \ar[dr] && E \ar[dl] \\
& B
}\]
commute. 
Sending $p:E\to B$ to $p^{-1}(b)$ with its action by $\pi_1(B,b)$ gives
a functor
\[
\mathbf{Cov}_B\to \mathbf{Set}\mathrm{-}\pi_1(B,b)
\]
to the category of right actions of $\pi_1(B,b)$ on sets. For connected
spaces, this is usually an equivalence of categories. The technical
assumption required is this: A space $B$ is {\em semilocally simply 
connected} if is path connected and for every point $b$ and every 
neighborhood $U$ of $b$, there exists a smaller neighborhood $V$ 
such that $\pi_1(V,b)\to\pi_1(X,b)$ is trivial. This is a very weak
condition! 

\begin{theorem} Assume that $B$ is semi-locally simply connected.
Then the functor 
$\mathbf{Cov}_B\to\mathbf{Set}\mathrm{-}\pi_1(B,b)$ is an equivalence of 
categories. 
\end{theorem}

This is another one of those perfect theorems in algebraic topology! 

The covering space corresponding under this equivalence to the translation
action of $\pi_1(B,b)$ on itself is the {\em universal cover} of $B$,
denoted by $\widetilde B\to B$. Since the automorphism group of $\pi$ as
a right $\pi$-set is $\pi$ (acting by left translation), the automorphism
group of $\widetilde B\to B$ as a covering space of $B$ is $\pi_1(B,b)$. 
This action is principal, and the covering space corresonding to 
a $\pi_1(B,b)$-set $S$ is given by the balanced product 
$S\times_{\pi_1(B,b)}\widetilde B$. 

Covering spaces come up naturally in our study of topological manifolds. 
For any space $X$, we can probe the structure of $X$ in the neighborhood
of $x\in X$ by studying $H_*(X,X-x)$. By excision, this group depends only
on the structure of $X$ ``locally at $x$'': For any neighborhood $U$ of $x$,
excising the complement of $U$ gives an isomorphism
\[
H_*(U,U-x)\xrightarrow{\cong} H_*(X,X-x)\,.
\]
The graded $R$-module $H_*(X,X-x;R)$ is the {\em local homology of $X$ at}
$x$. 

When the space is an $n$-manifold -- let's write $M$ for it -- the local
homology is very simple. It's nonzero only in dimension $n$. This has a
nice immediate consequence, by the way: there is a well-defined 
locally constant function
$\dim:M\to\NN$, sending $x$ to the dimension in which $H_*(M,M-x)$ is 
nontrivial. For an $n$-manifold, it's the constant function with value $n$.

In fact the whole family of homology groups $H_n(M,M-x)$ 
is ``locally constant.'' 
This is captured in the statement that taken together, as $x$ varies over
$M$, they constitute a covering space over $M$. So begin by defining
\[
o_M=\coprod_{x\in M}H_n(M,M-x)
\]
as sets.
There is an evident projection map $p:o_M\to M$. We aim to put a topology on
$o_M$ with the property that this map is a covering space. This will use
an important map $j_{A,x}$, defined for any closed set $A\subseteq M$ 
and $x\in A$ as the map induced by an inclusion of pairs: 
\[
j_x:H_n(M,M-A)\to H_n(M,M-x)
\]
Define a basis of opens $V_{U,x,\alpha}$ in $o_M$ 
indexed by triples $(U,x,\alpha)$ where $U$ is 
open in $M$, $x\in U$, and $\alpha\in H_n(M,M-\overline U)$:
\[
V_{U,x,\alpha}=\{j_x(\alpha):x\in U\}\,.
\]
Each $\alpha\in H_n(M,M-\overline U)$ thus defines a ``sheet'' of $o_M$
over $U$. We leave it to you to check that this is indeed a covering space. 

This covering space has more structure: each fiber is an abelian group,
an infinite cyclic abelian group. These structures vary continuously 
as you move from one fiber to another. To illuminate this structure,
observe that the category $\mathbf{Cov}_B$ has finite products; they are
given by the fiber product or pullback, $E'\times_BE\to B$. The empty
product is the terminal object, $B\to B$. This lets us define an ``abelian
group object'' in $\mathbf{Cov}_B$; it's an object $E\to B$ together with
maps $E\times_BE\to E$ and $B\to E$ over $B$, satisfying some evident
conditions that are equivalent to requiring that they render each fiber
an abelian group. If you have a ring around you can also ask for a map
$(B\times R)\times_BE\to E$ making each fiber an $R$-module. 

The structure we  have defined is a {\em local coefficient system} (of 
$R$-modules). We already have an example; if $M$ is an $n$-manifold, 
we have the {\em orientation local system} $o_M$ over $M$. 

It's useful to allow coefficients in a commutative ring $R$; so denote by
\[
o_M\otimes R
\]
the local system of $R$-modules obtained by tensoring
each fiber with $R$. 

The classification theorem for covering spaces has as a corollary:
\begin{theorem} 
Let $B$ be path connected and semi-locally simply connected.
Then forming the fiber over a point gives an equivalence of categories from
the category of local coefficient systems of $R$-modules over $B$
and the category of modules over the group algebre $R[\pi_1(B,b)]$. 
\end{theorem}

Our $R$-modules are quite simple: they are free of rank 1. Since any 
automorphism of such an $R$-module is given by multiplication by a unit
in $R$, we find that the local coefficient system is defined by 
giving a homomorphism 
\[
\pi_1(B,b)\to R^\times
\]
or, what is the same, an element of $H^1(B;R^\times)$. 

When $R=\Z$, this homomorphism
\[
w_1:\pi_1(B,b)\to\{\pm1\}
\]
is the ``first Stiefel-Whitney class.'' If it is trivial, you can pick
consistent generators for $H_n(M,M-x)$ as $x$ runs over $M$: the manifold
is ``orientable,'' and is {\em oriented} by one of the two possible choices. 
If it is nontrivial, the manifold is {\em nonorientable}. I hope it's clear
that the M\"obius band is nonorientable, and hence any surface containing
the M\"obius band is as well. 

The set of abelian group generators of the fibers of $o_M$ 
form a sub covering space, a double cover of $M$, denoted by $o_M^\times$.
It is the ``orientation double cover.'' If $M$ is
orientable (and connected) it is trivial; it consists of two copies of $M$. 
An orientation consists in chosing one or the other of the components. 
If $M$ is nonorientable (and connected) the orientation double cover is
again connected. An interesting fact is that its total space is a manifold
in its own right, and is orientable; in fact it carries a canonical 
orientation. 

Similarly we can form the sub covering space of $R$-module generators of
the fibers of $o_M\otimes R$; write $(o_M\otimes R)^\times$ for it. 

Now if $p:E\to B$ is a covering space, one of the things you may want to do 
is consider a {\em section} of $p$; that is, a continuous function 
$\sigma:B\to E$ such that $p\circ\sigma=1_B$. Write $\Gamma(B;E)$ for
the set of sections of $p:E\to B$. Under the corresondence
between covering spaces and actions of $\pi$, 
\[
\Gamma(B;E)=(p^{-1}(b))^{\pi_1(B,b)}\,,
\]
the fixed point set for the action of $\pi_1(B,b)$ on $p^{-1}(b)$. 
If $E$ is a local system of $R$-modules, this is a sub $R$-module. 

A ``local $R$-orientation at $x$'' is a choice of $R$-module generator of 
$H_n(M,M-x;R)$, and we make the following definition. 

\begin{definition} An $R$-{\em orientation} of an $n$-manifold $M$ is a 
section of $(o_M\otimes R)^\times$.
\end{definition}

For example, when $R=\FF_2$, every manifold is orientable, 
and uniquely so, since $\FF_2^\times=\{1\}$. A $\ZZ$-orientation (or simply
``orientation'') is a section of the orientation double cover. 
A manifold is ``$R$-orientable'' if it admits an $R$-orientation.  
A connected $n$-manifold is either non-orientable, or admits two 
orientations. Euclidean space is orientable. 

This relates to the ``globalization'' project we started out talking about. 
A section over $B$ is in fact called a ``global section.'' In the case of
the orientation local system, we have a canonical map
\[
j:H_n(M;R)\to \Gamma(M;o_M\otimes R)\,,
\]
described as follows. The value of $j(a)$ at $x\in M$ is the restriction
of $a$ to $H_n(M,M-x)$. The first ``local-to-global'' theorem, 
a special case of Poincar\'e duality, is this:
\begin{theorem} 
If $M$ is compact, the map $j:H_n(M;R)\to\Gamma(M;o_M\otimes R)$ 
is an isomorphism. 
\label{thm-orientation}
\end{theorem}

We will prove this theorem in the next lecture.

The representation of $\pi_1(B)$ on the fiber of $o_M\otimes R$ over $b$
is given by the composite $\pi_1(B)\to\{\pm1\}\to R^\times$. If this is the
trivial homomorphism, the fixed points of this representation on $R$ 
form all of $R$. If not, the fixed points are the subgroup of $R$
of elements of order 2, written $R[2]$.
\begin{corollary}
If $M$ is a compact connected $n$-manifold, then 
\[
H_n(M;R)\cong\begin{cases}R & \hbox{if $M$ is orientable} \\ 
R[2] & \hbox{if not}\,.
\end{cases}
\]
\end{corollary}
In the first case, a generator of $H_n(M;R)$ is a {\em fundamental class} 
for the manifold. You should think of the manifold itself as a cycle
representing this homology class. It is characterized as a class restricting
to a generator of $H_n(M,M-x)$ for all $x$; this is saying that the cycle
``covers'' the point $x$ once.

The first isomorphism in the theorem 
depends upon this choice of fundamental class. But in the second case,
the isomorphism is canonical. Over $\FF_2$, any compact connected manifold 
has a unique fundamental class, the generator of $H_n(M;\FF_2)=\FF_2$. 

