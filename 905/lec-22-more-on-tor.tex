\section{The fundamental lemma of homological algebra}

We will now show that the $R$-modules $\Tor^R_n(M,N)$ are well-defined 
and functorial. This will be an application of a very general principle.
\begin{theorem}[Fundamental theorem of homological algebra]
Let $M$ and $N$ be $R$-modules; let
\[
0\leftarrow M\leftarrow E_0\leftarrow E_1\leftarrow\cdots
\]
be a sequence in which each $E_n$ is free, and 
\[
0\leftarrow N\leftarrow F_0\leftarrow F_1\leftarrow\cdots
\]
be an exact sequence; and let $f:M\to N$ be a homomorphism.
Then we can lift $f$ to a chain map $f_*:E_*\to F_*$, uniquely up to chain
homotopy.
\end{theorem}
\begin{proof} Let's try to construct $f_0$. Consider:
\begin{equation*}
\xymatrix{0\ar[r] & K_0=\ker(\epsilon_M)\ar[r]\ar@{-->}[d]^{g_0} & E_0\ar[r]^{\epsilon_M}\ar@{-->}[d]^{f_0} & M\ar[d]^f &\\
0\ar[r] & L_0=\ker(\epsilon_N)\ar[r] & F_0\ar@{->>}[r]^{\epsilon_N} & N\ar[r] & 0}
\end{equation*}
We know that $E_0=R\langle S\rangle$. What we do is push the generators of $E$ into $M$ via $\epsilon_M$ and then into $F$ via $f$, and then lift them to $F_0$ via  $\epsilon_N$ (which is possible because it's surjective). Then extend to
a homomorphism, to get $f_0$. You can restrict it to kernels to get $g_0$. 

Now the map $d:E_1\to E_0$ satisifes $\epsilon_M\circ d=0$, and so factors through a map to $K_0=\ker\epsilon_M$. Similarly, $d:F_1\to F_0$ factors through 
a map $F_1\to L_1$, and this map must be surjective because the sequence
$F_1\to F_0\to N$ is exact. We find ourselves in exactly the same situation:
\begin{equation*}
\xymatrix{0\ar[r] & K_1\ar[r]\ar@{-->}[d]^{g_1} & E_1\ar[r]\ar@{-->}[d]^{f_1} & K_0\ar[d]^{g_0} &\\
0\ar[r] & L_1\ar[r] & F_1\ar[r] & L_0\ar[r] & 0}
\end{equation*}
So by we construct $f_*$ by induction.

Now we need to prove the chain homotopy claim. So suppose I have $f_\ast,f'_\ast:E_\ast\to F_\bullet$, both lifting $f:M\to N$. Then $f^\prime_n-f_n$ (which we'll rename $\ell_n$) is a chain map lifting $0:M\to N$. 
We want to consruct a chain null-homotopy of $\ell_\ast$; that is, we want $h:E_n\to F_{n+1}$ such that $dh+hd=\ell$. At the bottom, $E_{-1}=0$ so we want
$h:E_0\to F_1$ such that $dh=\ell_0$. This factorization happens in two steps. 
\begin{equation*}
\xymatrix{ & & E_0\ar[d]_{\ell_0}\ar[dl]\ar@{-->}[dll]_h \ar[r] & M\ar[d]^0 \\
F_1\ar@{->>}[r] & L_0\ar[r] & F_0 \ar[r] & N \,.
}\end{equation*}
First, $d\ell_0=0$ implies that $\ell_0$ factors through $L_1=\ker\epsilon_N$.
Next, $F_1\to L_0$ is surjective, by exactness, and $E_0$ is free, so we can lift generators and extend $R$-linearly as indicated. 

The next step is organized by the diagram
\begin{equation*}
\xymatrix{ & & E_1\ar[r]^d\ar[d]^{\ell_1}\ar@{-->}[dl]\ar@{-->}[dll]_h & 
E_0\ar[d]^{\ell_0}\ar[dl]_h & \\
F_2\ar@{->>}[r]\ar@/_/[rr]_d & L_1\ar[r] & F_1\ar[r]^d & F_0}
\end{equation*}
This diagram doesn't commute; while $dh=\ell_0$, we want to construct 
$h:E_1\to F_2$ such that $dh=\ell_1-hd$. Since 
\[
d(\ell_1-hd)=\ell_0d-dhd=(\ell_0-dh)d=0\,.
\]
the map $\ell_1-hd$ lifts to $L_1=\ker d$. But then it lifts through $F_2$, 
since $F_2\to L_1$ is surjective and $E_1$ is free. 

Exactly the same process continues.
\end{proof}

This proof uses a property of freeness that is shared by a broader class of
modules. 

\begin{definition}
An $R$-module $P$ is {\em projective} if any map out of $P$ factors through 
any surjection:
\begin{equation*}
\xymatrix{ & M\ar@{->>}[d]\\
P\ar@{-->}[ur]\ar[r] & N}
\end{equation*}
\end{definition}
Every free module is projective; this is what we have been using;
our proof of the fundamental lemma of homogolical algebra uses only that 
$E_n$ is projective. 
Anything that's a direct summand in a projective is also projective. Any projective module is a direct summand of a free module. Over a PID, every projective is free, because any submodule of a free is free. But there are examples of
nonfree projectives: 
\begin{example}
Let $k$ be a field and let $R$ be the product ring $k\times k$. It acts on $k$
in two ways, via $(a,b)c=ac$ and via $(a,b)c=bc$. This are both projective $R$-modules that are not free.
\end{example}

Now we will apply the Fundamental Lemma 
to verify that our proposed construction of 
$\Tor$ is independent of free (or projective!) resolution, and is functorial. 

Suppose I have $f:N'\to N$. Pick arbitrary free resolutions 
$N'\leftarrow N'_*$ and $N\leftarrow N_*$, and pick any chain map 
$f_*F'_*\to F_*$ lifting $f$. We claim that the map
induced in homology by $1\otimes f_*:M\otimes_RF'_*\to M\otimes_RF_*$
is independent of the choice of lift. Suppose $f'_*$ is another lift,
and pick a chain homotopy $h:f\sim f'$. Since $M\otimes_R-$ is additive, 
the relation 
\[
1\otimes h:1\otimes f\sim 1\otimes f'
\] 
still holds. So $1\otimes f$ and $1\otimes f'$ induce the same map in homology.

For example, suppose that $F_*$ and $F'_*$ are two projective resolutions of 
$N$. Any two lifts of the identity map are chain-homotopic, and so induce
the same map $H_*(M\otimes_RF_*)\to H_*(M\otimes_RF'_*)$. So if $f:F_*\to F'_*$
and $g:F'_*\to F_*$ are chain maps lifing the identity, then $f_*\circ g_*$
induces the same self-map of $H_*(M\otimes_RF'_*)$ as the identity self-map 
does, and so (by functoriality) is the identity. Similarly, $g_*\circ f_*$
induces the identity map on $H_*(M\otimes_RF_*)$. So they induce inverse
isomorphisms. 

Putting all this together shows that any two projective resolutions of $N$ 
induce canonically isomorphic modules $\Tor_n^R(M,N)$, and that a homomorphism
$f:N'\to N$ induces a well defined map $\Tor_n^R(M,N')\to\Tor_n^R(M,N)$
that renders $\Tor^R_n(M,-)$ a functor. 

Last comment about $\Tor$ is that there's a symmetry there. Of course, $M\otimes_R N\cong N\otimes_R M$. This uses the fact that $R$ is commutative. This leads right on to saying that $\Tor^R_n(M,N)\cong \Tor^R_n(N,M)$. We've been computing $\Tor$ by taking a resolution of the second variable. But I could equally have taken a resolution of the first variable. This follows from the fundamental theorem of homological algebra.


\begin{example}
I want to give an example when you do have higher $\Tor$ modules. Let $k$ be a field, and let $R=k[d]/(d^2)$. This is sometimes called the ``dual numbers'', or the exterior algebra over $k$. We're going to consider $R$-modules. Let's construct a projective resolution of $k$. What is an $R$-module $M$? It's just a $k$-vector space $M$ with an operator $d$ (given by multiplication by $e$) that satisfies $d^2=0$. Even though there's no grading around, I can still define the ``homology'' of $M$:
\[
H(M;d)=\frac{\ker d}{\img d}\,.
\]

This $k$-algebra is {\em augmented} by an algebra map $\epsilon:R\to k$ splitting the unit; $\epsilon(d)=0$. Let's construct a free $R$-module resolution of this module. Here's a picture.
\[
\xymatrix@R=8pt{
&&&& \bullet \ar@{..}[r] \ar@{-}[d] & \\
&&& \bullet \ar@{-}[d] & \bullet \ar[l]  \\
&& \bullet \ar@{-}[d] & \bullet \ar[l] \\
& \bullet \ar@{-}[d] & \bullet \ar[l] \\
\bullet & \bullet \ar[l]
}\]
The vertical lines indicate multiplication by $e$. We could write this as
\[
0\leftarrow 
k\xleftarrow{\epsilon}R\xleftarrow{e}R\xleftarrow{e}R\leftarrow\cdots\,.
\]

Now tensor this over $R$ with an $R$-module $M$; so $M$ is a vector space 
equipped with an operator $d$ with $d^2=0$. Each copy of $R$ gets replaced by
a copy of $M$, and the differential gives multiplication by $d$ on $M$. So 
taking homology gives 
\[
\Tor^R_n(k,M)=
\begin{cases}k\otimes_RM=M/dM\,&\,\hbox{for}\,\,n=0\\
H(M;d)\,&\,\hbox{for}\,n>0\,.
\end{cases}
\]
So for example 
\[
\Tor^R_n(k,k)=k\,\,\,\hbox{for}\,\,n\geq0\,.
\]
\end{example}

