\section{Applications}
Today we harvest consequences of Poincar\'e duality. We'll use the form
\begin{theorem}
Let $M$ be an $n$-manifold and $K$ a closed subset.
An $R$-orientation along $K$ determines a fundamental class
$[M]_K\in H_n(M,M-K)$, and capping gives an isomorphism:
\[
\cap[M]_K:\cHH^p(K;R)\xrightarrow{\cong}H_{n-p}(M,M-K;R)\,.
\]
\end{theorem}
\begin{corollary}
$\cHH^q(K;R)=0$ for $q>n$.
\end{corollary}
We can contrast this with singular (co)homology. Here's an example:
\begin{example}[Barratt-Milnor, \cite{barratt-milnor}]
A two-dimensional version $K$ of the Hawaiian earring, i.e., nested spheres all tangent to a point whose radii are going to zero. What they proved is that $H_q(K;\ZZ)$ is uncountable for every $q>1$. But \v{C}ech cohomology is much more
well-behaved.
\end{example}
\begin{theorem}[Alexander duality]
For any compact subset $K$ of $\RR^n$, the composite
\[
\cHH^{n-q}(K;R)\xrightarrow{\cap[\RR^n]_K}H_q(\RR^n,\RR^n-K;R)
\xrightarrow{\partial}\widetilde{H}_{q-1}(\RR^n-K;R)
\]
is an isomorphism.
\end{theorem}
\begin{proof} $\widetilde H^*(\RR^n;R)=0$. \end{proof}
This is extremely useful! For example
\begin{corollary}
If $K$ is a compact subset of $\RR^n$ then $\cHH^n(K;R)=0$. 
\end{corollary}
\begin{example}
Take the case $q=1$:
\[
\cHH^{n-1}(K;R)\xrightarrow{\cong}\widetilde{H}_0(\RR^n-K;R)
=\ker(\varepsilon:R\pi_0(\RR^n-K)\to R)\,.
\]
The augmentation is a split surjection, so this is a free $R$-module.
This shows, for example, that $\RP^2$ can't be embedded in $\RR^3$ -- at least
not with a regular neighborhood. 

If we take $n=2$ and suppose that $\cHH^*(K)=H^*(S^1)$, we find that the 
complement of $K$ has {\em two} path components. This is the 
{\em Jordan Curve Theorem}.
\end{example}
There is a useful purely cohomological consequence of Poincar\'e duality,
obtained by combining it with the universal coeffient theorem
\[
0\to\Ext^1_\Z(H_{q-1}(X),\Z)\to H^q(X)\to\Hom(H_q(X),\Z)\to 0\,.
\]
First, note that $\Hom(H_q(X),\Z)$ is always torsion-free. If I assume that $H_{q-1}(X)$ is finitely generated, then $\Ext^1_\Z(H_{q-1}(X),\Z)$ is a finite abelian group. So the UCT is providing the short exact sequence
\[
0\to\mathrm{tors}H^q(X)\to H^q(X)\to H^q(X)/\mathrm{tors}\to0
\]
-- that is, 
\[
H^q(X)/\mathrm{tors}\xrightarrow{\cong}\Hom(H_q(X)/\mathrm{tors},\Z)\,.
\]
That is to say, the Kronecker pairing descends to a perfect pairing
\[
\frac{H^q(X)}{\mathrm{tors}}\otimes\frac{H_q(X)}{\mathrm{tors}}\to\Z\,.
\]

Let's combine this with Poincar\'e duality.
Let $X=M$ be a compact oriented $n$-manifold, so that
\[
\cap[M]:H^{n-q}(M)\xrightarrow{\cong}H_q(M)\,.
\]
We get a perfect pairing
\[
\frac{H^q(X)}{\mathrm{tors}}\otimes\frac{H^{n-q}(X)}{\mathrm{tors}}\to\Z\,.
\]

And what is that pairing? It's given by the composite
\begin{equation*}
\xymatrix{
	H^q(M)\otimes H^{n-q}(M)\ar[r]\ar[d]_{1\otimes(-\cap [M])} & \Z\\
	H^q(M)\otimes H_q(M)\ar[ur]_{\langle-,-\rangle}
}
\end{equation*}
and we've seen that 
\begin{equation*}
\langle a,b\cap [M]\rangle = \langle a\cup b,[M]\rangle
\end{equation*}
We have used $R=\Z$, but the same argument works for any PID -- in particular
for any field, in which case $\mathrm{tors}V=0$. We have proven:
\begin{theorem} Let $R$ be a PID an $M$ a compact $R$-oriented $n$-manifold. 
Then 
\begin{equation*}
a\otimes b\mapsto\langle a\cup b,[M]\rangle
\end{equation*}
induces a perfect pairing (with $p+q=n$)
\[
\frac{H^p(M;R)}{\mathrm{tors}}\otimes_R\frac{H^q(M;R)}{\mathrm{tors}}\to R\,.
\]
\end{theorem}
\begin{example}
Complex projective 2-space is a compact 4-manifold, orientable since it is 
simply 
connected. It has a cell structure with cells in dimensions $0$, $2$, and $4$,
so its homology is $\Z$ in those dimensions and 0 elsewhere, and so the same is
true of its cohomology. Up till now the cup product structure has been a 
mystery. But now we know that 
\[
H^2(\CP^2)\otimes H^2(\CP^2)\to H^4(\CP^2)
\]
is a perfect pairing. So if we write $a$ for a generator of $H^2(\CP^2)$,
then $a\cup a=a^2$ is a free generator for $H^4(\CP^2)$. We have discovered 
that 
\[
H^*(\CP^2)=\Z[a]/a^3\,.
\]
By the way, notice that if we had chosen $-a$ as a generator, we would still
produce the same generator for $H^4(\CP^2)$: so there is a preferred
orientation, the one whose fundamental class pairs to 1 against $a^2$. 

This calculation shows that while $\CP^2$ and $S^2\vee S^4$ are both
simply connected and have the same homology, they are not homotopy
equivalent. This implies that the attaching map $S^3\to S^2$ for the top
cell in $\CP^2$ -- the {\em Hopf map} -- is essential.

How about $\CP^3$? It just adds a $6$-cell, so now $H^6(\CP^3)\cong\Z$.
The pairing $H^2(\CP^3)\otimes H^4(\CP^3)\to H^6(\CP^3)$ is perfect, so we
find that $a^3$ generates $H^6(\CP^3)$. Continuing in this way, we have
\[
H^\ast(\CP^n)=\Z[a]/(a^{n+1})\,.
\]
\end{example}
\begin{example}
Exactly the same argument shows that
\begin{equation*}
H^\ast(\RP^n;\FF_2)=\FF_2[a]/(a^{n+1})
\end{equation*}
where $|a|=1$. 
\end{example}
I'll end with the following application.
\begin{theorem}[Borsuk-Ulam]
Think of $S^n$ as the unit vectors in $\RR^{n+1}$. 
For any continuous function $g:S^n\to\RR^n$, there exists $x\in S^n$
such that $g(x)=g(-x)$.
\end{theorem}
\begin{proof} Suppose that no such $x$ exists. Then we may define
a continuous function $f:S^n\to S^{n-1}$ by 
\[
x\mapsto\frac{g(x)-g(-x)}{||g(x)-g(-x)||}\,.
\]
Note that $g(-x)=-g(x)$: $g$ is equivariant with respect to the antipodal 
action. It descends to a map $\overline g:\RP^n\to\RP^{n-1}$.

Now pick a basepoint $b\in S^n$ and choose a path $\sigma:I\to S^n$ 
such that $\sigma(0)=b$ and $\sigma(1)=-b$. 
Write $p:S^n\to\RP^n$ for the projection map, and look at 
$\overline gp\sigma\in\pi_1(\RP^{n-1})$. It must be nontrivial,
since it acts nontrivially on the fiber over $\overline g(b)$:
By equivariance, 
$g\circ\sigma$ sends a point in $S^{n-1}$ to its antipodal point. 

It follows that $\overline g$ is nontrivial on $\pi_1$, and hence also
nontrivial on $H_1$ and so on $H^1(-;\FF_2)$. But $H^n(\RP^{n-1};\FF_2)=0$, 
while (writing $a$ for the generator of $H^1(\RP^n)$) $a^n\neq0$ in 
$H^n(\RP^n;\FF_2)$: a contradiction.
\end{proof}
