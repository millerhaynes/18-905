\section{Real projective space}


Let's try to compute $H_\ast(\mathbf{RP}^n)$. This computation will invoke a 
second way to think of the cellular chain group $C_n(X)$. Each cell has a 
characteristic map $D^n\to X_n$, and we have the diagram
\[
\xymatrix{
\coprod(D^n,S^{n-1}) \ar[r] \ar[dr] & (X_n,X_{n-1}) \ar[d] \\
& (\bigvee S^{n-1},\ast).
}\]
We've shown that the vertical map induces an isomorphism in homology, and
the diagonal does as well. (For example, $\coprod D^n$ has a CW structure 
in which the $(n-1)$-skeleton is $\coprod S^{n-1}$.) So 
\[
H_n(\coprod(D_n,S^{n-1}))\xrightarrow{\cong}C_n(X).
\]
 
We have a CW structure on $\RP^n$ with $\mathrm{Sk}_k(\RP^n)=\RP^k$;
there is one $k$-cell -- which we'll denote by $e_k$ -- for each $k$ between $0$ and $n$. So the cellular chain complex looks like this:
\begin{equation*}
\xymatrix{0  & C_0(\mathbf{RP}^n)\ar@{=}[d]\ar[l] & C_1(\mathbf{RP}^n)\ar@{=}[d]\ar[l] & \cdots\ar[l] & C_n(\mathbf{RP}^n)\ar@{=}[d]\ar[l] & 0 \ar[l]\\
0 & \Z\langle e^0\rangle \ar[l] & \Z\langle e^1\rangle\ar[l]_{d=0} & \cdots\ar[l] & \Z\langle e^n\rangle\ar[l] & 0 \ar[l] }
\end{equation*}
The first differential is zero because we know what $ H_0(\mathbf{RP}^n)$ is (it's $\Z$!). The differential in the cellular chain complex is given by the top
row in the following commutative diagram.
\begin{equation*}
\xymatrix{C_n= H_n(\mathbf{RP}^n,\mathbf{RP}^{n-1})\ar[r]^\partial & 
H_{n-1}(\mathbf{RP}^{n-1}) \ar[r] \ar[dr] & 
H_{n-1}(\mathbf{RP}^{n-1},\mathbf{RP}^{n-2})=C_{n-1} \ar[d]_{\cong} \\
H_n(D^n,S^{n-1}) \ar[u]_\cong \ar[r]^\partial_\cong & 
H_{n-1}(S^{n-1}) \ar[u]_{\pi_*} \ar[r] & H_{n-1}(D^{n-1}/S^{n-2},\ast)\,.
}
\end{equation*}
The map $\pi:S^{n-1}\to\RP^{n-1}$ is the attaching map of the top cell of $RP^n$; that is, the double cover. The diagonal composite pinches the subspace 
$\RP^{n-1}$ to a point. The composite map $S^{n-1}\to D^{n-1}/S^{n-2}$ 
factors as follows: 
\begin{equation*}
\xymatrix{S^{n-1}\ar[r]^{\text{double cover}}\ar[dr] & \mathbf{RP}^{n-1}\ar[r]^{\text{pinch}} & D^{n-1}/S^{n-2}\\
 & S^{n-1}/S^{n-2}=S^{n-1}\vee S^{n-1}\ar[ur]}
\end{equation*}
One of the maps $S^{n-1}\to S^{n-1}$ from the wedge is the identity, and the other map is the antipodal map $\alpha:S^{n-1}\to S^{n-1}$. Write $\sigma$ for a generator of $ H_{n-1}(S^{n-1})$. Then in $H_{n-1}$ we have $\sigma\mapsto (\sigma,\sigma)\mapsto \sigma+\alpha_\ast\sigma$. So we need to know the degree of the antipodal map on $S^{n-1}$. The antipodal map reverses all $n$ coordinates in $\RR^n$. Each reversal is a reflection, and acts on $S^{n-1}$ by a map of degree $-1$. So 
\[
\deg\alpha=(-1)^n\,.
\]
Therefore the cellular complex of $\RP^n$ is as follows: 
\begin{equation*}
\xymatrix{\hbox{dim} & -1 & 0 & 1 & \cdots & n & n+1 & \cdots \\
& 0 & \Z\ar[l]_0 & \Z\ar[l]_2 & \cdots\ar[l]_0 & \Z\ar[l]_{2\text{ or }0} & 0\ar[l] & \cdots\ar[l]}
\end{equation*}
The homology is then easy to read off.
\begin{prop}
The homology of real projective space is as follows.
\begin{equation*}
 H_k(\RP^n)=\begin{cases}
\Z & k=0\text{ and }k=n\text{ odd}\\
\Z/2\Z & k\text{ odd, }0<k<n\\
0 & \text{otherwise}\,.
\end{cases}
\end{equation*}
\end{prop}
Here's a table. Missing entries are $0$.
\[
\xymatrix@R=8pt{
\text{dim} & 0 & 1 & 2 & 3 & 4 & 5 & \cdots \\
\RP^0 & \ZZ \\
\RP^1 & \ZZ & \ZZ \\
\RP^2 & \ZZ & \ZZ/2 \\
\RP^3 & \ZZ & \ZZ/2 & 0 & \ZZ \\
\RP^4 & \ZZ & \ZZ/2 & 0 & \ZZ/2 \\
\RP^5 & \ZZ & \ZZ/2 & 0 & \ZZ/2 & 0 & \ZZ 
}\]
The moral: In real projective space, 
odd cells create new generators; even cells (except for the zero-cell) create
torsion in the previous dimension. 

This example illustrates the significance of cellular homology, and, therefore,
of singular homology. A CW structure involves attaching maps 
\[
\coprod S^{n-1}\to\Sk_n X\,.
\]
Knowing these, up to homotopy, determines the full homotopy type of the CW 
complex. Homology does not record all this information. Instead, it records 
only information about the composite obtaind by pinching out $\Sk_{n-1}X$.
\[
\xymatrix{ 
\coprod_{a\in A_n} S^{n-1}_a \ar[r] \ar[dr] & \Sk_n X \ar[d] \\
& \bigvee_{b\in A_{n-1}} S^{n-1}_b\,.
}\]
In $H_{n-1}$, this can be identified with a map
\[
\partial:\Z[A_n]\to\Z[A_{n-1}]
\]
that is none other than the differential in the cellular chain complex.

The moral: homology picks off only the ``first order'' structure of a CW 
complex. 

On the other hand, we'll see in the next lecture that it does a very good job
of that. 


