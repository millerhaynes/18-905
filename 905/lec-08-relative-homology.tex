\section{Relative homology}

An ultimate goal of algebraic topology is to find means to compute the set of homotopy classes of maps from one space to another. This is important because many geometrical problems can be rephrased as such a computation. It's a lot more modest than wanting to characterize, somehow, all continuous maps from $X$ to $Y$; but the very fact that it still contains a great deal of interesting information means that it is still a very challenging problem.

Homology is in a certain sense the best ``additive'' approximation to this problem; and its additivity makes it much more computable. To justify this, we want to describe the sense in which homology is ``additive.'' Here are two related aspects of this claim.
	\begin{enumerate}
	\item If $A\subseteq X$ is a subspace, then $ H_\ast(X)$ a combination of $ H_\ast(A)$ and $H_\ast(X-A)$. 
	\item The homology $ H_\ast(A\cup B)$ is like $ H_\ast(A)+ H_\ast(B)- H_\ast(A\cap B)$. 
	\end{enumerate}
The first hope is captured by the long exact sequence of a pair, the second
by the Mayer-Vietoris Theorem. 
Both facts show that homology behaves like a measure. 
The precise statement of both facts uses the machinery of exact sequences.
I'll use the following language.

\begin{definition}
A {\em sequence} of abelian groups is a diagram of abelian groups of the form 
\[
\cdots\rightarrow C_{n+1}\xrightarrow{f_n}C_n\xrightarrow{f_{n-1}}C_{n-1}
\rightarrow\cdots\,,
\]
in which all composites are zero; that is, $\img f_n\subseteq\ker f_{n-1}$ for all $n$. It is 
{\em exact} at $C_n$ provided that this inequality is an equality. 
\end{definition}
A sequence is just another name for a chain complex; it is exact at 
$C_n$ if and only if $H_n(C_\ast)=0$. So homology measures the failure of exactness.
\begin{example}
Sequences may be zero for $n$ large or for $n$ small. We may just not
write them down if all the groups from some point on are zero. 
For example,
$0\to A\xrightarrow{i}B$ is exact iff $i$ is injective, and $B\xrightarrow{p}C\to 0$ is exact iff $p$ is surjective.
\end{example}
Exactness was a key concept in the development of algebraic topology, and ``exact'' is a great word for the concept. A foundational treatment
\cite{eilenberg-steenrod} of algebraic topology was published by 
Sammy Eilenberg and Norman Steenrod in 1952.
The story goes that in the galleys for the book they left a blank space whenever the word representing this concept was used, and filled it in at the last minute. 
\begin{definition}
A {\em short exact sequence} is an exact sequence of the form 
\[
0\to A\xrightarrow{i}B\xrightarrow{p} C\to 0\,.
\]
\end{definition}

Any sequence of the form $A\to B\to C$ expands to a diagram
	\begin{equation*}
	\xymatrix{\ker(p) \ar[dr] & & \\
	A\ar[u]\ar[r]^i & B\ar[r]^p\ar[dr] & C\\
	 & & \coker(i)\ar[u]}
	\end{equation*}
It is exact at $B$ if and only if $A\xrightarrow{\cong}\ker p$ or, equivalently,
$\coker(i)\xrightarrow{\cong}C$. It is short exact if furthermore $i$ is
injective and $p$ is surjective.

We will study the homology of a space $X$ by comparing it to the homology of a subspace $A$ and a complement or quotient modulo the subspace.
Note that $S_*(A)$ injects into $S_*(X)$. This suggests considering the quotient group
\[
\frac{S_n(X)}{S_n(A)}\,.
\]
This is the group of {\em relative $n$-chains} of the pair $(X,A)$. 

Let's formalize this a bit. Along with the category $\mathbf{Top}$ of spaces,
we have the category $\mathbf{Top_2}$ of {\em pairs} of spaces. An object of
$\mathbf{Top_2}$ is a space $X$ together with a subspace $A$. A map
$(X,A)\to(Y,B)$ is a continuous map $X\to Y$ that sends $A$ into $B$. 

There are four obvious functors relating $\mathbf{Top}$ and $\mathbf{Top_2}$:
\[
X\mapsto(X,\varnothing)\,,\quad X\mapsto(X,X)\,,
\]
\[
(X,A)\mapsto X\,,\quad(X,A)\mapsto A\,.
\]

Do the relative chains form themselves into a chain complex? 
\begin{lemma}
Let $A_*$ be a subcomplex of the chain complex $B_*$. There is a unique
structure of chain complex on the quotient graded abelian group $C_*$ with 
entries $C_n=B_n/A_n$ such that $B_*\to C_*$ is a chain map.
\end{lemma}
\begin{proof}
To define $d:C_n\to C_{n-1}$, represent $c\in C_n$ by $b\in B_n$, and
hope that $[db]\in B_{n-1}/A_{n-1}$ is well defined. If we replace
$b$ by $b+a$ for $a\in A_n$, we find 
\[
d(b+a)=db+da\equiv db\mod A_{n-1}\,,
\]
so our hope is justified. Then $d^2[b]=[d^2 b]=0$. 
\end{proof}

\begin{definition} The {\em relative singular chain complex} of the pair 
$(X,A)$ is 
\[
S_*(X,A)=\frac{S_*(X)}{S_*(A)}\,.
\]
\end{definition}
This is a functor from pairs of spaces to chain complexes. Of course
\[
S_*(X,\varnothing)=S_*(X)\,,\quad S_*(X,X)=0\,.
\]

\begin{definition} The {\em relative singular homology} of the pair $(X,A)$
is the homology of the relative singular chain complex:
\[
H_n(X,A)=H_n(S_*(X,A))\,.
\]
\end{definition}

One of the nice features of the absolute chain group $S_n(X)$ is that it is free as an abelian group. This is also the case for its quotent $S_n(X,A)$, since
the map $S_n(A)\to S_n(X)$ takes basis elements to basis elements. 
$S_n(X,A)$ is freely generated by the $n$-simplices in $X$ that do not lie 
entirely in $A$.
\begin{example}
Consider $\Delta^n$, relative to its boundary 
\[
\partial\Delta^n:=\bigcup \img d_i\cong S^{n-1}\,.
\] 
We have the identity map $\iota_n:\Delta^n\to \Delta^n$, the universal $n$-simplex, in $\Sin_n(\Delta^n)\subseteq S_n(\Delta^n)$. It is not a cycle; its boundary $d\iota_n\in S_{n-1}(\Delta^n)$ is the alternating sum of the faces of the $n$-simplex. Each of these singular simplices lies in $\partial\Delta^n$, so $d\iota_n\in S_{n-1}(\partial\Delta^n)$, and 
$[\iota_n]\in S_n(\Delta_n,\partial\Delta_n)$ {\em is} a {\em relative} cycle. 
We will see that the relative homology $H_n(\Delta^n,\partial\Delta^n)$ is infinite cyclic, with generator $[\iota_n]$. 
\end{example}
