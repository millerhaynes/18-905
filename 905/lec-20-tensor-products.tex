\section{Tensor product}

The category of $R$-modules is what might be called a ``categorical ring,'' in which addition corresponds to the direct sum, the zero element is the zero module, $1$ is $R$ itself, and multiplication is \ldots well, the subject for today. We care about the tensor product for two reasons: 
First, it allows us to deal smoothly with bilinear maps such that 
the cross-product. Second, and perhaps more important, it will allow us relate homology
with coefficients in an any $R$-module to homology with coefficients in the 
PID $R$; for example, relate $H_*(X;M)$ to $H_*(X)$, where $M$ is any
abelian group.

Let's begin by recalling the definition of a bilinear map over a commutative ring $R$.
\begin{definition}
Given three $R$-modules, $M,N,P$, a {\em bilinear map} (or, to be explicit, $R$-{\em bilinear map}) is a function $\beta:M\times N\to P$ such that
\[
\beta(x+x^\prime,y)=\beta(x,y)+\beta(x^\prime,y)\,,\quad
\beta(x,y+y^\prime)=\beta(x,y)+\beta(x,y^\prime)\,,
\]
and
\[
\beta(rx,y)=r\beta(x,y)\,,\quad\beta(x,ry)=r\beta(x,y)\,,
\]
for $x,x'\in M$, $y,y'\in N$, and $r\in R$.
\end{definition}

\begin{example}
$\RR^n\times\RR^n\to\RR$ given by the dot product is an $\RR$-bilinear map. The cross product $\RR^3\times\RR^3\to\RR^3$ is $\RR$-bilinear. If $R$ is a ring, the multiplication $R\times R\to R$ is $R$-bilinear, and the multiplication on an $R$-module $M$ given by $R\times M\to M$ is $R$-bilinear. This enters into topology because the cross-product $ H_m(X;R)\times H_n(Y;R)\xrightarrow{\times} H_{m+n}(X\times Y;R)$ is $R$-bilinear.
\end{example}
Wouldn't it be great to reduce stuff about bilinear maps to linear maps? We're going to do this by means of a universal property.
\begin{definition}
Let $M,N$ be $R$-modules. A \emph{tensor product} of $M$ and $N$ is an $R$-module $P$ and a bilinear map $\beta_0:M\times N\rightarrow P$ such that for every $R$-bilinear map $\beta:M\times N\rightarrow Q$ there is a unique factorization
\begin{equation*}
\xymatrix{M\times N\ar[r]^-{\beta_0}\ar[dr]^\beta & P\ar@{-->}[d]^f\\
 & Q}
\end{equation*}
through an $R$-module homomorphism $f$. 
\end{definition}
We should have pointed out that the composition $f\circ\beta_0$ is indeed 
again $R$-bilinear; but this is easy to check.

So $\beta_0$ is a universal bilinear map out of $M\times N$. Instead of $\beta_0$ we're going to write $\otimes:M\times N\rightarrow P$. This means that $\beta(x,y)=f(x\otimes y)$ in the above diagram. There are lots of things to say about this. When you have something that is defined via a universal property, you know that it's unique \ldots but you still have to check that it exists!
\begin{construction}
I want to construct a univeral $R$-bilinear map out of $M\times N$. Let $\beta:M\times N\to Q$ be any $R$-bilinear map. This $\beta$ isn't linear. Maybe we should first extend it to a linear map. There is a unique $R$-linear extension
over the free $R$-module $R\langle M\times N\rangle$ generated by the set 
$M\times N$:
\begin{equation*}
\xymatrix{M\times N\ar[rr]^\beta\ar[dr]^{[-]} & & Q\\
& R\langle M\times N\rangle\ar[ur]^{\overline{\beta}} &}
\end{equation*}
The map $[-]$, including a basis, isn't bilinear. So we should quotient $R\langle M\times N\rangle$ by a submodule $S$ of relations to make it bilinear. So $S$ is the sub $R$-module generated by the four familes of elements (corresponding to the four relations in the definition of $R$-bilinearity): 
\begin{enumerate}
\item $[(x+x^\prime,y)]-[(x,y)]-[(x^\prime-y)]$
\item $[(x,y+y^\prime)]-[(x,y)]-[(x,y^\prime)]$
\item $[(rx,y)]-r[(x,y)]$
\item $[(x,ry)]-r[(x,y)]$
\end{enumerate}
for $x,x^\prime\in M$, $y,y^\prime\in N$, and $r\in R$. Now the composite
$M\times N\to R\langle M\times N\rangle/S$ {\em is} $R$-bilinear - we've quotiented out by all things that prevented it from being so! And the map $R\langle M\times N\rangle\to Q$ factors as $R\langle M\times N\rangle\to R\langle M\times N\rangle/S\xrightarrow{f} Q$, where $f$ is $R$-linear, and uniquely because the map to the quotient is surjective. This completes the construction.
\end{construction}

If you find yourself using this construction, stop and think about what you're doing. You're never going to use this construction to compute anything. 
Here's an example: for any abelian group $A$, 
\[
A\times \Z/n\Z\to A/nA\,,\quad (a,b)\mapsto ba\mod nA
\]
is clearly bilinear, and is universal as such. Just look: If 
$\beta:A\times\Z/n\Z\to Q$ is bilinear then 
$\beta(na,b)=n\beta(a,b)=\beta(a,nb)=\beta(a,0)=0$, so $\beta$ factors through 
$A/nA$; and $A\times\Z/n\Z\to A/nA$ is surjective. So $A\otimes\Z/n\Z=A/nA$.
\begin{remark}
The image of $M\times N$ in $R\langle M\times N\rangle/S$ generates it as an $R$-module. These elements $x\otimes y$ are called ``decomposable tensors.'' 
\end{remark}
What are the properties of such a universal bilinear map? 

\begin{property}[Uniqueness]
Suppose $\beta_0:M\times N\to P$ and $\beta_0':M\times N\to P'$ are both universal. Then there's a linear map $f:P\to P'$ such that $\beta_0'=f\beta_0$ and a linear map $f':P'\to P$ such that $\beta_0=f'\beta_0'$. 
The composite $f'f:P\to P$ is a linear map such that $f'f\beta_0=f'\beta_0'=\beta_0$. The identity map is another. But by universality, there's only one such linear map, so $f'f=1_P$. An identical argument shows that $ff'=1_{P'}$ as well, so they are inverse linear isomorphism. In brief: 
\begin{quote}
The target of a univeral $R$-bilinear map $\beta_0:M\times N\to P$ is unique up to a unique $R$-linear isomorphism compatible with the map $\beta_0$.
\end{quote}
This entitles us to speak of ``the'' universal bilinear map out of $M\times N$,
and give the target a symbol: $M\otimes_R N$. If $R$ is the ring of integers, or otherwise understood, we will drop it from the notation. 
\end{property}

\begin{property}[Functoriality] Suppose $f:M\to M'$ and $:N\to N'$. Study the diagram
\begin{equation*}
\xymatrix{M\times N\ar[d]^{f\times g}\ar[r]^\otimes\ar[dr] & M\otimes N\ar@{-->}[d]^{f\otimes g}\\
M^\prime\times N^\prime\ar[r]^\otimes & M^\prime\otimes N^\prime}
\end{equation*}
There is a unique $R$-linear map $f\otimes g$ because the diagonal map 
is $R$-bilinear and the map $M\times N\to M\otimes N$ is the universal
$R$-bilinear map out of $M\times N$. 
You are invited to show that this construction is functorial. 
\end{property}

\begin{property}[Unitality, associativity, commutativity] 
I said that this was going to be a ``categorical ring,'' so we should check various properties of the tensor product. For example, $R\otimes_R M$ should be isomorphic to $M$. Let's think about this for a minute. We have an $R$-bilinear map $R\times M\to M$, given by multiplication. 
We just need to check the universal property. Suppose we have an $R$-bilinear map $\beta:R\times M\to P$. We have to construct a map $f:M\to P$ such that 
$\beta(r,x)=f(rx)$ and show it's unique. Our only choice is $f(x)=\beta(1,x)$,
and that works.

Similarly, we should check that there's a unique isomorphism $L\otimes(M\otimes N)\xrightarrow{\cong}(L\otimes M)\otimes N$ that's compatible with $L\times (M\times N)\cong (L\times M)\times N$, and that there's a unique isomorphism $M\otimes N\to N\otimes M$ that's compatible with the switch map $M\times N\to N\times M$. There are a few other things to check, too: Have fun!
\end{property}

\begin{property}[Sums]
What happens with $M\otimes\left(\bigoplus_{\alpha\in A}N_\alpha\right)$? This might be a finite direct sum, or maybe an uncountable collection. How does this relate to $\bigoplus_{\alpha\in A}(M\otimes N_\alpha)$? Let's construct a map 
\[
f:\bigoplus_{\alpha\in A}\left(M\otimes N_\alpha\right)\to 
M\otimes\left(\bigoplus_{\alpha\in A}N_\alpha\right)\,.
\]
We just need to define maps $M\otimes N_\alpha\to M\otimes\left(\bigoplus_{\alpha\in A}N_\alpha\right)$ because the direct sum is the coproduct. We can use $1\otimes\text{in}_\alpha$ where $\mathrm{in}_\alpha:N_\alpha\to \bigoplus_{\alpha\in A}N_\alpha$. These give you a map $f$. 

What about a map the other way? We'll define a map out of the tensor product
using the universal property. So we need to define a bilinear map out of
$M\times\left(\bigoplus_{\alpha\in A}N_\alpha\right)$. By linearity in the
second factor, it will suffice to say where to send elements of the form
$(x,y)\in M\otimes N_\beta$. Just send it to $x\otimes\mathrm{in}_\beta y$, 
where $\mathrm{in}_\beta:N_\beta\to \bigoplus_{\alpha\in A}N_\alpha$ 
is the inclusion of a summand. 
It's up to you to check that these are inverses.
\end{property}

\begin{property}[Distributivity] 
Suppose $f:M'\to M$, $r\in R$, and $g_0,g_1:N'\to N$. Then
\[
f\otimes(g_0+g_1)=f\otimes g_0+f\otimes g_1:M'\otimes N'\to M\otimes N
\]
and
\[
f\otimes rg_0=r(f\otimes g_0):M'\otimes N'\to M\otimes N\,.
\]
Again I'll leave this to you to check. 
\end{property}

Our immediate use of this construction is to give a clean definition of 
``homology with coefficients in $M$,'' where $M$ is any abelian group. 
First, endow singular chains with coefficients in $M$ like this:
\[
S_\ast(X;M)=S_\ast(X)\otimes M
\]
Then we define
\[
H_n(X;M)=H_n(S_\ast(X;M))\,.
\]
Since $S_n(X)=\ZZ\Sin_n(X)$, $S_n(X;M)$ is a direct sum of copies of $M$
indexed by the $n$-simplices in $X$. If $M$ happens to be a ring, this coincides 
with the notation used in the last lecture. The boundary maps are just 
$d\otimes 1:S_n(X)\otimes M\to S_{n-1}(X)\otimes M$.

As we have noted, the sequence 
\[
0\to S_n(A)\to S_n(X)\to S_n(X,A)\to0
\]
is split short exact, and therefore applying the functor $-\otimes M$ 
to it produces another split short exact sequence. So 
\[
S_n(X,A)\otimes M=S_n(A;M)/S_n(X;M)\,,
\]
and it makes sense to use the notation $S_n(X,A;M)$ for this. This is
again a chain complex (by functoriality of the tensor product), and we define
\[
H_n(X,A;M)=H_n(S_n(X,A;M))\,.
\]

Notice that
\[
H_n(\ast;M)=
\begin{cases}M & \hbox{for}\,\, n=0\\0 & \hbox{otherwise}\,.\end{cases}
\]
The following result is immediate: 
\begin{prop}
For any abelian group $M$, $(X,A)\mapsto H_*(X,A;M)$ provides a homology 
theory satisfying the Eilenberg-Steenrod axioms with $H_0(\ast;M)=M$.
\end{prop}

Suppose $R$ is a commutative ring and $A$ is an abelian group. Then $A\otimes R$ is naturally an $R$-module. So $S_*(X;R)$ is a chain complex of $R$-modules -- {\em free} $R$-modules. We can go a little further: suppose that $M$ is an $R$-module. Then $A\otimes M$ is an $R$-module; and $S_*(X;M)$ is a chain complex of $R$-modules. We can also write 
\[
S_*(X;M)=S_*(X;R)\otimes_RM\,.
\]

This construction is natural in the $R$-module $M$; and, again using the fact
that sums of exact sequences are exact, a short exact sequence
of $R$-modules
\[
0\to M'\to M\to M''\to 0
\]
leads to a short exact sequence of chain complexes
\[
0\to S_*(X;M')\to S_*(X;M)\to S_*(X;M'')\to0
\]
and hence to a long exact sequence in homology, a ``coefficient long exact
sequence'':
\[
\xymatrix{ 
& \cdots \ar[r] & H_{n+1}(X;M'') \ar[dll]_\partial \\
H_n(X;M') \ar[r] & H_n(X;M) \ar[r] & H_n(X;M'') \ar[dll]_\partial \\
H_{n-1}(X;M') \ar[r] & \cdots\,.
}\]

A particularly important case is when $R$ is a field; then $S_*(X;R)$ is a 
chain complex of vector spaces over $R$, and $H_*(X;R)$ is a graded vector
space over $R$.

\begin{question}
A reasonable question is this: Suppose we know $H_*(X)$. Can we compute 
$H_*(X;M)$ for an abelian group $M$? More generally, suppose we know $H_*(X;R)$ and $M$ is an $R$-module. Can we compute $H_*(X;M)$?
\end{question}

