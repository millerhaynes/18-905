\section{Hom and Lim}\label{limits}

We will now deveop more properties of the tensor product: its relationship to
homomorphisms and to direct limits. 

The tensor product arose in our study of bilinear maps. Even more natural are
{\em linear maps}. Given a commutative ring $R$ and two $R$-modules $M$ and 
$N$, we can think about the collection of all $R$-linear maps from $M$ to $N$.
Not only does this set form an abelian group (under pointwise addition of 
homomorphisms); it forms an $R$-module, with
\[
(rf)(y)=f(ry)=rf(y)\,,\quad r\in R,\, y\in M\,.
\]
The check that this is again an $R$-module homomorphism uses commutativity of
$R$. We will write $\underline{\mathrm{Hom}}_R(M,N)$, or just $\Hom(M,N)$, 
for this $R$-module. 

Since $\Hom(M,N)$ is an $R$-module, we are entitled to think about what 
an $R$-module homomorphism into it is. Given 
\[
f:L\to\Hom(M,N)
\]
we can define a new function 
\[
\hat f:L\times M\to N\,,\quad \hat f(x,y)=(f(x))(y)\in N\,.
\]
You should check that this new function $\hat f$ is $R$-bilinear! So we get 
a natural map
\[
\Hom(L,\Hom(M,N))\to\Hom(L\otimes M,N)\,.
\]

Conversely, given a map $\hat f:L\otimes M\to N$ and $x\in L$, we can define
$f(x):M\to N$ by the same formula. These are inverse operations, so:
\begin{lemma}
The natural map $\Hom(L,\Hom(M,N))\to\Hom(L\otimes M,N)$ is an isomorphism.
\end{lemma}

One says that $\otimes$ and $\Hom$ are {\em adjoint}, a word suggested by 
Sammy Eilenberg to Dan Kan, who first formulated this relationship between
functors. 

This relationship makes a number of things obvious. For example, it's clear 
that 
\[
\Hom(\bigoplus_\alpha M_\alpha,N)=\prod_\alpha\Hom(M_\alpha,N)
\]
and this implies that the tensor product distributes over arbitrary
direct sums. We'll see another example in a minute. 

The second thing we will discuss is a generalization of one perspective on
how the rational numbers are constructed from the integers -- by a limit 
process: there are compatible maps in the diagram 
\[
\xymatrix{
\Z \ar[r]^2 \ar[d]^1 & \Z \ar[r]^3 \ar[dl]^{1/2} & \Z \ar[r]^4 \ar[dll]^{1/3!}
& \Z \ar[r]^5 \ar[dlll]^{1/4!} & \cdots \\
\QQ 
}\]
and $\QQ$ is the ``universal,'' or ``initial,'' abelian group you can put
in that position. 

We will formalize this process, using partially ordered sets as indexing sets.
Recall that a {\em partially ordered set}, or {\em poset}, is a small category $\cI$ such that $\#\cI(i,j)\leq 1$ and the only isomorphisms are the identity maps. We will be interested in a particular class of posets.
\begin{definition}
A poset $(\cI,\leq)$ is \emph{directed} if for every $i,j$, there exists a $k$ such that $i\leq k$ and $j\leq k$.
\end{definition}
\begin{example}
This is a very common condition.
For example, the natural numbers $\NN$ with inequality. Another example: if $X$ is a space and $I$ is the set of open subsets of $X$. It's directed by saying that $U\leq V$ if $U\subseteq V$. This is because $U,U^\prime$ need not be comparable, but $U,U^\prime\subseteq U\cup U^\prime$. Another example is the positive natural numbers, with $i\leq j$ if $i|j$. This is because $i,j|(ij)$.
\end{example}
\begin{definition}
Let $\cI$ be a directed set. An $\cI$-{\em directed diagram} in  a category $\cc$ is a functor $\cI\to\cc$. This means that for every $i\in \cI$ we are given an object $X_i\in\cc$, and for every $i\leq j$ we are given a map $X_i\xrightarrow{f_{i,j}} X_j$, in such a way that $f_{i,i}=1_{X_i}$ and if $i\leq j\leq k$
then $f_{i,k}=f_{j,k}\circ f_{i,j}:X_i\to X_k$. 
\end{definition}
\begin{example}\label{linear}
If $\cI=(\NN,\leq)$, then you get a ``linear system'' $X_0\xrightarrow{f_{01}}X_1\xrightarrow{f_{12}}X_2\to\cdots$. 
\end{example}
\begin{example}
Suppose $\cI=(\NN_{>0},|)$, i.e., the third example above. You can consider $\cI\to\mathbf{Ab}$, say assigning to each $i$ the integers $\Z$, and $f_{ij}:\Z\xrightarrow{j/i}\Z$. This seems like a more ``choice-free'' way to get at a construction of $\QQ$ out of $\ZZ$. 
\end{example}
These directed systems can be a little complicated. But there's a simple one, namely the constant one. 
\begin{example}
Let $\cI$ be any directed system. Any object $A\in\cc$ determines an $\cI$-directed set, namely the constant functor $c_A:\cI\to\cc$.
\end{example}
Not every directed system is constant, but we can try to find a best approximating constant system. To compare systems, we need morphisms. 
Of course, $\cI$-directed systems in $\cc$ are functors $\cI\to\cc$. They have natural transformations, and those are the morphisms in the category of $\cI$-directed systems. That is to say, a morphism is a choice of map $g_i:X_i\to Y_i$, for each $i\in\cI$, such that 
\begin{equation*}
\xymatrix{X_i\ar[r]\ar[d]^{g_i} & X_j\ar[d]^{g_j}\\
Y_i\ar[r] & Y_j}
\end{equation*}
commutes for all $i\leq j$.

\begin{definition}
Let $X:\cI\to\cc$ be a directed system.
A {\em direct limit} is an object $L$ and a map $X\to c_L$ that is initial among maps to constant systems. This means that given any other map to a constant system, say $X\to c_A$, there is a unique map $f:L\to A$ such that 
\[
\xymatrix@R=8pt{
& c_L \ar[dd]^{c_f} \\
X \ar[ur] \ar[dr] \\
& c_A
}\]
commutes.
\end{definition}
This is a universal property. So two different direct limits are canonically isomorphic; but a directed system may fail to have a direct limit. For example, the linear directed systems we used to creat the rational numbers exist in the category of finitely generated abelian groups; but $\QQ$ is not finitely generated, and there's no finitely generated group that will serve as a direct limit in the category of finitely generated abelian groups. 
\begin{example}
Suppose we have an increasing sequence of subspaces, 
$X_0\subseteq X_1\subseteq\cdots\subseteq X$. This gives us a directed system
of spaces, directed by the poset $(\NN,\leq)$. It's pretty clear that as a 
{\em set} the direct limit of this system is the union of the subspaces. 
Suppose that union is $X$. Then saying that $X$ is the direct limit of this 
directed system of spaces is saying that the topology on $X$ is determined
by the topology on the subspaces; it's the ``weak topology,'' characterized
by the property that a map $f:X\to Y$ is continuous if and only if the 
restriction of $f$ to each $X_n$ is continuous. This is saying that a subset 
of $X$ is open if and only if its intersection with each $X_n$ is open in $X$.
\end{example}
Direct limits may be constructed from the material of coproducts and quotients. So suppose $X:\cI\to\cc$ is a directed system. To construct the direct limit, begin by forming the coproduct over the elements of $\cI$,
\[
\coprod_{i\in\cI}X_i\,.
\]
There are maps $\mathrm{in}_i:X_i\to\coprod X_i$, but they are not yet compatible with the order relation in $\cI$. Form a quotient of the coproduct to enforce that compatibility: 
\[
\varinjlim_{i\in\cI}X_i=\left(\coprod_{i\in\cI}X_i\right)/\sim
\]
where $\sim$ is the equivalence relation generated by requiring that 
for any $i\in\cI$ and any $x\in X_i$, 
\[
\mathrm{in}_ix\sim\mathrm{in}_j f_{ij}(x)\,.
\]
The process of forming the coproduct and the quotient will depend upon the 
category you are working in, and may not be possible. 
In sets, coproduct is disjoint union and the
quotient just forms equivalence classes. In abelian groups, the coproduct
is the direct sum and to form the quotient you divide by the subgroup 
generated by differences. 

\begin{prop}
Let $\cI$ be a direct set, and let $M:\cI\to\mathbf{Mod}_R$ be a $\cI$-directed system of $R$-modules. There is a natural isomorphism 
\[
(\varinjlim_I M_i)\otimes_R N\cong \varinjlim_I (M_i\otimes_R N)\,.
\]
\end{prop}
\begin{proof}
Let's verify that both sides satisfy the same universal property. 
A map from $\varinjlim_I M_i)\otimes_R N$ to an $R$-module $L$ is the same
thing as a linear map $\varinjlim_IM_i\to\Hom_R(N,L)$. This is the same as
a compatible family of maps $M_i\to\Hom_R(N,L)$, which in turn is the same
as a compatible family of maps $M_i\otimes_RN\to L$, which is the same as
a linear map $\varinjlim_I(M_i\otimes_RN)\to L$. 
\end{proof}

Here's a lemma that lets us identify when a map to a constant functor is a
direct limit.
\begin{lemma}
Let $X:\cI\to\mathbf{Ab}$ (or $\mathbf{Mod}_R$). A map $f:X\to c_L$ (given
by $f_i:X_i\to L$ for $i\in\cI$) is the direct limit if and only if:
\begin{enumerate}
\item For every $x\in L$, there exists an $i$ and an $x_i\in X_i$ such that $f_i(x_i)=x$.
\item Let $x_i\in X_i$ be such that $f_i(x_i)=0$ in $L$. Then there exists some $j\geq i$ such that $f_{ij}(x_i)=0$ in $X_j$.
\end{enumerate}
\end{lemma}
\begin{proof}
Straightforward.
\end{proof}
This lemma generalizes the observation that $\QQ$ is the colimit of the diagram we drew above for $\cI=(\NN_{>0},|)$. 
\begin{prop}
The direct limit functor $\varinjlim_I:\Fun(\cI,\mathbf{Ab})\to\mathbf{Ab}$ is exact. In other words, if $X\xrightarrow{p}Y\xrightarrow{q}Z$ is an exact sequence of $\cI$-directed systems (meaning that at every degree we get an exact sequence of abelian groups), then $\varinjlim_IX\rightarrow \varinjlim_IY\rightarrow\varinjlim_IZ$ is exact.
\end{prop}
\begin{proof}
First of all, $pi:X\to Z$ is zero, which is to say that it factors through the constant zero object, so $\varinjlim_IX\to \varinjlim_I Z$ is certainly the zero map. Let $y\in \varinjlim_IY$, and suppose $y$ maps to $0$ in $\varinjlim_IZ$. By condition (1), there exists $i$ such that $y=f_i(y_i)$ for some $y_i\in Y_i$. Then $0=q(y)=f_iq(y_i)$ because $q$ is a map of direct systems. This means that there is $j\geq i$ such that $f_{ij}q(y_i)=0$ in $Z_j$. So $qf_{ij}y_i=0$,
again because $q$ is a map of direct systems. We have an element in $Y_j$ that maps to zero under $q$, so there is some $x_j\in X_j$ such that $p(x_j)=y_j$.
Then $f_j(x_j)\in \varinjlim_IX$ maps to $y$.
\end{proof}

The exactness of the direct limit has many useful consequences. For example:
\begin{corollary}
Let $i\mapsto C(i)$ be a directed system of chain complexes. 
Then there is a natural isomorphism 
\[
\varinjlim_{i\in\cI} H_*(C(i))\to H_*(\varinjlim_{i\in\cI}C(i))\,.
\]
\end{corollary}
Putting together things we have just said: 
\begin{corollary}
$H_*(X;\QQ)=H_*(X)\otimes\QQ$.
\end{corollary}
So we can redefine the Betti numbers of a space $X$ as 
\[
\beta_n=\dim_\QQ H_n(X;\QQ)
\]
and discuss the Euler characteristic entirely in terms of the rational vector
spaces making up the rational homology of $X$. 






