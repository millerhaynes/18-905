\section{Tensor and Tor}

We continue to study properties of the tensor product. Recall that 
\[
A\otimes \Z/n\Z=A/nA\,.
\]
Consider the exact sequence 
\[
0\to \Z\xrightarrow{2}\Z\to \Z/2\Z\to 0\,.
\]
Let's tensor it with $\Z/2\Z$. We get
\[
0\to \Z/2\Z\to\Z/2\Z\to\Z/2\Z\to 0\,.
\]
This cannot be a short exact sequence! 
This is a major tragedy: tensoring doesn't preserve exact sequences; one says that $\Z/n\Z\otimes-$ is not ``exact.'' This is why we can't form homology with coefficients in $M$ by simply tensoring homology with $M$. 

Tensoring does respect certain exact sequences:
\begin{prop}
The functor $N\mapsto M\otimes_R N$ preserves cokernels; it is \emph{right exact}. 
\end{prop}
\begin{proof}
Suppose that $N'\to N\to N''\to0$ is exact and let $f:M\otimes N\to Q$.  
We wish to show that there is a unique factorization as shown in the diagram
\[
\xymatrix{ 
M\otimes N' \ar[r] \ar[dr]^0 & M\otimes N \ar[d]^f \ar[r] &
M\otimes N'' \ar[r] \ar@{.>}[dl] & 0 \\
& Q\,.
}\]
This is equivalent to asking whether there is a unique factorization
of the corresponding diagram of bilinear maps,
\[
\xymatrix{
M\times N' \ar[r] \ar[dr]^0 & M\times N \ar[d]^\beta \ar[r] &
M\times N'' \ar[r] \ar@{.>}[dl] & 0 \\
& Q 
}\]
-- uniqueness of the linear factorization is guaranteed by the fact that
$M\times N''$ generates $M\otimes N''$. This unique factorization reflects
the fact that $M\times-$ preserves cokernels. 
\end{proof}

Failure of exactness is bad, so let's try to repair it. A key observation is that if 
$M$ is {\em free}, then $M\otimes_R-$ is exact. If $M=R\langle S\rangle$, the
free $R$-module on a set $S$, then $M\otimes_RN=\oplus_SN$, since tensoring
distributes over direct sums. Then we remember the following ``obvious'' fact:
\begin{lemma}
If $M'_i\to M_i\to M''_i$ is exact for all $i\in I$, then so is 
\[
\bigoplus M'_i\to\bigoplus M_i\to\bigoplus M''_i\,.
\]
\end{lemma}
\begin{proof} Clearly the composite is zero. Let 
$(x_i\in M_i,i\in I)\in\bigoplus M_i$ and suppose it maps to zero. 
That means that each $x_i$ maps to zero in $M''_i$ and hence is in the
image of some $x'_i\in M'_i$. Just make sure to take $x'_i=0$ if $x_i=0$. 
\end{proof}

To exploit this observation, we'll ``resolve'' $M$ by free modules. This means:
find a surjection from a free $R$-module, $F_0\to M$. This amounts to specifying $R$-module generators. For a general ring $R$, the kernel of $F_0\to M$ may
not be free. 
Let's make sure that it is by assuming that $R$ is a PID, and write $F_1$ for
the kernel. The failure of $M\otimes-$ to be exact is measured, at least
partially, by the leftmost term (defined as a kernel) in the exact sequence
\[
0\to\Tor_1^R(M,N)\to F_1\otimes_RN\to F_0\otimes_RN\to M\otimes_RN\to0\,.
\]

The notation suggests that this Tor term is independent of the resolution. 
This is indeed the case, as we shall show presently. But before we do, let's 
compute some Tor groups. 

\begin{example} 
For any PID $R$, if $M=F$ is free over $R$ we can take $F_0=F$
and $F_1=0$, and discover that then $\Tor_1^R(F,N)=0$ for any $N$. 
\end{example}
\begin{example} 
Let $R=\Z$ and $M=\Z/n\Z$, and $N$ any abelian group. When $R=\Z$ it is often
omitted from the notation for Tor.
There is a nice free resolution staring at us:
$F_0=F_1=\Z$, and $F_1\to F_0$ given by multiplication by $n$. The sequence
looks like 
\[
0\to\Tor_1(\Z/n\Z,N)\to\Z\otimes N\xrightarrow{n\otimes1}\Z\otimes N\to
\Z/n\Z\otimes N\to 0\,,
\]
so
\[
\Z/n\Z\otimes N=N/nN\,,\quad\Tor_1(\Z/n\Z,N)=\ker(n|N)\,.
\]
The torsion in this case is the ``$n$-torsion'' in $N$. This accounts for
the name.
\end{example}

Functors like $\Tor_1$ can be usefully defined for any ring, and moving to that
general case makes their significance clearer and illuminates the reason why 
$\Tor_1$ is independent of choice of generators. 

So let $R$ be any ring and $M$ a module over it. By picking $R$-module
generators I can produce a surjection from a free $R$-module, $F_0\to M$.
Write $K_0$ for the kernel of this map. It is the module of relations among
the generators. We can no longer guarantee that it's free, but we can at 
least find a set of module generators for it, and construct a surjection
from a free $R$-module, $F_1\to K_0$. Continuing in this way, we get a 
diagram like this -- 
\begin{equation*}
\xymatrix{\cdots\ar[rr]\ar[dr] && F_2\ar[dr]\ar[rr]^d & & 
F_1\ar[dr]\ar[rr]^d & & F_0\ar[dr]\\
& K_2\ar[ur]\ar[dr] & & K_1\ar[ur]\ar[dr] & & K_0\ar[ur]\ar[dr] & & N\ar[dr]\\
0\ar[ur] & & 0\ar[ur] & & 0\ar[ur] & & 0\ar[ur] & & 0}
\end{equation*}
-- in which the upside-down V subdiagrams are short exact sequences
and $F_s$ is free for all $s$. Splicing these exact sequences gives you an 
exact sequence in the top row. This is a \emph{free resolution of $N$}.
The top row, $F_*$, is a chain complex. It maps to the very short chain
complex with $N$ in degree 0 and 0 elsewhere, and this chain map is a 
homology isomorphism (or ``quasi-isomorphism''). We have in effect replaced
$N$ with this chain complex of free modules. The module $N$ may be very
complicated, with generators, relations, relations between relations \ldots.
All this is laid out in front of us by the free resolution. Generators of $F_0$
map to generators for $N$, and generators for $F_1$ map to relations among
those generators. 

Now we can try to define higher Tor functors by tensoring $F_*$ with $N$
and taking homology. If $R$ is a PID and the resolution is just $F_1\to F_0$,
forming homology is precisely taking cokernel and kernel, as we did above. 
In general, we define
\[
\Tor^R_n(M,N)=H_n(M\otimes_R F_\ast)\,.
\]

In the next lecture we will check that this is well-defined -- independent 
of free resolution, and functorial in the arguments. For the moment, notice 
that \[
\Tor^R_n(M,F)=0\quad\hbox{for}\, n>0 \quad\hbox{if $F$ is free}\,,
\]
since I can take $F\xleftarrow{\cong}F\leftarrow0\leftarrow\cdots$ as a
free resolution; and that 
\[
\Tor^R_0(M,N)=M\otimes_RN
\]
since we know that $M\otimes_R-$ is right-exact.

