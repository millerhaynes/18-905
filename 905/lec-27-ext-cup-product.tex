\section{Ext and UCT}
Let $R$ be a ring (probably a PID) and $N$ an $R$-module. The singular cochains
on $X$ with values in $N$, 
\[S^\ast(X;N)=\Map(\Sin_\ast(X),N)\,,
\]
then forms a cochain complex of $R$-modules. It is contravariantly functorial
in $X$ and covariantly functorial in $N$. The Kronecker pairing defines a map
\[
H^n(X;N)\otimes_R H_n(X;R)\to N
\]
whose adjoint 
\[
\beta:H^n(X;N)\rightarrow\Hom_R( H_n(X;R),N)
\]
gives us an estimate of the cohomology in terms of the homology of $X$. 
Here's how well it does:
\begin{theorem}[Mixed variance Universal Coefficient Theorem]
\label{thm-mvuct}
Let $R$ be a PID and $N$ an $R$-module, and let $C_*$ 
be a chain-complex of free $R$-modules. Then there is a short exact sequence
of $R$-modules,
\[
0\to\Ext_R^1(H_{n-1}(C_*),N)\to H^n(\Hom_R(C_*,N))\to\Hom_R(H_n(C_*),N)\to0\,,
\]
natural in $C_*$ and $N$, that splits (but not naturally). 
\end{theorem}
Taking $C_*=S_*(X;R)$, we have the short exact sequence
\begin{equation*}
0\to\Ext^1_R( H_{n-1}(X;R),N)\to H^n(X;N)\xrightarrow{\beta}\Hom_R( H_n(X;R),N)\to 0
\end{equation*}
that splits, but not naturally. This also holds for relative cohomology.

What is this Ext?

The problem that arises is that $\Hom_R(-,N):\mathbf{Mod}_R\to\mathbf{Mod}_R$ is not exact. 
Suppose I have an injection $M^\prime\to M$. Is $\Hom(M,N)\to\Hom(M^\prime,N)$ surjective? Does a map $M^\prime\to N$ necessarily extend to a map $M\to N$? No! For example, $\Z/2\Z\hookrightarrow\Z/4\Z$ is an injection, but the identity
map $\Z/2\Z\to\Z/2\Z$ does not extend over $\Z/4\Z$. 

On the other hand, if $M^\prime\xrightarrow{i} M\xrightarrow{p} M^{\prime\prime}\to 0$ is an exact sequence of $R$-modules then 
\[
0\to \Hom_R(M^{\prime\prime},N)\to\Hom_R(M,N)\to \Hom_R(M^{\prime},N)
\]
is again exact. Check this statement! 

Now homological algebra comes to the rescue to repair the failure of exactness. Pick a free resolution of $M$,
\[
0\leftarrow M\leftarrow F_0\leftarrow F_2\leftarrow\cdots\,.
\]
Apply $\Hom(-,N)$ to get a cochain complex 
\[
0\to \Hom_R(F_0,N)\to \Hom_R(F_1,N)\to \Hom_R(F_2,N)\to\cdots\,.
\]
\begin{definition} 
$\Ext_R^n(M,N)= H^n(\Hom_R(F_\ast,N))$.
\end{definition}
\begin{remark}
$\Ext$ is well-defined and functorial, by the Fundamental Theorem of Homological Algebra, Theorem \ref{thm-ftha}. If $M$ is free (or projective) then $\Ext_R^n(M,-)=0$ for $n>0$, since we can take $M$ as its own projective resolution. If $R$ is a PID, then we can assume $F_1=\ker(F_0\to M)$ and $F_n=0$ for $n>1$, so $\Ext_R^n=0$ if $n>1$. If $R$ is a field, then $\Ext_R^n=0$ for $n>0$. 
\end{remark}

\begin{example}
Let $R=\Z$ and take $M=\Z/k\Z$. This admits a simple free resolution:
$0\to \Z\xrightarrow{k}\Z\to\Z/k\Z\to 0$. Apply $\Hom(-,N)$ to it,
and remember that $\Hom(\ZZ,N)=N$, to get 
the very short cochain complex, with entries in dimensions 0 and 1:
\[
0\to N\xrightarrow{k} N\to0\,.
\]
Taking homology gives us
\[
\Hom(\Z/k\Z,N)=\ker(k|N)\,\quad\Ext^1(\Z/k\Z,N)=N/kN\,.
\]
\end{example}

\begin{proof}[Proof of Theorem \ref{thm-mvuct}]
First of all, we can't just copy the proof (in Lecture 24) of the
homology universal coefficient theorem, since $\Ext^1_R(-,R)$ is not
generally trivial. 

Instead, we start by thinking about what an $n$-cocycle in $\Hom_R(C_*,N)$ is:
it's a homomorphism $C_n\to N$ such that the composite $C_{n+1}\to C_n\to N$ 
is trivial. Write $B_n\subseteq C_n$ for the submodule of boundaries.
We have a homomorphism that kills $B_n$; that is,
\[
Z^n(\Hom_R(C_*,N)\xrightarrow{\cong}\Hom_R(C_n/B_n,N)\,.
\]
Now $H_n(C_*)$ (which we'll abbreviate as $H_n$) is the submodule
$Z_n/B_n$ of $C_n/B_n$; we have an exact sequence
\begin{equation*}
0\to H_n\to C_n/B_n\to B_{n-1}\to0\,.
\end{equation*}
Apply $\Hom_R(-,N)$ to this short exact sequence. The result is again 
short exact, because $B_{n-1}$ is a submodule of the free $R$-module $C_{n-1}$
and hence is free. This gives us the bottom line in the map of short
exact sequences
\[
\xymatrix{
0 \ar[r] & B^n\Hom_R(C_*,N) \ar[r] \ar[d] & 
Z^n\Hom_R(C_*,N) \ar[d]^\cong \ar[r]
& H^n(\Hom_R(C_*,N)) \ar[d]^\beta \ar[r] & 0\\
0 \ar[r] & \Hom_R(B_{n-1},N) \ar[r] & \Hom_R(C_n/B_n,N) \ar[r] & 
\Hom_R(H_n,N) \ar[r] & 0\,.
}\]
The map $\beta$ is the one we started with. The snake lemma
now shows that it is surjective and that
\[
\ker\beta\cong\coker(B^n\Hom_R(C_*,N)\to\Hom_R(B_{n-1},N))\,.
\]

An element of $B^n\Hom_R(C_*,N)$ is a map $C_n\to N$ that factors as
$C_n\xrightarrow{d}C_{n-1}\to N$. The observation is now that this is
the same as factoring as $C_n\xrightarrow{d}Z_{n-1}\to N$;
once this factorization has been achieved, the map $Z_{n-1}\to N$ 
automatically extends to all of $C_{n-1}$. This is because  
$Z_{n-}\subseteq C_{n-1}$ as a direct summand: the short exact sequence 
\[
0\to Z_{n-1}\to C_{n-1}\to B_{n-2}\to0
\]
splits since $B_{n-2}$ is free. Consequently we can rewrite our forumula for
$\ker\beta$ as 
\[
\ker\beta\cong\coker(\Hom_R(Z_{n-1},N)\to\Hom_R(B_{n-1},N))\,.
\]
But after all 
\[
0\leftarrow H_{n-1}\leftarrow Z_{n-1}\leftarrow B_{n-1}\leftarrow0
\]
is a free resolution, so this cokernel is precisely 
$\Ext_R^1(H_{n-1}(C_*),N)$.
\end{proof}

\begin{remark}
\textbf{Question:} Why is $\Ext$ called Ext?

\noindent
\textbf{Answer:} It classifies extensions. Let $R$ be a commutative ring, and let $M,N$ be two $R$-modules. I can think about ``extensions of $M$ by $N$,''
that is, short exact sequences of the form
\[
0\to N\to L\to M\to 0\,.
\]
For example, I have two extensions of $\Z/2\Z$ by $\Z/2\Z$: 
\[
0\to\Z/2\Z\to\Z/2\Z\oplus\Z/2\Z\to\Z/2\Z\to 0
\]
and
\[0\to \Z/2\Z\to\Z/4\Z\to\Z/2\Z\to 0\,.
\]
We'll say that two extensions are {\em equivalent} if there's a map of short exact sequences between them that is the identity on $N$ and on $M$. The two extensions above aren't equivalent, for example.

Another definition of $\Ext^1_R(M,N)$ is: the set of extensions like this modulo this notion of equivalence. The zero in the group is the split extension.
\end{remark}

The universal coefficient theorem is useful in transferring properties of 
homology to cohomology. For example, if $f:X\to Y$ is a map that induces an
isomorphism in $H_*(-;R)$, then it induces an isomorphism in $H^*(-;N)$ for
any $R$-module $N$, at least provided that $R$ is a PID. (This is in fact 
true in general.) 

Cohomology satisfies the appropriate analogues of the Eilenberg-Steenrod axioms.

\noindent
\textbf{Homotopy invariance:}
If $f_0\simeq f_1:(X,A)\to (Y,B)$, then 
\[
f_0^*=f_1^*:H^\ast(Y,B;N)\to H^\ast(X,A;N)\,.
\]
I can't use the UCT to address this. But we established a chain homotopy $f_{0,\ast}\simeq f_{1,\ast}:S_\ast(X,A)\to S_\ast(Y,B)$, and applying $\Hom$ converts chain homotopies to cochain homotopies. 

\noindent
\textbf{Excision:} If $U\subseteq A\subseteq X$ such that $\overline{U}\subseteq\mathrm{Int}(A)$, then $ H^\ast(X,A;N)\to H^\ast(X-U,A-U;N)$ is an isomorphism. This follows from excision in homology and the mixed variance UCT.

\noindent
\textbf{Milnor axiom:} The inclusions induce an isomorphism
\[
H^*(\coprod_\alpha X_\alpha;N)\to \prod_\alpha H^*(X_\alpha;N)\,.
\]

As a result, it enjoys the fruit of these axioms, such as:

\noindent
\textbf{The Mayer-Vietoris sequence:} If $A,B\subseteq X$ are such that their interiors cover $X$, then there is a long exact sequence
\begin{equation*}
\xymatrix{ 
 H^{n+1}(X;N)\ar[r] & \cdots \\
 H^n(X;N)\ar[r] & H^n(A;N)\oplus H^n(B;N)\ar[r] & H^n(A\cap B;N)\ar[ull]\\
& \cdots \ar[r] & H^{n-1}(A\cap B;N) \ar[ull]
}
\end{equation*}







