\section{Ext and UCT}
Let $R$ be a ring (probably a PID) and $N$ an $R$-module. The singular cochains
on $X$ with values in $N$, 
\[S^\ast(X;N)=\Map(\Sin_\ast(X),N)\,,
\]
then forms a cochain complex of $R$-modules. It is contravariantly functorial
in $X$ and covariantly functorial in $N$. The Kronecker pairing defines a map
\[
H^n(X;N)\otimes_R H_n(X;R)\to N
\]
whose adjoint 
\[
\beta:H^n(X;N)\rightarrow\Hom_R( H_n(X;R),N)
\]
gives us an estimate of the cohomology in terms of the homology of $X$. 
Here's how well it does:
\begin{theorem}[Mixed variance Universal Coefficient Theorem]
\label{thm-mvuct}
Let $R$ be a PID and $N$ an $R$-module, and let $C_*$ 
be a chain-complex of free $R$-modules. Then there is a short exact sequence
of $R$-modules,
\[
0\to\Ext_R^1(H_{n-1}(C_*),N)\to H^n(\Hom_R(C_*,N))\to\Hom_R(H_n(C_*),N)\to0\,,
\]
natural in $C_*$ and $N$, that splits (but not naturally). 
\end{theorem}
Taking $C_*=S_*(X;R)$, we the short exact sequence
\begin{equation*}
0\to\Ext^1_R( H_{n-1}(X;R),N)\to H^n(X;N)\xrightarrow{\beta}\Hom_R( H_n(X;R),N)\to 0
\end{equation*}
that splits, but not naturally. This also holds for relative cohomology.

What is this Ext?

The problem that arises is that $\Hom_R(-,N):\mathbf{Mod}_R\to\mathbf{Mod}_R$ is not exact. 
Suppose I have an injection $M^\prime\to M$. Is $\Hom(M,N)\to\Hom(M^\prime,N)$ surjective? Does a map $M^\prime\to N$ necessarily extend to a map $M\to N$? No! For example, $\Z/2\Z\hookrightarrow\Z/4\Z$ is an injection, but the identity
map $\Z/2\Z\to\Z/2\Z$ does not extend over $\Z/4\Z$. 

On the other hand, if $M^\prime\xrightarrow{i} M\xrightarrow{p} M^{\prime\prime}\to 0$ is an exact sequence of $R$-modules then 
\[
0\to \Hom_R(M^{\prime\prime},N)\to\Hom_R(M,N)\to \Hom_R(M^{\prime},N)
\]
is again exact. Check this statement! 

Now homological algebra comes to the rescue to repair the failure of exactness! Pick a free resolution of $M$,
\[
\cdots\to F_2\to F_1\to F_0\to M\to 0\,.
\]
Apply $\Hom$ to get a chain complex 
\[
0\to \Hom(F_0,N)\to \Hom(F_1,N)\to \Hom(F_2,N)\to\cdots\,.
\]
\begin{definition} 
$\Ext_R^n(M,N)= H^n(\Hom_R(F_\ast,N))$.
\end{definition}
\begin{remark}
$\Ext$ is well-defined and functorial, by the fundamental lemma of homological algebra. If $M$ is free (or projective) then $\Ext^n(M,-)=0$ for $n>0$, since we can take $M$ as its own projective resolution. If $R$ is a PID, then we can assume $F_1=\ker(F_0\to M)$ and $F_n=0$ for $n>1$, so $\Ext^n=0$ if $n>1$. If $R$ is a field, then $\Ext^n=0$ for $n>0$. 
\end{remark}

\begin{example}
Let $R=\Z$ and take $M=\Z/k\Z$. This admits a simple free resolution:
$0\to \Z\xrightarrow{k}\Z\to\Z/k\Z\to 0$. Apply $\Hom(-,N)$ to it,
and remember that $\Hom(\ZZ,N)=N$, to get 
the very short cochain complex, with entries in dimensions 0 and 1:
\[
0\to N\xrightarrow{k} N\to0\,.
\]
Taking homology gives us
\[
\Hom(\Z/k\Z,N)=\ker(k|N)\,\quad\Ext^1(\Z/k\Z,N)=N/kN\,.
\]
\end{example}



\begin{proof} of Theorem \ref{thm-mvuct}
First of all, notice that 
\[
C_n/Z_n\cong B_{n-1}
\]
is a submodule $C_{n-1}$ and hence is free. Thus both of the following short
exact sequences split:
\begin{equation}
0\to Z_n\to C_n\to C_n/Z_n\to0
\label{one}
\end{equation}
\begin{equation}
0\to Z_n/B_n\to C_n/B_n\to C_n/Z_n\to0\,.
\label{two}
\end{equation}
Note that the second one can be rewritten as
\begin{equation*}
0\to H_n\to C_n/B_n\to B_{n-1}\to0\,.
\end{equation*}
Start with the diagram
\[
\xymatrix{
0 \ar[r] & B^n\Hom(C_*,N) \ar[r] \ar@{.>}[d] & Z^n\Hom(C_*,N) \ar[d]^\cong \ar[r]
& H^n(\Hom(C_*,N)) \ar@{.>}[d] \ar[r] & 0\\
0 \ar[r] & \Hom(B_{n-1},N) \ar[r] & \Hom(C_n/B_n,N) \ar[r] & 
\Hom(H_n,N) \ar[r] & 0
}\]
The bottom row arises from \eqref{two} and is exact because 
\eqref{two} splits.
The middle arrow starts with $f:C_n\to N$ such that 
$C_{n+1}\xrightarrow{d}C_n\xrightarrow{f}N$ 
is zero. This condition is equivalent to requiring that $f$ kill boundaries,
and so it factors through a unique map $C_n/B_n\to N$. 

We claim that the composite $B^n\Hom(C_*,N)\to\Hom(H_n,N)$ is trivial.
So start with $f:C_n\to N$ such that $C_{n+1}\to C_n\to N$ is trivial. 
Then $f$ kills $B_n$ and so factors through $C_n/B_n$, giving an element 
of $\Hom(C_n/B_n,N)$; but it also kills the larger submodule $Z_n$, and
hence factors through $C_n/Z_n$. This implies that the composite
$H_n\to C_n/B_n\to N$ is trivial since $H_n\to C_n/B_n\to C_n/Z_n$ is. 

So we can fill in the maps to get a map of short exact sequences. By the
snake lemma, the right arrow is surjective and its kernel $K$ fits into the 
short exact sequence at the top of the following diagram.
\[
\xymatrix{
0 \ar[r] & B^n\Hom(C_*,N) \ar[r] \ar@{.>>}[d] & \Hom(B_{n-1},N) \ar[d]^= \ar[r]
& K \ar[r] \ar@{.>}[d] & 0 \\
0 \ar[r] & I \ar[r] & \Hom(B_{n-1},N) \ar[r] & \Ext^1(H_{n-1},N) \ar[r] & 0 
}\]
The bottom row of this diagram comes from the long exact sequence 
associated to the short exact sequence 
\[
0\to B_{n-1}\to Z_{n-1}\to H_{n-1}\to0\,,
\]
so
\[
I=\img\left(\Hom(Z_{n-1},N)\to\Hom(B_{n-1},N)\right)\,.
\]

We claim that there is a surjection on the left as shown (making the square
commutative). This completes the proof, since the snake lemma then implies
that the right arrow is an isomorphism.

The left arrow occurs at the right of the diagram
\[
\xymatrix{
0 \ar[r] & Z^{n-1}\Hom(C_*,N) \ar@{.>}[d] \ar[r] & 
\Hom(C_{n-1},N) \ar@{->>}[d] \ar[r] & B^n \ar@{.>}[d] \ar[r] & 0 \\
0 \ar[r] & \Hom(H_{n-1},N) \ar[r] & \Hom(Z_{n-1},N) \ar[r] & I \ar[r] & 0 
}\]
Here the middle surjection is induced from the split injection 
$Z_{n-1}\to C_{n-1}$. We need to construct the left arrow; this will
finish the proof, since then the right arrow exists and is surjective again by
the snake lemma. 

So let $f:C_{n-1}\to N$ be such that $C_n\to C_{n-1}\to N$ is trivial. Its 
image under the vertical surjection is the composite $Z_{n-1}\to C_{n-1}\to N$.
Now 
\[
\xymatrix{
C_n \ar@{->>}[d] \ar[r] & C_{n-1} \ar[r] & N \\
B_{n-1} \ar[r] & Z_{n-1} \ar[u] \ar[ur] \ar[r] & H_{n-1} \ar@{.>}[u] \ar[r] & 0 
}\]
The composite $B_{n-1}\to N$ is trivial since $C_n\to B_{n-1}$ is surjective,
and the desired factorization through $H_{n-1}$ follows. 
\end{proof}

\begin{remark}
\textbf{Question:} Why is $\Ext$ called Ext?

\noindent
\textbf{Answer:} It classifies extensions. Let $R$ be a commutative ring, and let $M,N$ be two $R$-modules. I can think about ``extensions of $M$ by $N$, that is, short exact sequences of the form
\[
0\to N\to L\to M\to 0\,.
\]
For example, I have two extensions of $\Z/2\Z$ by $\Z/2\Z$: 
\[
0\to\Z/2\Z\to\Z/2\Z\oplus\Z/2\Z\to\Z/2\Z\to 0
\]
and
\[0\to \Z/2\Z\to\Z/4\Z\to\Z/2\Z\to 0\,.
\]
We'll say that two extensions are equivalent if there's a map of short exact sequences between them is the identity on $N$ and $M$. The two extensions above aren't equivalent, for example.

Another definition of $\Ext^1_R(M,N)$ is the set of extensions like this modulo this notion of equivalence. The zero in the group is the split extension.
\end{remark}

The universal coefficient theorem is useful in transferring properties of 
homology to cohomology. For example, if $f:X\to Y$ is a map that induces an
isomorphism in $H_*(-;R)$, then it induces an isomorphism in $H^*(-;N)$ for
any $R$-module $N$, at least provided that $R$ is a PID. (This is true in 
general, however.) 

Cohomology satisfies the appropriate analogues of the Eilenber Steenrod axioms.

\noindent
\textbf{Homotopy invariance:}
If $f_0\sim f_1:(X,A)\to (Y,B)$, then 
\[
f_0^*=f_1^*:H^\ast(Y,B;N)\to H^\ast(X,A;N)\,.
\]
I can't use the UCT to address this because the UCT only tells you that things are isomorphic. But we did establish a chain homotopy $f_{0,\ast}\sim f_{1,\ast}:S_\ast(X,A)\to S_\ast(Y,B)$, and applying $\Hom$ converts chain homotopies to cochain homotopies. 

\noindent
\textbf{Excision:} If $U\subseteq A\subseteq X$ such that $\overline{U}\subseteq\mathrm{Int}(A)$, then $ H^\ast(X,A;N)\to H^\ast(X-U,A-U;N)$ is an isomorphism. This follows from excision in homology and by the mixed variance UCT.

\noindent
\textbf{Mayer-Vietoris sequence:} If $A,B\subseteq X$ are such that their interiors cover $X$, then there is a long exact sequence
\begin{equation*}
\xymatrix{ & & \cdots\ar[dll]\\
 H^n(X;N)\ar[r] & H^n(A;N)\oplus H^n(B;N)\ar[r] & H^n(A\cap B;N)\ar[dll]\\
 H^{n+1}(X;N)\ar[r] & \cdots & }
\end{equation*}

\noindent
\textbf{Milnor axiom:} The inclusions induce an isomorphism
\[
H^*(\coprod_\alpha X_\alpha;N)\to \prod_\alpha H^*(X_\alpha;N)\,.
\]





