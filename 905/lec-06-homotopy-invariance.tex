\section{Homotopy invariance of homology}
We now know that the homology of a star-shaped region is trivial: in such a space, every cycle with augmentation 0 is a boundary. We will use that fact, which is a special case of homotopy invariance of homology, to prove the general result, which we state in somewhat stronger form:
	\begin{theorem}
A homotopy $h:f_0\simeq f_1:X\to Y$ determines a natural chain homotopy
$f_{0,\ast}\simeq f_{1,\ast}:S_\ast(X)\to S_\ast(Y)$.
	\end{theorem}

The proof uses naturality (a lot). For a start, notice that if $k:g_0\simeq g_1:C_*\to D_*$ is a chain homotopy, and $j:D_*\to E_*$ is another chain map, then the composites $j\circ k_n:C_n\to E_{n+1}$ give a chain homootpy $j\circ g_0\simeq j\circ g_1$. So if we can produce a chain homotopy between the chain maps induced by the two inclusions $i_0,i_1:X\to X\times I$, we can get a chain homotopy $k$ between $f_{0*}=h_*\circ i_{0*}$ and 
$f_{1*}=h_*\circ i_{1*}$ in the form $h_*\circ k$. 

So now we want to produce a natural chain homotopy, with components 
$k_n:S_n(X)\to S_{n+1}(X\times I)$. The unit interval hosts a natural 1-simplex
given by an identification $\Delta^1\to I$, and we should imagine $k$ as being
given by ``multiplying'' by that 1-chain. This ``multiplication'' is a special
case of a chain map 
\[
\times:S_*(X)\times S_*(Y)\to S_*(X\times Y)\,,
\]
defined for any two spaces $X$ and $Y$, 
with lots of good properties. It will ultimately be used to compute the homology of a product of two spaces in terms of the homology groups of the factors. 

Here's the general result. 
\begin{theorem}
There exists a map $\times:S_p(X)\times S_q(Y)\to S_{p+q}(X\times Y)$, 
the {\em cross product}, that is:
	\begin{itemize}
	\item Natural, in the sense that if $f:X\to X^\prime$ and $g:Y\to Y^\prime$, and $a\in S_p(X)$ and $b\in S_p(Y)$ so that $a\times b\in S_{p+q}(X\times Y)$, then $f_\ast(a)\times g_\ast(b)=(f\times g)_\ast(a\times b)$.
	\item Bilinear, in the sense that $(a+a^\prime)\times b=(a\times b)+(a^\prime\times b)$, and $a\times (b+b^\prime)=a\times b+a\times b^\prime$.
	\item The Leibniz rule is satisfied, i.e., $d(a\times b)=(da)\times b + (-1)^pa\times db$.
	\item Normalized, in the following sense. Let $x\in X$ and $y\in Y$. Write $j_x:Y\to X\times Y$ for $y\mapsto (x,y)$, and write $i_y:X\to X\times Y$ 
for $x\mapsto (x,y)$. If $b\in S_q(Y)$, then $c^0_x\times b=(j_x)_\ast b\in S_q(X\times Y)$, and if $a\in S_p(X)$, then $a\times c^0_y=(i_y)_\ast a\in S_p(X\times Y)$.
	\end{itemize}
\end{theorem}
The Leibniz rule contains the first occurence of the 
``topologist's sign rule'';
we'll see these signs appearing often. Watch for when it appears in our proof.

\begin{proof} We're going to use induction on $p+q$; the normalization axiom 
gives us the cases $p+q=0,1$. Let's assume that we've constructed the cross-product in total dimension $p+q-1$. We want to define $\sigma\times\tau$ for 
$\sigma\in S_p(X)$ and $\tau\in S_q(Y)$. 

Note that there's a universal example of a $p$-simplex, namely the identity map $\iota_p:\Delta^p\to \Delta^p$. It's universal in the sense any $p$-simplex $\sigma:\Delta^p\to X$ can be written as $\sigma_\ast(\iota_p)$ where $\sigma_\ast:\Sin_p(\Delta^p)\to \Sin_p(X)$ is the map induced by $\sigma$. To define $\sigma\times\tau$ in general, then, it suffices to define $\iota_p\times\iota_q\in S_{p+q}(\Delta^p\times\Delta^q)$; we can (and must) then take
$\sigma\times\tau=(\sigma\times\tau)_*(\iota_p\times\iota_q)$. 

Our long list of axioms is useful in the induction. For one thing, if $p=0$ or
$q=0$, normalization provides us with a choice. So now assume that both $p$ and $q$ are positive. We want the cross-product
to satisfy the Leibnitz rule: 
		\begin{equation*}
d(\iota_p\times\iota_q) = (d\iota_p)\times\iota_q + (-1)^p\iota_p\times d\iota_q\in		S_{p+q-1}(\Delta^p\times\Delta^q)
		\end{equation*}
Since $d^2=0$, a necessary condition for $\iota_p\times\iota_q$ to exist is that $d((d\iota_p)\times\iota_q + (-1)^p\iota_p\times d\iota_q) =0$. Let's compute what this is, using the Leibnitz rule in dimension
$p+q-1$ where we have it by the inductive assumption:
		\begin{equation*}
		d((d\iota_p)\times\iota_q + (-1)^p\iota_p\times(d\iota_q)) = (d^2\iota_p)\times\iota_q + (-1)^{p-1}(d\iota_p)\times(d\iota_q) + (-1)^p(d\iota_p)\times d\iota_q + (-1)^q\iota_p\times(d^2\iota_q) = 0
		\end{equation*}
because $d^2=0$. Note that this calculation would not have worked without the sign! 

The subspace $\Delta^p\times\Delta^q\subseteq\mathbf{R}^{p+1}\times\mathbf{R}^{q+1}$ is convex and nonempty, and hence star-shaped. Therefore we know that $ H_{p+q-1}(\Delta^p\times\Delta^q)=0$ (remember, $p+q>1$), which means that every cycle is a boundary. In other words, our necessary condition is also sufficient! So, choose any element 
with the right boundary and declare it to be $\iota_p\times\iota_q$.

The induction is now complete provided we can check that this choice satisfies naturality, bilinearity, and the Leibniz rule. I leave this as a relaxing exercise for the listener. 
\end{proof}

The essential point here is that the space supporting the universal pair of
simplices -- $\Delta^p\times\Delta^q$ -- has trivial homology. Naturality 
transports the result of that fact to the general situation. 
 
The cross-product that this procedure constructs is not unique; it depends on a choice a choice of the chain $\iota_p\times\iota_q$ for each pair $p,q$ with $p+q>1$. The cone construction in the proof that star-shaped regions have vanishing homology provids us with a specific choice; but it turns out that any two choices are equivalent up to natural chain homotopy. 

We return to homotopy invariance. To define our chain homotopy
$h_X:S_n(X)\to S_{n+1}(X\times I)$, pick any 1-simplex $\iota:\Delta^1\to I$
such that $d_0\iota=1$ and $d_1\iota=0$, and define
\[
h_X\sigma =(-1)^n\sigma\times\iota\,.
\]
Let's compute:
		\begin{equation*}
		dh_X\sigma =(-1)^nd(\sigma\times \iota) = 
(-1)^n(d\sigma)\times\iota + \sigma\times(d\iota)
		\end{equation*}
But $d\iota = c_1^0 - c_0^0\in S_0(I)$, 
which means that we can continue (remembering that $|\partial\sigma|=n-1$):
\[
=-h_X d\sigma+(\sigma\times c_1^0-\sigma\times c_0^0)
=-h_X d\sigma+(\iota_{1*}\sigma-\iota_{0*}\sigma)\,,
\]
using the normalization axiom of the cross-product. This is the result.


