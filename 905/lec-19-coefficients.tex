\section{Coefficients}

Abelian groups can be quite complicated, even finitely generated ones. Vector spaces over a field are so much simpler! A vector space is determined up to isomorphism by a single cardinality, its dimension. Wouldn't it be great to have a version of homology that took values in the category of vector spaces over a field? 

We can do this, and more. Let $R$ be any commutative ring at all.
Instead of forming the free abelian group on $\Sin_*(X)$, we could just as 
form the free $R$-module:
\[
S_*(X;R)=R\Sin_*(X)
\]
This gives, first, a simplicial object in the category of $R$-modules.
Forming the alternating sum of the face maps produces a chain complex 
{\em of $R$-modules}: $S_n(X;R)$ is an $R$-module for each $n$, and 
$d:S_n(X;R)\to S_{n-1}(X;R)$ is an $R$-module homomorphism. The homology
groups are then again $R$-modules:
\[
H_n(X;R)=\frac{\ker(d:S_n(X;R)\to S_{n-1}(X;R))}
{\img(d:S_{n+1}(X;R)\to S_n(X;R))}\,.
\]

This is the {\em singular homology of $X$ with coefficients in the commutative 
ring $R$}. It satisfies all the Eilenberg-Steenrod axioms, but 
\[
H_n(\ast;R)=
\begin{cases}R\,\,&\hbox{for}\quad n=0\\0\,\,&\hbox{otherwise}\,.\end{cases}
\]
We could actually have replaced the ring $R$ by any abelian group here, 
but this will become much clearer after we have the tensor product as a tool.

The coefficient rings that are most important in algebraic topology are 
simple ones: the integers and the prime fields $\FF_p$ and $\QQ$; 
typically, a PID.

As an experiment, let's compute $ H_\ast(\RP^n;R)$ for various rings $R$.
Let's start with $R=\FF_2$, the field with 2 elements. This is a favorite
among algebraic topologists, because using it for coefficients eliminates 
all sign issues. The cellular chain complex has $S_k(\RP^n;\FF_2)=\FF_2$ for 
$0\leq k\leq n$, and the differential alternates between multiplication by 
2 and by 0. But in $\FF_2$, $2=0$: so $d=0$, and the cellular chains
coincide with the homology:
\[
H_k(\RP^n;\FF_2)=
\begin{cases}\FF_2\,&\hbox{for}\quad 0\leq k\leq n\\0\,\,&\hbox{otherwise}\,.
\end{cases}
\]

On the other hand, suppose that $R$ is a ring in which $2$ is invertible.
The universal case is $\Z[1/2]$, but any subring of the rationals containing
$1/2$ would do just as well, as would $\FF_p$ for $p$ odd. 
Now the cellular chain complex (in dimensions 0 through $n$)
looks like
\[
R\xleftarrow{0}R\xleftarrow{\cong}R\xleftarrow{0}R
\xleftarrow{\cong}\cdots\xleftarrow{\cong}R
\]
for $n$ even, and
\[
R\xleftarrow{0}R\xleftarrow{\cong}R\xleftarrow{0}R
\xleftarrow{\cong}\cdots\xleftarrow{0}R
\]
for $n$ odd. Therefore
\[
H_k(\RP^n;R)=
\begin{cases}R\,&\hbox{for}\quad k=0\\0\,&\,\hbox{otherwise}\end{cases}
\]
for $n$ even, and 
\[
H_k(\RP^n;R)=
\begin{cases}R\,&\hbox{for}\quad k=0\\
R\,&\,\hbox{for}\quad k=n\\
0\,&\,\hbox{otherwise}\,.
\end{cases}
\]
You get a much simpler result: Away from 2, even projective spaces look like a point, and odd projective spaces look like a sphere!

I'd like to generalize this process a little bit, and allow coefficients 
not just in
a commutative ring, but more generally in a module $M$ over a commutative ring;
in particular, any abelian group. This is most cleanly done using the
mechinism of the tensor product. That mechanism will also let us address
the following natural question: 
\begin{question}
Given $H_*(X;R)$, can we deduce $H_*(X;M)$ for an $R$-module $M$?
\end{question}
The answer is called the ``universal coefficient theorem''. I'll spend a few days developing what we need to talk about this.



