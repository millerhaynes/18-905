\documentclass[12pt]{article}

\usepackage{amssymb,amsmath}
\usepackage{xypic}

\textwidth 6in
\textheight 9in
\topmargin -.5in
\oddsidemargin .25in
\evensidemargin .25in
\parskip 3pt
\parindent 0pt
\pagestyle{empty}


\newtheorem{theorem}{Theorem}[section]
\newtheorem{proposition}[theorem]{Proposition}
\newtheorem{lemma}[theorem]{Lemma}
\newtheorem{definition}[theorem]{Definition}
\newtheorem{examples}[theorem]{Examples}
\newtheorem{remarks}[theorem]{Remarks}
\newtheorem{corollary}[theorem]{Corollary}
\newtheorem{remark}[theorem]{Remark}
\newtheorem{example}[theorem]{Example}


\begin{document}

\thispagestyle{empty}

\def\da#1{\downarrow\rlap{$\vcenter{\hbox{$\scriptstyle#1$}}$}}
\def\ua#1{\uparrow\rlap{$\vcenter{\hbox{$\scriptstyle#1$}}$}}

\def\coker{\mathrm{coker}\,}
\def\im{\mathrm{im}\,}
\def\ker{\mathrm{ker}\,}
\def\NN{\mathbb N}
\def\ZZ{\mathbb Z}
\def\RR{\mathbf R}
\def\Ext{\mathrm{Ext}}
\def\Tor{\mathrm{Tor}}
\def\Hom{\mathrm{Hom}}
\def\Der{\mathrm{Der}}
\def\Map{\mathrm{Map}}
\def\Gp{\mathbf{Gp}}
\def\Mon{\mathbf{Mon}}
\def\mod{\hbox{mod}}
\def\be{\begin{equation}}
\def\ee{\end{equation}}
\def\tensor{\otimes}
\def\iso{\cong}
\def\Ho{\mathrm{Ho}\,}
\def\rin{\mathrm{in}}
\def\la#1{\mathop{\longleftarrow}\limits^{#1}}
\def\ra#1{\mathop{\longrightarrow}\limits^{#1}}
\def\bS{\mathbf{S}}

\def\inj{\mathrm{in}}
\def\pr{\mathrm{pr}}
\def\div{\mathrm{div}}
\def\grad{\mathrm{grad}}
\def\curl{\mathrm{curl}}
\def\Sin{\mathrm{Sin}}

\def\SF{\mathcal{C}^\infty}
\def\VF{\mathcal{VF}^\infty}


\def\TT{\mathbb{T}}
\def\Tensor{\bigotimes}
\def\bDelta{\mathbf{\Delta}}
\def\bSet{\mathbf{Set}}
\def\bAb{\mathbf{Ab}}
\def\bTop{\mathbf{Top}}
\def\bC{\mathbf{C}}
\def\ob{\mathrm{ob}}
\def\bVS{\mathbf{VS}}



\begin{center}
{\bf 18.905: Problem Set II}
\end{center}

Due October 5, 2016, in class. 

Homework is an important part of this class. I hope you gain from the
struggle. Collaboration can be effective, but be sure that you
grapple with each problem on your own as well. If you do work with others,
you must indicate with whom on your solution sheet. Scores will be posted
on the Stellar website.

\bigskip

{\bf 5. (a)} Let $A$ be a chain complex (of abelian groups). 
It is {\em acyclic} if $H(A)=0$, and 
{\em contractible} if it is chain-homotopy-equivalent to the trivial 
chain complex. Prove that a chain complex is contractible if and only
if it is acyclic and for every $n$ the inclusion 
$Z_nA\hookrightarrow A_n$ is a split monomorphism of abelian groups. 

{\bf(b)} Give an example of an acyclic chain complex that is not 
contractible. 

\medskip
{\bf 6.} Propose a construction of the product and the coproduct of 
two spaces in the homotopy category, and check that your proposal serves
the purpose.
 
\medskip
{\bf 7.(a)} Let $S$ and $T$ be sets and $A$ an abelian group. Establish 
a bijection between the set of maps of sets from $S\times T$ to $A$ and
the set of bilinear maps $\ZZ S\times\ZZ T\to A$.

{\bf(b)} For positive integers $m,n$, let $\ZZ/m$, $\ZZ/n$ denote the cyclic 
groups of order $m,n$.
Construct a surjective bilinear map $\mu:\ZZ/m\times\ZZ/n\to\ZZ/\gcd\{m,n\}$.
Show that any bilinear map $\ZZ/m\times\ZZ/n\to A$ factors uniquely as 
$f\circ\mu$ where $f:\ZZ/\gcd\{m,n\}\to A$ is a homomorphism.

\medskip
{\bf8. (a)} Let $0\ra{}A\ra{i}B\ra{p}C\ra{}0$ be a short exact sequence.
Show that the following three sets are in bijection with one another.

(i) The set of homomorphisms $\sigma:C\rightarrow B$ such that $p\sigma=1_C$.

(ii) The set of homomorphisms $\pi:B\rightarrow A$ such that $\pi i=1_A$.

(iii) The set of homomorphisms $\alpha:A\oplus C\rightarrow B$ such that 
$\alpha(a,0)=ia$ for all $a\in A$ and $p\alpha(a,c)=c$ for all 
$(a,c)\in A\oplus C$.

Moreover, show that any homomorphism as in (iii) is an isomorphism.

Any one of these structures 
is a {\em splitting} of the short exact sequence, and
the sequence is then said to be {\em split}.


{\bf(b)} Suppose that 
\[
\xymatrix{
\cdots \ar[r] & A_n \ar[r] \ar[d] & B_n \ar[r] \ar[d] & C_n \ar[r] \ar[d]^\iso
& A_{n-1} \ar[r] \ar[d] & \cdots \\
\cdots \ar[r] & A'_n \ar[r] & B'_n \ar[r] & C'_n \ar[r] & 
A'_{n-1} \ar[r] & \cdots
}
\]
is a ``ladder'': a map of long exact sequences. So both rows are exact
and each square commutes. Suppose also that every third vertical map is
an isomorphism, as indicated. Prove that these data naturally 
determine a long exact sequence 
\[
\cdots\ra{}A_n\ra{}A'_n\oplus B_n\ra{}B'_n\ra{}A_{n-1}\ra{}\cdots
\]

\medskip
{\bf 9. (a)} (``$3\times3$ lemma.'') Let  
\[
\xymatrix{
& 0 \ar[d] & 0 \ar[d] & 0 \ar[d] \\
0 \ar[r] & A' \ar[d] \ar[r] & B' \ar[d] \ar[r] & C' \ar[d] \ar[r] & 0 \\
0 \ar[r] & A \ar[d] \ar[r] & B \ar[d] \ar[r] & C \ar[d] \ar[r] & 0 \\
0 \ar[r] & A'' \ar[d] \ar[r] & B'' \ar[d] \ar[r] & C'' \ar[d] \ar[r] & 0 \\
& 0 & 0 & 0
}
\]
be a commutative diagram of abelian groups. Assume that all three
columns are exact, that all but one of the rows is exact,
and that the compositions in the remaining row are trivial.
Prove that the remaining row is also exact. (Hint: view each 
row as a chain complex \ldots.)

{\bf(b)} (``Long exact homology sequence of a triple.'')
Let $(C,B,A)$ be a ``triple,'' so $C$ is a space, $B$ is a
subspace of $C$, and $A$ is a subspace of $B$. Show that there  are
natural transformations $\partial:H_n(C,B)\rightarrow H_{n-1}(B,A)$ such 
that 
\[
\cdots\ra{}H_n(B,A)\ra{i_*}H_n(C,A)\ra{j_*}H_n(C,B)\ra{\partial}H_{n-1}(B,A)
\ra{}\cdots
\]
is exact, where $i:(B,A)\rightarrow(C,A)$ and $j:(C,A)\rightarrow(C,B)$ are
the inclusions of pairs. ({\bf 9 (a)} might be useful.)

\medskip
{\bf 10.} This exercise generalizes our computation of the homology of
spheres, and introduces several important constructions.

The {\em cone} on a space $X$ is the quotient space 
$CX=X\times I/X\times\{0\}$, where $I$ is the unit interval $[0,1]$. 
The cone is a pointed space, with basepoint $*$ given by the ``cone point,''
i.e. the image of $X\times\{0\}$. (By convention, the cone on the empty
space $\varnothing$
is a single point, the cone point.) Regard $X$ as the subspace of $CX$
of all points of the form $(x,1)$.

Define the {\em suspension} of a space $X$ to be $SX=CX/X$. 
Make $SX$ a pointed space by declaring
the image of $X\subseteq CX$ to be the basepoint in $SX$. 
(By convention, the quotient $W/\varnothing$ is the disjoint union of $W$ with
a single point, which is declared to be the basepoint. So 
$S\varnothing=*/\varnothing$ is the discrete two-point space, with the
new point as basepoint.)

The quotient map induces a map of pairs $f:(CX,X)\rightarrow(SX,*)$.

{\bf(a)} Show that $CX$ is contractible.

{\bf(b)} Show that there is a natural isomorphism 
$\widetilde{H}_{n-1}(X)\rightarrow H_n(SX,*)$, for any $n$.

\end{document}
 
