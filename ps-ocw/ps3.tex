\documentclass[12pt]{article}

\usepackage{amssymb,amsmath}
\usepackage{xypic}

\textwidth 6in
\textheight 9in
\topmargin -.5in
\oddsidemargin .25in
\evensidemargin .25in
\parskip 3pt
\parindent 0pt
\pagestyle{empty}


\newtheorem{theorem}{Theorem}[section]
\newtheorem{proposition}[theorem]{Proposition}
\newtheorem{lemma}[theorem]{Lemma}
\newtheorem{definition}[theorem]{Definition}
\newtheorem{examples}[theorem]{Examples}
\newtheorem{remarks}[theorem]{Remarks}
\newtheorem{corollary}[theorem]{Corollary}
\newtheorem{remark}[theorem]{Remark}
\newtheorem{example}[theorem]{Example}


\begin{document}

\thispagestyle{empty}

\def\da#1{\downarrow\rlap{$\vcenter{\hbox{$\scriptstyle#1$}}$}}
\def\ua#1{\uparrow\rlap{$\vcenter{\hbox{$\scriptstyle#1$}}$}}

\def\coker{\mathrm{coker}\,}
\def\im{\mathrm{im}\,}
\def\ker{\mathrm{ker}\,}
\def\NN{\mathbb N}
\def\ZZ{\mathbb Z}
\def\RR{\mathbf R}
\def\Ext{\mathrm{Ext}}
\def\Tor{\mathrm{Tor}}
\def\Hom{\mathrm{Hom}}
\def\Der{\mathrm{Der}}
\def\Map{\mathrm{Map}}
\def\Gp{\mathbf{Gp}}
\def\Mon{\mathbf{Mon}}
\def\mod{\hbox{mod}}
\def\be{\begin{equation}}
\def\ee{\end{equation}}
\def\tensor{\otimes}
\def\iso{\cong}
\def\Ho{\mathrm{Ho}\,}
\def\rin{\mathrm{in}}
\def\la#1{\mathop{\longleftarrow}\limits^{#1}}
\def\ra#1{\mathop{\longrightarrow}\limits^{#1}}
\def\bS{\mathbf{S}}

\def\inj{\mathrm{in}}
\def\pr{\mathrm{pr}}
\def\div{\mathrm{div}}
\def\grad{\mathrm{grad}}
\def\curl{\mathrm{curl}}
\def\Sin{\mathrm{Sin}}

\def\SF{\mathcal{C}^\infty}
\def\VF{\mathcal{VF}^\infty}


\def\TT{\mathbb{T}}
\def\Tensor{\bigotimes}
\def\bDelta{\mathbf{\Delta}}
\def\bSet{\mathbf{Set}}
\def\bAb{\mathbf{Ab}}
\def\bTop{\mathbf{Top}}
\def\bC{\mathbf{C}}
\def\ob{\mathrm{ob}}
\def\bVS{\mathbf{VS}}



\begin{center}
{\bf 18.905: Problem Set III}
\end{center}

Due October 19, 2016, in class. 

Homework is an important part of this class. I hope you gain from the
struggle. Collaboration can be effective, but be sure that you
grapple with each problem on your own as well. If you do work with others,
you must indicate with whom on your solution sheet. Scores will be posted
on the Stellar website.

\bigskip
{\bf 11. (a)} In class I discussed a monoid homomorphism
\[
\deg:[S^n,S^n]\to\ZZ_\times
\]
(where $\ZZ_\times$ is the integers with monoid structure given by 
multiplication, and $n\geq1$),
given by sending $f$ to its effect on $H_n(S^n)$. 
I asserted that it was surjective, but this was an induction based on
the case $n=1$. Regard $S^1$ as the unit circle in the complex plane. 
In class I claimed that the map $z\mapsto z^d$ has degree $d$.
Please verify this claim. 

{\bf (b)} Regard $S^{n-1}$ as the unit sphere in $\RR^n$. 
Let $L$ be a line through the origin in $\RR^n$, and $L^\perp$ its 
orthogonal complement. Let $\rho_L$ be the linear map given by $-1$ on $L$
and $+1$ on $L^\perp$. What is $\deg\rho_L$?

{\bf(c)} What is the degree of the ``antipodal map,'' 
$\alpha:S^{n-1}\to S^{n-1}$ sending $x$ to $-x$? 

{\bf(d)} The tangent space to a point $x$ on the sphere $S^{n-1}$  can be
regarded as the subspace of $\RR^n$ of vectors perpendicular to $x$. 
A ``vector field'' on $S^{n-1}$ is thus a continuous function 
$v:S^{n-1}\to\RR^n$ such that $v(x)\perp x$ for all $x\in S^{n-1n}$.

Show that if $n$ is odd then every vector field vanishes at some point on
the sphere. (When $n-1=2$, this is the ``hairy ball theorem.'')

On the other hand, construct a nowhere vanishing vector field on $S^{n-1}$
for any even $n$.

\medskip
{\bf 12.} Use the Mayer-Vietoris sequence to compute the homology groups of
the projective plane $P$, the Klein bottle $K$, and the torus $T$. 
(The projective
plane is obtained by sewing a disk onto a M\"obius band along their
boundaries. The Klein bottle is obtained either by sewing two M\"obius bands 
together, or by sewing the two boundary components of a cylinder together 
in a funny way. 
A torus is obtained by sewing the boundary components of a cylinder together
in a less funny way. In each case, it's a good idea to give yourself a hem: 
glue open ``collars'' together.)

{\bf(b)} Hopefully you computed that $H_2(T)$ is an infinite cyclic group.
Say something sensible about whether the ``fundamental class'' you 
constructed in Problem {\bf 2} is indeed a generator of that abelian group.

\medskip
{\bf 13.} $\varnothing$.

\medskip
{\bf 14.} 
The constructions sketched in Problem {\bf 12} are examples of the following
general procedure. Take two closed surfaces,
$\Sigma_1$ and $\Sigma_2$, cut a disk out from each one, and glue them 
together along the hem. This is the {\em connected sum} $\Sigma_1\#\Sigma_2$. 
Write $T_1$ for the torus, and $T_g=T_1\# T_{g-1}$. Write $P_1$ for the 
projective plane, and $P_g=P_1\#P_{g-1}$.
A theorem of Rado asserts that this is a complete list of compact connected 
2-manifolds. 

{\bf (a)} What is the Klein bottle, in this notation?

{\bf (b)} Complete the work from Problem {\bf 12}: compute the homology groups
of these closed surfaces. 

\end{document}
 



Recall the Eilenberg-Zilber chain: 
\[
\zeta_{p,q}=\sum_\omega(-1)^{|A|}\omega\in S_{p+q}(\Delta^p\times\Delta^q)
\]
where $\omega:[p+q]\to[p]\times[q]$ runs over the injections such that 
$\pr_1\omega:[p+q]\t[p]$ and $\pr_2\omega:[p+q]\to[q]$ are both order
preserving. Actually $\omega$ in the formula denotes the affine map 
$\Delta^{p+q}\to\Delta^p\times\Delta^q$ sending
verticies in the manner dictated by $\omega$; and $\pr_1,\pr_2$ denote
the two projections, $\pr_1:\Delta^p\times\Delta^q\to\Delta^p$ and
$\pr_2:\Delta^p\times\Delta^q\to\Delta^q$. 

Draw a picture of this chain when $(p,q)=(1,1)$ and $(p,q)=(1,2)$. 
Show that this chain has property
\[
\partial\zeta_{p,1}=\






 Show that a bilinear map $A\times\ZZ\to C$ is no more and 
no less than a homomorphism $A\to C$. 

Let $cI$ denote the following chain complex: 
$cI_0$ is the free abelian group generated by elements $e_0$ and $e_1$;
$cI_1$ is the free abelian group generated by an element $e$; all
other degrees are the zero group; and $\partial e=e_0-e_1$. 
Write $\ZZ$ for the chain complex with $\ZZ$ in degree 0 and zero everywhere
else. Let $i_0:\ZZ\to\cI$ send $1$ to $e_0$ and $i_1:\ZZ\to\cI$ send $1$ to 
$e_1$. 

(b) Let $A$ and $C$ be chain complexes, and $f_0,f_1:A\to C$ chain maps.
Show that a chain-homotopy $f_0\tilde f_1$ is no more and no less than


