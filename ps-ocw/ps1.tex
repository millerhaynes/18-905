\documentclass[12pt]{article}

\usepackage{amssymb,amsmath}

\textwidth 6in
\textheight 9in
\topmargin -.5in
\oddsidemargin .25in
\evensidemargin .25in
\parskip 3pt
\parindent 0pt
\pagestyle{empty}


\newtheorem{theorem}{Theorem}[section]
\newtheorem{proposition}[theorem]{Proposition}
\newtheorem{lemma}[theorem]{Lemma}
\newtheorem{definition}[theorem]{Definition}
\newtheorem{examples}[theorem]{Examples}
\newtheorem{remarks}[theorem]{Remarks}
\newtheorem{corollary}[theorem]{Corollary}
\newtheorem{remark}[theorem]{Remark}
\newtheorem{example}[theorem]{Example}


\begin{document}

\thispagestyle{empty}

\def\da#1{\downarrow\rlap{$\vcenter{\hbox{$\scriptstyle#1$}}$}}
\def\ua#1{\uparrow\rlap{$\vcenter{\hbox{$\scriptstyle#1$}}$}}

\def\coker{\mathrm{coker}\,}
\def\im{\mathrm{im}\,}
\def\ker{\mathrm{ker}\,}
\def\NN{\mathbb N}
\def\ZZ{\mathbb Z}
\def\RR{\mathbf R}
\def\Ext{\mathrm{Ext}}
\def\Tor{\mathrm{Tor}}
\def\Hom{\mathrm{Hom}}
\def\Der{\mathrm{Der}}
\def\Map{\mathrm{Map}}
\def\Gp{\mathbf{Gp}}
\def\Mon{\mathbf{Mon}}
\def\mod{\hbox{mod}}
\def\be{\begin{equation}}
\def\ee{\end{equation}}
\def\tensor{\otimes}
\def\iso{\cong}
\def\Ho{\mathrm{Ho}\,}
\def\rin{\mathrm{in}}
\def\la#1{\mathop{\longleftarrow}\limits^{#1}}
\def\ra#1{\mathop{\longrightarrow}\limits^{#1}}
\def\bS{\mathbf{S}}

\def\inj{\mathrm{in}}
\def\pr{\mathrm{pr}}
\def\div{\mathrm{div}}
\def\grad{\mathrm{grad}}
\def\curl{\mathrm{curl}}
\def\Sin{\mathrm{Sin}}

\def\SF{\mathcal{C}^\infty}
\def\VF{\mathcal{VF}^\infty}


\def\TT{\mathbb{T}}
\def\Tensor{\bigotimes}
\def\bDelta{\mathbf{\Delta}}
\def\bSet{\mathbf{Set}}
\def\bAb{\mathbf{Ab}}
\def\bTop{\mathbf{Top}}
\def\bC{\mathbf{C}}




\begin{center}
{\bf 18.905: Problem Set I}
\end{center}

Due September 21, 2016, in class. 

Homework is an important part of this class. I hope you gain from the
struggle. Collaboration can be effective, but be sure that you
grapple with each problem on your own as well. If you do work with others,
you must indicate with whom on your solution sheet. Scores will be posted
on the Stellar website.

\bigskip

{\bf 1. (a)} Let $[n]$ denote the totally ordered set $\{0,1,\ldots,n\}$. 
Let $\phi:[m]\rightarrow[n]$ be an order preserving function
(so that if $i\leq j$ then $\phi(i)\leq\phi(j)$). Identifying the elements of
$[n]$ with the vertices of the standard simplex $\Delta^n$, $\phi$ extends to
an affine map $\Delta^m\rightarrow\Delta^n$ that we also denote by $\phi$.
Give a formula for
this map in terms of barycentric coordinates: If 
$\phi(s_0,\ldots,s_m)=(t_0,\ldots,t_n)$, what is $t_j$ as a function
of $(s_0,\ldots,s_m)$?

{\bf (b)} Write $d^j:[n-1]\rightarrow[n]$ for the order preserving
injection that omits $j$ as a value. Show that an order preserving
injection $\phi:[n-k]\rightarrow[n]$ is uniquely a composition of the form
$d^{j_k}d^{j_{k-1}}\cdots d^{j_1}$, with $0\leq j_1<j_2<\cdots<j_k\leq n$.
Do this by describing the integers $j_1,\ldots,j_k$ directly in terms of 
$\phi$, and then verify the straightening rule 
\[
d^id^j=d^{j+1}d^i \quad\mathrm{for}\quad i\leq j
\]

{\bf (c)} Show that any order preserving map $\phi:[m]\to[n]$ factors
uniquely as the composition of an order preserving surjection followed by an 
order preserving injection

{\bf (d)} Write $s^i:[m+1]\rightarrow[m]$ for the order-preserving surjection 
that repeats the value $i$. Show that any order-preserving surjection 
$\phi:[m]\to[n]$ has a unique 
expression $(s^n)^{i_n}(s^{n-1})^{i_{n-1}}\cdots(s^0)^{i_0}$.
Do this by describing the numbers $i_0,\ldots,i_n$,
directly in terms of $\phi$, and finding a straightening rule
of the form $s^is^j=\cdots$ for $i<j$. 

{\bf(e)} 
Finally, implement your assertion that any order preserving map factors
as a surjection followed by an injection by establishing a straightening 
rule of the form $s^id^j=\cdots$.

Recall the notation $\Sin_n(X)$ for the set of continuous maps from $\Delta^n$ to the space $X$. 
The affine extension $\phi:\Delta^m\to\Delta^n$ of an order-preserving
map $\phi:[m]\to[m]$ induces a map $\phi^*:\Sin_n(X)\to\Sin_m(X)$.
In particular, write 
\[
d_i=(d^i)^*\quad\quad s_j=(s^j)^*\,.
\]
The $d_i$'s are {\em face maps}, the $s_i$'s are {\em degeneracies}.

{\bf(f)} 
Write down the identities satisfied by these operators, resulting from
the identities you found relating the $d^i$'s and $s^j$'s.

A {\em simplicial set} is a sequence of
sets $K_0,K_1,\ldots$, with maps $d_i:K_n\to K_{n-1}$, $0\leq i\leq n$, 
and $s_i:K_n\to K_{n+1}$, $0\leq i\leq n$, satisfying these identities.
For example, we have the {\em singular simplicial set} $\Sin_*(X)$
of a space $X$.

The group of $n$-chains on $X$ is the free abelian group generated by the
$n$-simplices:
\[
S_n(X)=\ZZ\Sin_n(X)\,.
\]
Each operator $\phi:\Sin_n(X)\to\Sin_m(X)$ extends uniquely to a homomorphism
$\phi:S_n(X)\to S_m(X)$. In particular, $d_i:S_n(X)\to S_{n-1}(X)$, 
$0\leq i\leq n$. We can combine these homomorphism to form the 
{\em boundary map} 
\[
d=\sum_{i=0}^n(-1)^id_i:S_n(X)\to S_{n-1}(X)\,.
\]

{\bf (g)} Use the relations among the $d_i$'s to prove that 
\[
d^2=0:S_n(X)\to S_{n-2}(X)\,.
\]

{\bf (h)} Let $f:X\to Y$ be a continuous map. By composition, we get a 
map $f_*:\Sin_n(X)\to\Sin_n(Y)$. Show that together these form a map of
simplicial sets (i.e. they commute with the maps induced by order preserving
maps $\phi:[m]\to[n]$).


\medskip
{\bf2. (a)} 
Write down a singular 2-cycle representing the ``fundamental class''
of the torus $T^2=S^1\times S^1$. We will give a precise definition of
the fundamental class of a manifold later, but for now let's just say that 
this cycle should be made up of singular 
2-simplices which together cover all but a small (e.g. nowhere dense) subset 
of $T^2$ exactly once. 

{\bf(b)} Construct an isomorphism
\[
H_n(X)\oplus H_n(Y)\to H_n(X\amalg Y)\,.
\]

\medskip
{\bf 3. (a)} Write $\pi_0(X)$ for the set of path-components of a space $X$.
Construct an isomorphism
\[
\ZZ\pi_0(X)\to H_0(X)\,.
\]

{\bf(b)} Say what it means to assert that the isomorphisms 
you constructed in {\bf 2(b)} and {\bf 3(a)} are {\em natural},
and make sure they are.

\medskip
{\bf 4.} Here are a couple more ``categorical'' definitions, giving you some
practice with the idea of constructions being defined by universal mapping
properties.  

Let $\bC$ be
a category, $A$ a set, and $a\mapsto X_a$ an assignment of an object of $\bC$
to each element of $A$. 
A {\em product} of these objects is an object $Y$ together with maps
$\pr_a:Y\to X_a$ with the following property. For any object $Z$ and 
any family of maps $f_a:Z\to X_a$, there is a unique map $Z\to Y$ such that
$f_a=\pr_a\circ f$ for all $a\in A$. 
A {\em coproduct} of these objects is an object $Y$ together with maps
$\inj_a:X_a\to Y$ with the following property. For any object $Z$ and 
any family of maps $f_a:X_a\to Z$, there is a unique map $Y\to Z$ such that
$f_a=f\circ \inj_a$ for all $a\in A$. 

{\bf (a)} Describe constructions of the product and coproduct (if they exist) 
in the following categories:
sets, pointed sets, spaces, abelian groups. 
(A {\em pointed set} is a pair $(S,*)$ where $S$ is a set and $*\in S$.)

{\bf (b)} What should be meant by the product when $A=\varnothing$? How about
the coproduct? What are these objects in the four categories mentioned in
{\bf(a)}? Give an example of a category in which neither one of these
constructions exists. 

{\bf (c)} Show that if $(Y,\{\pr_a\})$ and $(Y',\{\pr'_a\})$ are both products
of a family $\{X_a:a\in A\}$, then there is a unique map $f:Y\to Y'$ such that
$pr'_a\circ f=\pr_a$ for all $a\in A$, and that this map is an isomorphism. 

{\bf(d)} 
A partially ordered set $S$ defines a category as follows. The objects
are the elements of $S$. (They constitute a {\em set}, rather than 
something larger, so this is a ``small category.'') For $s,s'\in S$,
there is exactly one morphism $s\to s'$ if $s\leq s'$, and none otherwise.

Take $S$ to be the real numbers $\RR$ with their natural order, 
for example, and consider a map $A\to\RR$. Under what conditions does the 
product of these objects exist, and if it does what is it? Same question
for the coproduct. 




\end{document}
















A morphism $f:X\rightarrow Y$ is an {\em epimorphism} (``epi'' for short)
if for every object 
$Z$ and every pair of morphisms $u,v:Y\rightarrow Z$, $uf=vf\Rightarrow u=v$.

A morphism $f:X\rightarrow Y$ is a {\em monomorphism} (``mono'' for short)
if for every object
$W$ and evey pair of morphism $u,v:W\rightarrow X$, $fu=fv\Rightarrow u=v$.

{\bf(a)} Assume that $g:Y\rightarrow Z$ is mono. Show that $f:X\rightarrow Y$
is mono if and only if $gf$ is mono.  State and prove an analogous 
statement about epis. 

{\bf(b)} What are epis and monos in $\bSet$?

{\bf(c)} What are epis and monos in $\bAb$?

{\bf(d)} What are epis and monos in $\bTop$? (In considering epis,
it may be useful to use the space $T$ with two points, with the 
``indiscrete'' topology in which the only open sets are the empty
set and the entire space.)

{\bf(e)} Show that $\bTop$ provides an example of a category in which 
a morphism can be both epi and mono but not an iso (which is short for
{\em isomorphism}). 

\medskip
{\bf 5.} to follow


\end{document}



