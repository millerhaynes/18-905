\documentclass[12pt]{article}
\usepackage{amssymb,amsmath}
\usepackage{xypic}
\usepackage{hyperref}

\textwidth 6in
\textheight 9in
\topmargin -.5in
\oddsidemargin .25in
\evensidemargin .25in
\parskip 3pt
\parindent 0pt
\pagestyle{empty}


\newtheorem{theorem}{Theorem}[section]
\newtheorem{proposition}[theorem]{Proposition}
\newtheorem{lemma}[theorem]{Lemma}
\newtheorem{definition}[theorem]{Definition}
\newtheorem{examples}[theorem]{Examples}
\newtheorem{remarks}[theorem]{Remarks}
\newtheorem{corollary}[theorem]{Corollary}
\newtheorem{remark}[theorem]{Remark}
\newtheorem{example}[theorem]{Example}


\begin{document}

\thispagestyle{empty}

\def\da#1{\downarrow\rlap{$\vcenter{\hbox{$\scriptstyle#1$}}$}}
\def\ua#1{\uparrow\rlap{$\vcenter{\hbox{$\scriptstyle#1$}}$}}

\def\coker{\mathrm{coker}\,}
\def\im{\mathrm{im}\,}
\def\ker{\mathrm{ker}\,}
\def\NN{\mathbb N}
\def\ZZ{\mathbb Z}
\def\RR{\mathbb R}
\def\QQ{\mathbb Q} 
\def\CC{\mathbb C}
\def\FF{\mathbb F}
\def\Ext{\mathrm{Ext}}
\def\Tor{\mathrm{Tor}}
\def\Hom{\mathrm{Hom}}
\def\Der{\mathrm{Der}}
\def\Map{\mathrm{Map}}
\def\Gp{\mathbf{Gp}}
\def\Mon{\mathbf{Mon}}
\def\mod{\hbox{mod}}
\def\be{\begin{equation}}
\def\ee{\end{equation}}
\def\tensor{\otimes}
\def\iso{\cong}
\def\Ho{\mathrm{Ho}\,}
\def\rin{\mathrm{in}}
\def\Fun{\mathrm{Fun}}
\def\id{\mathrm{id}}
\def\nt{\mathrm{nt}}

\def\la#1{\mathop{\longleftarrow}\limits^{#1}}
\def\ra#1{\mathop{\longrightarrow}\limits^{#1}}


\def\bS{\mathbf{S}}

\def\inj{\mathrm{in}}
\def\pr{\mathrm{pr}}
\def\div{\mathrm{div}}
\def\grad{\mathrm{grad}}
\def\curl{\mathrm{curl}}
\def\Sin{\mathrm{Sin}}

\def\SF{\mathcal{C}^\infty}
\def\VF{\mathcal{VF}^\infty}


\def\TT{\mathbb{T}}
\def\Tensor{\bigotimes}
\def\bDelta{\mathbf{\Delta}}
\def\bSet{\mathbf{Set}}
\def\bAb{\mathbf{Ab}}
\def\bTop{\mathbf{Top}}
\def\bC{\mathbf{C}}
\def\ob{\mathrm{ob}}
\def\bVS{\mathbf{VS}}

\def\cP{\mathcal{P}}
\def\cE{\mathcal{E}}


\begin{center}
{\bf 18.905: Problem Set V}
\end{center}

Due November 16, 2016, in class. 

Homework is an important part of this class. I hope you gain from the
struggle. Collaboration can be effective, but be sure that you
grapple with each problem on your own as well. If you do work with others,
you must indicate with whom on your solution sheet. Scores will be posted
on the Stellar website. Extra credit for calling attention to mistakes!

\bigskip
{\bf 19. (a)} Verify the Lemma stated in lecture on Nov 4:
Let $I$ be a directed set, $L$ an abelian group, and $A:I\to\bAb$ an 
$I$-directed diagram of abelian groups, with bonding maps
$f_{ij}:A_i\to A_j$ for $i\leq j$. A map $A\to c_L$, given by 
compatible maps $f_i:A_i\to L$, is a direct limit if and only if:\\
(i) For any $b\in L$ there exists $i\in I$ and $a_i\in A_i$ such that 
$f_ia_i=b$, and\\
(ii) For any $a_i\in A_i$ such that $f_ia_i=0\in L$, there exists $j\geq i$
such that $f_{ij}a_i=0\in A_j$.

\medskip
{\bf 20. (a)}
Embed $\ZZ/p^n\ZZ$ into $\ZZ/p^{n+1}\ZZ$ by sending $1$ to $p$, 
and write $\ZZ_{p^\infty}$ for the union. It's called the Pr\"ufer group (at 
$p$). Show that $\ZZ_{p^\infty}\iso\ZZ[1/p]/\ZZ$ and that 
\[
\QQ/\ZZ\iso\bigoplus_p\ZZ_{p^\infty}
\]
where the sum runs over the prime numbers. 

{\bf(b)} Compute $\ZZ_{p^\infty}\tensor_\ZZ A$ for $A$ each of the following
abelian groups: $\ZZ/n\ZZ$, 
$\ZZ[1/q]$ (for $q$ a prime), and $\ZZ_{q^\infty}$ (for $q$ a prime). 

{\bf(c)} Compute $\Tor_1^{\ZZ}(M,\ZZ[1/p])$ and 
$\Tor_1^{\ZZ}(M,\ZZ_{p^\infty})$, for any abelian group $M$ in terms of the
self-map $p:M\to M$. 

\medskip
{\bf 21.} Show that if $f:X\to Y$ induces an isomorphism in homology 
with coefficients in
the prime fields $\FF_p$ (for all primes $p$) and $\QQ$, then it induces an
isomorphism in homology with coefficients in $\ZZ$. 
(Hint: {\bf 18 (c)}.)

\medskip
{\bf 22.} The construction of an isomorphism between the singular homology
and the cellular homology of a CW complex carries over {\em verbatim} with any
coefficients. Use this observation to compute the homology of $\RR P^n$ with
coefficients in $\QQ$, $\FF_p$, $\ZZ_{p^\infty}$, and $\ZZ[1/p]$. 

\medskip
{\bf 23.} $\varnothing$

\medskip
{\bf 24.} Suppose that $f:X\to Y$ induces an isomorphism in homology with
coefficients in $\ZZ/n\ZZ$. Show that it induces an isomorphism in homology
with coefficients in any abelian group in which every element is killed by
some power of $n$. 










\end{document}

