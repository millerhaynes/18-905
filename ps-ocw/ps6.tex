\documentclass[12pt]{article}
\usepackage{amssymb,amsmath}
\usepackage{xypic}
\usepackage{hyperref}

\textwidth 6in
\textheight 9in
\topmargin -.5in
\oddsidemargin .25in
\evensidemargin .25in
\parskip 3pt
\parindent 0pt
\pagestyle{empty}


\newtheorem{theorem}{Theorem}[section]
\newtheorem{proposition}[theorem]{Proposition}
\newtheorem{lemma}[theorem]{Lemma}
\newtheorem{definition}[theorem]{Definition}
\newtheorem{examples}[theorem]{Examples}
\newtheorem{remarks}[theorem]{Remarks}
\newtheorem{corollary}[theorem]{Corollary}
\newtheorem{remark}[theorem]{Remark}
\newtheorem{example}[theorem]{Example}


\begin{document}

\thispagestyle{empty}

\def\da#1{\downarrow\rlap{$\vcenter{\hbox{$\scriptstyle#1$}}$}}
\def\ua#1{\uparrow\rlap{$\vcenter{\hbox{$\scriptstyle#1$}}$}}

\def\coker{\mathrm{coker}\,}
\def\im{\mathrm{im}\,}
\def\ker{\mathrm{ker}\,}
\def\NN{\mathbb N}
\def\ZZ{\mathbb Z}
\def\RR{\mathbb R}
\def\QQ{\mathbb Q} 
\def\CC{\mathbb C}
\def\FF{\mathbb F}
\def\Ext{\mathrm{Ext}}
\def\Tor{\mathrm{Tor}}
\def\Hom{\mathrm{Hom}}
\def\Der{\mathrm{Der}}
\def\Map{\mathrm{Map}}
\def\Gp{\mathbf{Gp}}
\def\Mon{\mathbf{Mon}}
\def\mod{\hbox{mod}}
\def\be{\begin{equation}}
\def\ee{\end{equation}}
\def\tensor{\otimes}
\def\iso{\cong}
\def\Ho{\mathrm{Ho}\,}
\def\rin{\mathrm{in}}
\def\Fun{\mathrm{Fun}}
\def\id{\mathrm{id}}
\def\nt{\mathrm{nt}}

\def\la#1{\mathop{\longleftarrow}\limits^{#1}}
\def\ra#1{\mathop{\longrightarrow}\limits^{#1}}


\def\bS{\mathbf{S}}

\def\inj{\mathrm{in}}
\def\pr{\mathrm{pr}}
\def\div{\mathrm{div}}
\def\grad{\mathrm{grad}}
\def\curl{\mathrm{curl}}
\def\Sin{\mathrm{Sin}}

\def\SF{\mathcal{C}^\infty}
\def\VF{\mathcal{VF}^\infty}


\def\TT{\mathbb{T}}
\def\Tensor{\bigotimes}
\def\bDelta{\mathbf{\Delta}}
\def\bSet{\mathbf{Set}}
\def\bAb{\mathbf{Ab}}
\def\bTop{\mathbf{Top}}
\def\bC{\mathbf{C}}
\def\ob{\mathrm{ob}}
\def\bVS{\mathbf{VS}}

\def\cP{\mathcal{P}}
\def\cE{\mathcal{E}}


\begin{center}
{\bf 18.905: Problem Set VI}
\end{center}

Due November 30, 2016, in class. A couple more problems to follow.

Homework is an important part of this class. I hope you gain from the
struggle. Collaboration can be effective, but be sure that you
grapple with each problem on your own as well. If you do work with others,
you must indicate with whom on your solution sheet. Scores will be posted
on the Stellar website. Extra credit for calling attention to mistakes!

\bigskip
{\bf 25.} $\varnothing$.

\medskip
{\bf 26.} (Another $3\times3$ puzzle.) Suppose all the rows and columns in
the commutative diagram
\[
\xymatrix{
& 0 \ar[d] & 0 \ar[d] & 0 \ar[d] & \\
0 \ar[r] & A' \ar[d] \ar[r] & A \ar[d] \ar[r] & A'' \ar[d] \ar[r] & 0 \\
0 \ar[r] & B' \ar[d] \ar[r] & B \ar[d] \ar[r] & B'' \ar[d] \ar[r] & 0 \\
0 \ar[r] & C' \ar[d] \ar[r] & C \ar[d] \ar[r] & C'' \ar[d] \ar[r] & 0 \\ 
& 0 & 0 & 0 &
}\]
are exact. Construct from this a natural exact sequence 
\[
0\to A'\to A\oplus B'\to B\to C''\to0\,,
\]
It may be easiest to construct two short exact sequences and then splice them
together.

What is the dual statement? 

\medskip
{\bf 27. (a)} Let $0\ra{}M'\ra{i}M\ra{p}M''\ra{}0$ be a short exact sequence
of abelian groups, and let $C_*$ be a chain complex of free abelian groups.
Construct and prove exactness of the ``long exact coefficient sequence'' 
\[
\cdots\ra{}H_n(C_*\tensor M')\ra{i_*}H_n(C_*\tensor M)\ra{p_*}
H_n(C_*\tensor M'')\ra{\partial}H_{n-1}(C_*\tensor M')\ra{}\cdots
\]

{\bf(b)} For example, the short exact sequence 
$0\ra{}\ZZ/2\ra{}\ZZ/4\ra{}\ZZ/2\ra{}0$ induces a natural transformation
\[
\beta:H_n(X;\ZZ/2)\ra{}H_{n-1}(X;\ZZ/2)\,.
\]
(called the ``Bockstein operator'').
Show that $\beta^2=0:H_n(X;\ZZ/2)\ra{}H_{n-2}(X;\ZZ/2)$.

{\bf (c)} 
Compute this operator in the mod 2 homology of real projective $n$ space.


\medskip
{\bf 28.} 
Recall that $\alpha_p:\Delta^p\to\Delta^n$ is the affine map sending the vertex
$i$ to the vertex $i$ (for $0\leq i\leq p$), and 
$\omega_q:\Delta^q\to\Delta^n$ is the affine map sending the vertex
$j$ to the vertex $p+j$ (for $0\leq j\leq q$). The {\em Alexander-Whitney map}
\[
\alpha_{X,Y}:S_*(X\times Y)\to S_*(X)\tensor S_*(Y)
\]
sends a simplex $\sigma:\Delta^n\to X\times Y$, with components
$\sigma_1:\Delta^n\to X$ and $\sigma_2:\Delta^n\to Y$, to the sum
\[
\alpha(\sigma)=
\sum_{p+q=n}(\sigma_1\circ\alpha_p)\tensor(\sigma_2\circ\omega_q)\,.
\]

{\bf (a)} Check that this is a chain map. 

{\bf (b)} Check that it is ``associative'' and ``unital'':
\[
\xymatrix{
S_*(X\times Y\times Z) \ar[r]^{\alpha_{X\times Y,Z}} 
\ar[d]^{\alpha_{X,Y\times Z}} & 
S_*(X\times Y)\tensor S_*(Z) \ar[d]^{\alpha_{X,Y}\tensor1} \\
S_*(X)\tensor S_*(Y\times Z) \ar[r]^{1\tensor\alpha_{Y,Z}} &
S_*(X)\tensor S_*(Y)\tensor S_*(Z)
}\]
and
\[
\xymatrix{
& S_*(*\times X) \ar[dl]_c \ar[d]^{\alpha_{*,X}} & 
S_*(X\times *) \ar[dr]^c \ar[d]_{\alpha_{X,*}} \\
S_*(X) & S_*(*)\tensor S_*(X) \ar[l]_{\epsilon\cdot1} & 
S_*(X)\tensor S_*(*) \ar[r]^{1\cdot\epsilon} & S_*(X)
}\]
Here the map $c$ is induced from one of the projection isomorphisms
$*\times X\to X$ or $X\times*\to X$, 
\[
\epsilon\cdot1:S_*(*)\tensor S_*(X)\ra{\epsilon\tensor1}\ZZ\tensor S_*(X)
\ra{\iso}S_*(X)
\]
and $1\cdot\epsilon$ is similar.

{\bf(c)} Now observe that $\alpha$ is {\em not} ``commutative'': give an
example to show that 
\[
\xymatrix{
S_*(X\times Y) \ar[r]^{S_*(T)} \ar[d]^{\alpha_{X,Y}} & 
S_*(Y\times X) \ar[d]^{\alpha_{Y,X}} \\
S_*(X)\tensor S_*(Y) \ar[r]^{\tau} & S_*(Y)\tensor S_*(X)
}\]
cannot commute for either sign in $\tau(x\tensor y)=\pm y\tensor x$. 

{\bf(d)} But show that this diagram {\em is} naturally homotopy 
commutative, with 
\[
\tau(x\tensor y)=(-1)^{pq}y\tensor x\,,\quad x\in S_p(X)\,,\quad y\in S_q(Y)\,.
\]

\medskip
{\bf 29.} Construct a ``semi-relative cross product,'' 
natural in $X$ and the pair $(Y,B)$:
\[
\times:H^p(X;R)\tensor_R H^q(Y,B;R)\to H^{p+q}(X\times Y,X\times B;R)
\]
that agrees with the cross product we constructed in class if $B=\varnothing$ 
and that makes ($R$ coefficients understood)
\[
\xymatrix{
H^p(X)\tensor H^q(B) \ar[r]^\times \ar[d]^{1\tensor\partial} & 
H^{p+q}(X\times B) \ar[d]^\partial \\
H^p(X)\tensor H^{q+1}(Y,B) \ar[r]^\times & H^{p+q+1}(X\times Y,X\times B)
}
\]
commute, at least up to sign. Give conditions under which it is an isomorphism.

\medskip
{\bf 30.} Let $A\subseteq X$ and $B\subseteq Y$ be subsets. 
Construct a natural chain map
\[
S_*(X,A)\tensor S_*(Y,B)\to S_*(X\times Y,A\times Y\cup X\times B)
\]
that is a homology isomorphism if $A$ and $B$ are open. (Hint: Problem 
{\bf 26.}, or its proof, might be useful.) So 
there is a natural ``relative cross product'' map
\[
H_*(X,A;R)\tensor_R H_*(Y,B;R)\to H_*(X\times Y,A\times Y\cup X\times B;R)
\]
that is an isomorphism if $A$ and $B$ are open, $R$ is a PID, and either 
$H_*(X,A;R)$ or $H_*(Y,B;R)$ is free over $R$.

\medskip
{\bf 31. (a)} What is the $k$th Betti number of $(S^1)^n$?

{\bf(b)} Define an equivalence relation on $\RR^n$ by saying that two vectors
are equivalent if they differ by a vector with entries in $\ZZ$. Identify the
quotient space of $\RR^n$ by this equivalence relation with the product space 
$(S^1)^n$. Let $M$ be an $n\times n$ matrix with entries in $\ZZ$. 
It defines a linear map $\RR^n\to\RR^n$ in the usual way. Show that this
map descends to a self-map of $(S^1)^n$. Compute the effect of this map on
$H_n((S^1)^n)$. 






\end{document}


\medskip
{\bf 27. (a)} Let 
\[
\alpha_{X,Y}:S_*(X\times Y)\to S_*(X)\tensor S_*(Y)
\]
be any natural chain map covering the canonical isomorphism 
$H_0(X\times Y)\to H_0(X)\tensor H_0(Y)$. Show that the diagram
\[
\xymatrix{
S_*(X\times Y) \ar[r]^{S_*(T)} \ar[d]^{\alpha_{X,Y}} & 
S_*(Y\times X) \ar[d]^{\alpha_{Y,X}} \\
S_*(X)\tensor S_*(Y) \ar[r]^T & S_*(Y)\tensor S_*(X)
}\]
commutes up to natural chain homotopy. Here $T:X\times Y\to Y\times X$
sends $(x,y)$ to $(y,x)$, and $T:C_*\tensor D_*\to D_*\tensor C_*$
(for graded abelian groups $C_*$ and $D_*$) sends $x\tensor y$ to
$(-1)^{|x||y|}y\tensor x$. Hint: Acyclic models.

{\bf(b)} Explain why this proves that the cup product on $H^*(X;R)$ is 
graded commutative: 
\[
\alpha\cup\beta=(-1)^{|\alpha||\beta|}\beta\cup\alpha\,.
\]



{\bf 27. (a)} Show that for any choice of chain-level cross product,
(i.e. natrual map $\times:S_*(X)\tensor S_*(Y)\to S_*(X\times Y)$ covering 
the canonical map $H_0(X)\tensor H_0(Y)\to H_0(X\times Y)$) the diagrams 
\[
\xymatrix{
S_*(X)\tensor S_*(Y)\tensor S_*(X) \ar[r]^{\times\tensor1} 
\ar[d]^{1\tensor\times} & S_*(X\times Y)\tensor S_*(Z) \ar[d]^{\times} \\
S_*(X)\tensor S_*(Y\times Z) \ar[r]^\times & S_*(X\times Y\times Z)
}\]
and
\[
\xymatrix{
S_*(X)\tensor S_*(Y) \ar[r]^T \ar[d]^\times & 
S_*(Y)\tensor S_*(X) \ar[d]^\times \\
S_*(X\times Y) \ar[r]^{S_*(T)} & S_*(Y\times X)
}\]
commute. In the second diagram the bottom $T$ denotes the swap map 
$(x,y)\mapsto(y,x)$, while the top map is defined on decomposable tensors by
\[
x\tensor y\mapsto(-1)^{|x||y|}y\tensor x\,.
\]
Hint: use acyclic models!

{\bf (b)} Let $\epsilon:H_*(X)\to\ZZ$ be the augmentation. Check that 
\[
\xymatrix{
H_0(X)\tensor\ZZ \ar[d]^{c} & 
H_0(X)\tensor H_0(Y) 
\ar[l]_{1\tensor\epsilon} \ar[d]^{\times} \ar[r]^{\epsilon\tensor1} &
\ZZ\tensor H_0(Y) \ar[d]^{c} \\
H_0(X) & H_0(X\times Y) \ar[l]_{\pr_1} \ar[r]^{\pr_2} & H_0(Y)
}\]
where $c$ denotes the canonical map, twice.
