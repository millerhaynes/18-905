\section{Weak Hausdorff, Basepoints}\label{basepoints}
The ancients came up with a good definition of a topology ---
but $k$-spaces are better!
Sometimes, though, we can be greedy and ask for even more:
for instance, we can demand a Hausdorff condition.
This leads to a further refinement of $k$-spaces. But ``Hausdorff'' isn't
quite the appropriate condition in the context of $k$-spaces. Rather: 

\begin{definition}
    A space $X$ is {\em weak Hausdorff} if the image of every continuous 
map $K\to X$ from a compact Hausdorff space $K$ is closed.
\end{definition}
Another way to say this is that the map itself if closed. Clearly Hausdorff implies weak Hausdorff. Every singleton subset of a weak Hausdorff space is 
closed (since the one-point space is compact Hausdorff).
\begin{prop}
    Let $X$ be a $k$-space.
    \begin{enumerate}
	\item $X$ is weak Hausdorff if and only if 
$\Delta:X\to X\times^{k\Top}X$ is closed.
(In algebraic geometry such a condition is termed {\em separated}.)
	\item Let $R\subseteq X\times X$ be an equivalence relation. If $R$ is closed, then $X/R$ is weak Hausdorff.
    \end{enumerate}
\end{prop}
\begin{definition}
    A space is {\em compactly generated} if it is weak Hausdorff and a
$k$-space. The category of such spaces is written $\CG$.
\end{definition}
We have a pair of adjoint pairs:
\[
\Top\,\underset{k}{\overset{i}{\lra}}\,k\Top\,
\underset{j}{\overset{h}{\rla}}\,\CG
\]
so 
\[
ikX\to X \quad,\qquad hjZ\xrightarrow{\cong}Z \,,
\]
\[
Y\xrightarrow{\cong}kiY \quad,\qquad Y\to jhY\,.
\]
The functor $h$ is defined by dividing $Y$ by the intersection of all closed
equivalence relations on it. $k$-ification is a right adjoint, but 
weak Hausdorfification, being formation of a quotient, is a left adjoint.
The dangerous and annoying feature of $h$ is that if $Y$ is not weak Hausdorff 
then the map $Y\to jhY$ is not a bijection; the underlying point-set changes. 


\begin{lemma} If $Y$ is a $k$-space and $Z$ is weak Hausdorff, then $Z^Y$ is
weak Hausdorff.
\end{lemma}

\begin{corollary}
The category $\CG$ is Cartesian closed.
\end{corollary}

We will essentially always be working with either $k\Top$ or $\CG$; 
most of the time either one will do. I will just call the objects ``spaces,''
and (I'm sorry) write $\Top$ for the corresponding category.

Here's an example of how useful the formation of mapping spaces can be. 
We already know that a {\em homotopy} between maps $f,g:X\to Y$ is a map 
$h:I\times X\to Y$ such that the following diagram commutes.
\begin{equation*}
    \xymatrix{
X\ar[d]_{\inc_0}\ar[dr]^f & &\\
I\times X\ar[r]^h & Y\\
X\ar[u]^{\inc_1}\ar[ur]_g & &
    }
\end{equation*}
We write $f\sim g$ to indicate that $f$ and $g$ are homotopic. This is an
equivalence relation on $\Top(X,Y)$, and 
we define 
\[
[X,Y]=\Top(X,Y)/\sim\,.
\]

The maps $f$ and $g$ are points in the space $Y^X$, and the homotopy $h$ 
is the same thing as a path $\hat f:I\to Y^X$ from $f$ to $g$. So
\[
[X,Y]=\pi_0(Y^X)\,.
\]

To talk about the fundamental group, and higher homotopy groups, using
this strategy, we have to get basepoints into the picture.

\subsection{Basepoints}
A {\em pointed space} is a space $X$ together with a specified ``point'' 
in it, with default notation $*$. The point $\ast$ is called the 
{\em basepoint}. This leads some people refer to ``based spaces,'' but 
to my ear this makes it sound as if we are doing chemistry, or worse, 
and I prefer ``pointed.'' 
 
This gives a category $\Top_\ast$ where the morphisms respect the basepoints. 
This category is complete and cocomplete. For example
\[
(X,\ast)\times (Y,\ast)=(X\times Y,(\ast,\ast))
\]
(where of course I mean to be taking the product in $k\Top$ or $\CG$).
The coproduct is not disjoint union; which basepoint would you pick? So you
identify the two basepoints, to get the ``wedge'' (or in TeX, $\backslash$vee)
\[
X\vee Y=X\sqcup Y/\ast_X\sim \ast_Y\,.
\]

The one-point space $\ast$ is the terminal object in $\Top_*$, as in $\Top$, 
but it is also {\em initial} in $\Top_*$: 
there is exactly one map of pointed spaces from it to 
any $(X,\ast)$. We have a {\em pointed category}: 
A category with initial and terminal objects such that the 
unique map from the first to the second is an isomorphism.

Pointed categories are (almost) never Cartesian closed: for, as we saw,
in a Cartesian closed category $\cc$, $\cc(X,Y)=\cc(\ast,Y^X)$, and
in a pointed category this is a singleton.

But we still know what we would like to take as a ``mapping object'' 
in $\Top_*$:
Define $Y^X_*$ to be the subspace of $Y^X$ consisting of the pointed maps. 
As a replacement for Cartesian closure, 
let's ask: For fixed $X\in\Top_*$, 
does the functor $Y\mapsto Y^X_\ast$ have a left adjoint? 
This would be an analogue in $\Top_*$ of the functor $A\otimes-$ in $\Ab$. 
Compute:
\[
\xymatrix{
\Top(W,Y^X) \ar@{}[rr]|{=} && \{f:X\times W\to Y\} \\
\Top(W,Y_*^X) \ar@{}[rr]|{=} \ar[u]^\subseteq && 
\{f(*,w)=* \,\,\,\forall \,w\in W\} \ar[u]^\subseteq \\
\Top_*(W,Y_*^X) \ar@{}[rr]|{=} \ar[u]^\subseteq && 
{\left\{\begin{array}{ll}
f(*,w)=*&\forall\,w\in W\\
f(x,*)=*&\forall\,x\in X\end{array}\right\}}
\ar[u]^\subseteq 
}
\]
So the map $X\times W\to Y$ corresponding to $f:W\to Y_*^X$ sends the 
wedge $X\vee W\subseteq X\times W$ to the basepoint of $Y$, and hence
factors through the {\em smash product}
\[
X\wedge W=X\times W/X\vee W
\]
obtained by pinching the ``axes'' in the product to a point. We have an
adjoint pair 
\[
X\wedge-:\Top_*\rightleftarrows\Top_*:(-)^X_*\,.
\]

You are invited to check the various properties enjoyed by the smash product,
analogous to properties of the tensor product. So it's functorial in both
variables; the two-point pointed space serves as a unit; and it is associative
and commutative. Associativity is a blessing bestowed by assuming compact
generation; notice that in forming it we are mixing limits (the product)
with colimits (the quotient by the axes), and indeed the smash product
turns out {\em not} to be associative in the full category of spaces. 

Based spaces and all spaces are related by another adjoint pair, 
\[
(-)_+:\Top\rightleftarrows\Top_*:u
\]
where $u$ forgets the basepoint and $(-)_+$ adjoins a disjoint base point. 
The two-point pointed space is then $*_+$, but everyone writes it as $S^0$. 
Explain why this is a reasonable symbol for this pointed space. 

