\section{$I$-invariance of $\Bun_G$, and $G$-CW-complexes}
Let $G$ be a topological group. We need to show that the functor
$\Bun_G:\Top^{op}\to\Set$ is $I$-invariant, i.e., the projection
$X\times I\xrightarrow{\pr}X$ induces an isomorphism
$\Bun_G(X)\xrightarrow{\simeq}\Bun_G(X\times I)$.
Injectivity is easy: the composite $X\xrightarrow{\inc_0} X\times
I\xrightarrow{\pr}X$ gives you a splitting
$\Bun_G(X)\xrightarrow{\pr_\ast}\Bun_G(X\times I)\xrightarrow{\inc_0}\Bun_G(X)$
whose composite is the identity.

The rest of this lecture is devoted to proving surjectivity. We will prove this
when $X$ is a CW-complex (Husemoller does the general case; see \cite[\S
4.9]{husemoller}). We begin with a small digression.
%Consider two principal $G$-bundles $P\to X$ and $Q\to Y$, with a map $f:X\to
%Y$. We would like to understand maps $P\to f^\ast Q$ over $X$, i.e.,
%$G$-equivariant dotted maps that make the following diagram commute.
%\begin{equation*}
%    \xymatrix{
%	P\ar@{-->}[r]^g\ar[d] & Q\ar[d]\\
%	X\ar[r]_f & Y
%    }
%\end{equation*}
%Suppose I have $P\to X\times I$; then we get $in_0^\ast P\to X$. This has to
%be what you get when you ---. All we have to do is construct a map $P\to
%in_0^\ast P$ like in the diagram above.
\subsection{$G$-CW-complexes}
We would like to define CW-complexes with an action of the group $G$. The
na\"ive definition (of a space with an action of the group $G$) will not be
sufficient; rather, we will require that each cell have an action of $G$.

In other words, we will build $G$-CW-complexes out of ``$G$-cells''. This is
supposed to be something of the form $D^n\times H\backslash G$, where $H$ is a
closed subgroup of $G$. Here, the space $H\backslash G$ is the orbit space,
viewed as a right $G$-space. The boundary of the $G$-cell $D^n\times
H\backslash G$ is just $\partial D^n\times H\backslash G$. More precisely:
\begin{definition}
    A $G$-CW-complex is a (right) $G$-space $X$ with a filtration
    $0=X_{-1}\subseteq X_0\subseteq \cdots\subseteq X$ such that for all $n$,
    there exists a pushout square:
    \begin{equation*}
	\xymatrix{
	    \coprod\partial D^n_\alpha\times H_\alpha\backslash G\ar[r]\ar[d] &
	    \coprod D^n_\alpha\times H_\alpha\backslash G\ar[d]\\
	    X_{n-1}\ar[r] & X_n,
	    }
    \end{equation*}
    and $X$ has the direct limit topology.
\end{definition}
Notice that a CW-complex is a $G$-CW-complex for the trivial group $G$.
\begin{theorem}
    If $G$ is a compact Lie group and $M$ a compact smooth $G$-manifold, then
    $M$ admits a $G$-CW-structure.
\end{theorem}
This is the analogue of the classical result that a compact smooth manifold is
homotopy equivalent to a CW-complex, but it is much harder to prove the
equivariant statement.

Note that if $G$ acts principally (Definition \ref{principaldefn}) on $P$, then
every $G$-CW-structure on $P$ is ``free'', i.e., $H_\alpha = 0$.
\begin{enumerate}
    \item If $X$ is a $G$-CW-complex, then $X/G$ inherits a CW-structure whose
	$n$-skeleton is given by $(X/G)_n = X_n/G$.
    \item If $P\to X$ is a principal $G$-bundle, then a CW-structure on $X$
	lifts to a $G$-CW-structure on $P$.
\end{enumerate}
\subsection{Proof of $I$-invariance}
Recall that our goal is to prove that every $G$-bundle over $X\times I$ is
pulled back from some vector bundle over $X$.

As a baby case of Theorem \ref{Iinvariance} we will prove that if
$X$ is contractible, then any principal $G$-bundle over $X$ is trivial, i.e.,
$P\simeq X\times G$ as $G$-bundles.

Let us first prove the following: if $P\downarrow X$ has a section, then it's
trivial. Indeed, suppose we have a section $s:X\to P$. Since $P$ has an action
of the group on it, we may extend this to a map $X\times G\to P$ by sending
$(x,g)\mapsto gs(x)$. As this is a map of $G$-bundles over $X$, it is an
isomorphism by Theorem \ref{morphismiso}, as desired.

To prove the statement about triviality of any principal $G$-bundle over a
contractible space, it therefore suffices to construct a section for any
principal $G$-bundle. Consider the constant map $X\to P$. Then the following
diagram commutes up to homotopy, and hence (by Exercise
\ref{sectionuptohomotopy}(1)) there is an \emph{actual} section of $P\to X$, as
desired.
\begin{equation*}
    \xymatrix{
	& P\ar[d]\\
	X\ar[ur]^{\mathrm{const}}\ar[r] & X
    }
\end{equation*}
For the general case, we will assume $X$ is a CW-complex. For notational
convenience, let us write $Y=X\times I$. We will use descending induction to
construct the desired principal $G$-bundle over $X$.

To do this, we will filter $Y$ by subcomplexes. Let $Y_0 = X\times 0$; in
general, we define
$$Y_n = X\times 0\cup X_{n-1}\times I.$$
It follows that we may construct $Y_n$ out of $Y_{n-1}$ via a pushout:
\begin{equation*}
    \xymatrix{
	\coprod_{\alpha\in\Sigma_{n-1}}(\partial D^{n-1}\times I \cup
	D^{n-1}_\alpha\times 0) \ar[r]\ar[d]_{\coprod_{\alpha\in\Sigma_{n-1}}
	f_\alpha\times 1_I\cup\phi_\alpha\times 0} &
	\coprod_\alpha(D^{n-1}_\alpha\times I) \ar[d]\\
	Y_{n-1}\ar[r] & Y_n,
    }
\end{equation*}
where the maps $f_\alpha$ and $\phi_\alpha$ are defined as:
\begin{equation*}
    \xymatrix{
	\partial D^{n-1}_\alpha\ar[r]^{f_\alpha}\ar[d] & X_{n-2}\ar[d]\\
	D^{n-1}_\alpha\ar[r]_{\phi_\alpha} & X_{n-1}
    }
\end{equation*}
In other words, the $f_\alpha$ are the attaching maps and the $\phi_\alpha$ are
the characteristic maps.

Consider a principal $G$-bundle $P\xrightarrow{p}Y = X\times I$. Define $P_n =
p^{-1}(Y_n)$; then we can build $P_n$ from $P_{n-1}$ in a similar way:
\begin{equation*}
    \xymatrix{
	\coprod_\alpha(\partial D^{n-1}_\alpha\times I\cup D^{n-1}_\alpha\times
	0)\times G\ar[r]\ar[d] & \coprod_\alpha (D^{n-1}_\alpha\times I)\times
	G\ar[d]\\
	P_{n-1}\ar[r] & P_n
    }
\end{equation*}
%This makes sense since $D^n_\alpha\times I$ is a contractible space, and same
%thing for the other factor.
Note that this isn't \emph{quite} a $G$-CW-structure. Recall that we are
attempting to fill in a dotted map:
\begin{equation*}
    \xymatrix{
	P\ar@{-->}[r]\ar[d] & P_0\ar[d]\\
	Y\ar[r]_{\pr} & Y_0 = X
    }
\end{equation*}
\todo{finish this...}
I'm constructing this inductively-- we have $P_{n-1}\to P_0$. So I want to define $\coprod_\alpha(D^{n-1}_\alpha\times I)\times G\to P_0$ that's equivariant. That's the same thing as a map $\coprod_\alpha (D^{n-1}_\alpha\times I)\to P_0$ that's compatible with the map from $\coprod(\partial D^{n-1}_\alpha\times I\cup D^{n-1}_\alpha\times 0)$. Namely, I want to fill in:
\begin{equation}\label{finally}
    \xymatrix{
	\coprod_\alpha (\partial D^{n-1}_\alpha\times I\cup D^{n-1}_\alpha\times 0)\ar[r]\ar[d] & \coprod_\alpha(D^{n-1}_\alpha\times I)\ar@{-->}[dddr]\ar[d] & \\
	\coprod_\alpha(\partial D^{n-1}_\alpha\times I\cup D^{n-1}_\alpha\times 0)\times G\ar[r]\ar[d] & \coprod_\alpha (D^{n-1}_\alpha\times I)\times G\ar@{-->}[ddr]\ar[d] & \\
	P_{n-1}\ar[drr]_{\mathrm{induction}}\ar[r] & P_n\ar[dr] & \\
	& & P_0\ar[d]\\
	& & X
    }
\end{equation}
Now, I know that $(D^{n-1}\times I,\partial D^{n-1}\times I\cup D^{n-1}\times 0)\simeq (D^{n-1}\times I,D^{n-1}\times 0)$. So what I have is:
\begin{equation*}
    \xymatrix{
	D^{n-1}\times 0\ar[d]\ar[r]^{\mathrm{induction}} & P_0\ar[d]\\
	D^{n-1}\times I\ar[r]_{\phi\circ pr}\ar@{-->}[ur] & X
    }
\end{equation*}
So the dotted map exists, since $P_0\to X$ is a fibration!

OK, so note that I haven't checked that the outer diagram in Equation \ref{finally} commutes, because otherwise we wouldn't get $P_n\to P_0$.
\begin{exercise}
    Check my question above.
    
    Turns out this is easy, because you have a factorization:
\begin{equation*}
    \xymatrix{
	D^{n-1}\times 0\ar[d]\ar[r] & P_{n-1}\ar[r]^{\mathrm{induction}} & P_0\ar[d]\\
	D^{n-1}\times I\ar[rr]_{\phi\circ pr}\ar@{-->}[urr] & & X
    }
\end{equation*}
\end{exercise}
Oh my god, look what time it is! Oh well, at least we got the proof done.
