\section{Cellular approximation, cellular homology, obstruction theory}
%pset 3 is up in part. Today I'm going to tell you how to think about CW-complexes and work with them, and I won't give a lot of details. Some good sources are lecture notes by Davis-Kirk, and a beautiful book by Glen Bredon, as well as Hatcher.
In previous sections, we saw that homotopy groups play well with (maps between) CW-complexes.
Here, we will study maps between CW-complexes themselves, and prove that they are, in some sense, ``cellular'' themselves.
\subsection{Cellular approximation}
\begin{definition}\label{cellularmaps}
    Let $X$ and $Y$ be CW-complexes, and let $A\subseteq X$ be a subcomplex.
    Suppose $f:X\to Y$ is a continuous map.
    We say that $f|_{A}$ is skeletal\footnote{Some would say cellular.} if $f(\Sigma_n)\subseteq Y_n$. 
\end{definition}
Note that a skeletal map might not take cells in $A$ to cells in $Y$, but it takes $n$-skeleta to $n$-skeleta.
\begin{theorem}[Cellular approximation]\label{cellularapprox}
    In the setup of Definition \ref{cellularmaps}, the map $f$ is homotopic to some other
    continuous map $f^\prime:X\to Y$, relative to $A$, such that $f^\prime$ is skeletal on all of $X$.
\end{theorem}
To prove this, we need the following lemma.
\begin{lemma}[Key lemma]
    Any map $(D^n,S^{n-1})\to (Y,Y_{n-1})$ factors as:
    \begin{equation*}
	\xymatrix{
	(D^n,S^{n-1})\ar[r]\ar@{-->}[dr] & (Y,Y_{n-1})\\
	& (Y_n,Y_{n-1})\ar[u]
	}
    \end{equation*}
\end{lemma}
\begin{proof}[``Proof.'']
    Since $D^n$ is compact, we know that $f(D^n)$ must lie in some finite subcomplex $K$ of $Y$.
    The map $D^n\to K$ might hit some top-dimensional cell $e^m\subseteq K$, which does not have anything attached to it;
    hence, we can homotope this map to miss a point, so that it contracts onto a lower-dimensional cell.
    Iterating this process gives the desired result.
\end{proof}
Using this lemma, we can conclude the cellular approximation theorem.
\begin{proof}[``Proof'' of Theorem \ref{cellularapprox}]
    We will construct the homotopy $f\simeq f^\prime$ one cell at a time.
    Note that we can replace the space $A$ by the subspace to which we have extended the homotopy.
    
    Consider a single cell attachment $A\to A\cup D^m$; then, we have
    \begin{equation*}
	\xymatrix{
	A\ar[d]_{\text{skeletal}}\ar[r] & A\cup D^m\ar[dl]^{\text{ may not be skeletal}}\\
	Y &
	}
    \end{equation*}
    Using the ``compression lemma'' from above, the rightmost map factors (up to homotopy) as the composite
    $A\cup D^m\to Y_m \to Y$.
    Unfortunately, we have not extended this map to the whole of $X$, although we could do this
    if we knew that the inclusion of a subcomplex is a cofibration.
    But this is true: there is a cofibration $S^{n-1}\to D^n$, and so any pushout of these maps is a cofibration!
    This allows us to extend; we now win by iterating this procedure.
\end{proof}
As a corollary, we find:
\begin{exercise}
    The pair $(X,X_n)$ is $n$-connected.
\end{exercise}
\subsection{Cellular homology}
Let $(X,A)$ be a relative CW-complex with $A\subseteq X_{n-1}\subseteq X_n\subseteq\cdots\subseteq X$.
In the previous part \todo{provide a link!} that $H_\ast(X_n,X_{n-1})\simeq \widetilde{H}_\ast(X_n/X_{n-1})$.
More generally, if $B\to Y$ is a cofibration, there is an isomorphism (see \cite[p. 433]{bredon}):
$$H_\ast(Y,B)\simeq\widetilde{H}_\ast(Y/B).$$
Since $X_n/X_{n-1} = \bigvee_{\alpha\in \Sigma_n}S^n_\alpha$, we find that
$$H_\ast(X_n,X_{n-1})\simeq \Z[\Sigma_n] = C_n.$$
The composite $S^{n-1}\to X_{n-1}\to X_{n-1}/X_{n-2}$ is called a relative attaching map.

There is a boundary map $d:C_n\to C_{n-1}$, defined by
$$d:C_n = H_n(X_n,X_{n-1})\xrightarrow{\partial}H_{n-1}(X_{n-1})\to H_{n-1}(X_{n-1},X_{n-2}) = C_{n-1}.$$
\begin{exercise}
    Check that $d^2 = 0$.
\end{exercise}
Using the resulting chain complex, denoted $C_\ast(X,A)$, one can prove that there is an isomorphism
$$H_n(X,A)\simeq H_n(C_\ast(X,A)).$$
(In the previous part\todo{provide a link!}, we proved this for CW-pairs, but not for relative CW-complexes.)
The incredibly useful cellular approximation theorem therefore tells us that the effect of maps on homology can be computed.

Of course, the same story runs for cohomology: one gets a chain complex which, in dimension $n$, is given by
$$C^n(X,A;\pi) = \Hom(C_n(X,A),\pi) = \Map(\Sigma_n,\pi),$$
where $\pi$ is any abelian group.
\subsection{Obstruction theory}
Using the tools developed above, we can attempt to answer some concrete, and useful, questions.
\begin{question}
    Let $f:A \to Y$ be a map from a space $A$ to $Y$.
    Suppose $(X,A)$ is a relative CW-complex.
    When can we find an extension in the diagram below?
    \begin{equation*}
	\xymatrix{
	    X\ar@{-->}[dr] & \\
	    A\ar@{^(->}[u]\ar[r]_f & Y
	    }
    \end{equation*}
\end{question}
The lower level obstructions can be worked out easily:
\begin{equation*}
    \xymatrix{
	A\ar[d]\ar@{^(->}[r] & X_0\ar@{-->}[dl]\ar@{^(->}[r] & X_1\ar@{-->}[dll]\\
	\emptyset\neq Y & &
    }
\end{equation*}
Thus, for instance, if two points in $X_0$ are connected in $X_1$, we only have to check that they are also connected in $Y$.

For $n\geq 2$, we can form the diagram:
\begin{equation*}
    \xymatrix{
	\coprod_{\alpha\in\Sigma_n}S^{n-1}_\alpha\ar[r]^f\ar@{^(->}[d] & X_{n-1}\ar[r]^g\ar[d] & Y\\
	\coprod_{\Sigma_n}D^n_\alpha\ar[r] & X_n\ar@{-->}[ur] & 
    }
\end{equation*}
The desired extension exists if the composite $S^{n-1}_\alpha\xar{f_\alpha} X_{n-1}\to Y$ is nullhomotopic.

Clearly, $g\circ f_\alpha\in [S^{n-1},Y]$.
To simplify the discussion, let us assume that $Y$ is simple;
then, Exercise \ref{simplequotient} says that $[S^{n-1},Y] = \pi_{n-1}(Y)$.
This procedure begets a map $\Sigma_n\xrightarrow{\theta}\pi_{n-1}(Y)$, which is a $n$-cochain, i.e., an element of
$C^n(X,A;\pi_{n-1}(Y))$.
It is clear that $\theta = 0$ if and only if the map $g$ extends to $X_n\to Y$.
\begin{prop}
    $\theta$ is a cocycle in $C^n(X,A;\pi_{n-1}(Y))$, called the ``obstruction cocycle''.
\end{prop}
\begin{proof}
    $\theta$ gives a map $H_n(X_n,X_{n-1})\to \pi_{n-1}(Y)$.
    We would like to show that the composite
    $$H_{n+1}(X_{n+1},X_n)\xrightarrow{\partial}H_n(X_n)\to H_n(X_n,X_{n-1})\xrightarrow{\theta}\pi_{n-1}(Y)$$
    is trivial.
    %OK, we know that $\pi_n(X_n,X_{n-1})\to H_n(X_n,X_{n-1})$ is surjective. This relative homotopy group holds the characteristic maps, doesn't it? So picking a representative back there is no problem; now I have the lexseq of a pair, which gives:
    We have the long exact sequence in homotopy of a pair (see Equation \eqref{lexseqhomotopy}):
    \begin{equation*}
	\xymatrix{
	    \pi_{n+1}(X_{n+1},X_n)\ar[d]\ar@{->>}[r] & H_{n+1}(X_{n+1},X_n)\ar[d]^\partial\\
	    \pi_n(X_n)\ar[r]\ar[d] & H_n(X_n)\ar[d]\\
	    \pi_n(X_n,X_{n-1})\ar@{->>}[r] \ar[d]_\partial & H_n(X_n,X_{n-1})\ar[d]^\theta\\
	    \pi_{n-1}(X_{n-1})\ar[r]_{g_\ast} & \pi_{n-1}(Y)
	    }
    \end{equation*}
    This diagram commutes, so $\theta$ is indeed a cocycle.
\end{proof}
Our discussion above allows us to conclude:
\begin{theorem}
    Let $(X,A)$ be a relative CW-complex and $Y$ a simple space.
    Let $g:X_{n-1}\to Y$ be a map from the $(n-1)$-skeleton of $X$.
    Then $g|_{X_{n-2}}$ extends to $X_n$ if and only if $[\theta(g)]\in H^n(X,A;\pi_{n-1}(Y))$ is zero.
\end{theorem}
\begin{corollary}
    If $H^n(X,A;\pi_{n-1}(Y)) = 0$ for all $n>2$, then any map $A\to Y$ extends to a map $X\to Y$ (up to homotopy\footnote{In fact, this condition is unnecessary, since the inclusion
    of a subcomplex is a cofibration.});
    in other words, there is a dotted lift in the following diagram:
    \begin{equation*}
	\xymatrix{
	A\ar[r]\ar[d] & Y\\
	X\ar@{-->}[ur] & 
	}
    \end{equation*}
\end{corollary}
For instance, every map $A\to Y$ factors through the cone if $H^n(CA,A;\pi_{n-1}(Y)) \simeq \widetilde{H}^{n-1}(A;\pi_{n-1}(Y)) = 0$.
%Of course, one still needs to know the homotopy groups of $Y$ are, but often, you can get away even with partial information.
%\begin{remark}
%    Fix a prime $p$. Later we'll see (I hope) that $Y$ is simply connected (maybe even simple?), and suppose that $\widetilde{H}_\ast(Y;\Z_{(p)}) = 0$ (i.e., no $\Z$-summands and no $\Z/p^k$-summands). Then $\pi_\ast(Y)\otimes\Z_{(p)} = 0$ too. So if $H_\ast(A;\Z/\ell\Z) = 0$ for $\ell\neq p$, then $[A,Y]\simeq \ast$.
%\end{remark}
