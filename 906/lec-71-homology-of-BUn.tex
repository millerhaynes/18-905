\section{$H^\ast(BU(n))$, splitting principle}\label{homology-bun}
%I wanted to share some ideas about the question on the first nonzero homology group of an Eilenberg-MacLane space.
%One idea was to use this extension $0\to \Z\xar{p^k}\Z\to \Z/p^k\to 0$.
%This gives rise to a fiber sequence $K(\Z,n)\to K(\Z,n)\to K(\Z/p^k,n)\to 0$.
%You could maybe use the Serre spectral sequence here?
%Another idea that came up was that this is embedded in a long sequence of fibrations, so that you have $K(\Z,n)\to K(\Z,n)\to K(\Z/p^k, n) \to K(\Z,n+1)\to\cdots$.
%
%I'm really excited today -- it's more characteristic classes!
%Last week, I gave you Grothendieck's construction of Chern classes:
%if you have a complex $n$-plane bundle $\xi\downarrow X$, I described $\PP(\xi)$, which has a canonical tautologous line bundle $\lambda_\xi\downarrow \PP(\xi)$.
%This has an Euler class $e=e(\lambda_\xi)\in H^2(\PP(\xi))$, and this satisfies the following unique relation:
%$$
%e^n+c_1e^{n-1} + \cdots + c_n = 0
%$$
%Then $c_k\in H^{2k}(X)$ are called the Chern classes.
%
%This doesn't give you much insight as to how to compute them.
Theorem \ref{chern-classes} claimed that the Chern classes, which we
constructed in the previous section, generate the cohomology of $BU$ as a
polynomial algebra. Our goal in this section is to prove this result.
\subsection{The cohomology of $BU(n)$}
Recall that $BU(n)$ supports the universal principal $U(n)$-bundle $EU(n)\to
BU(n)$. Given any left action of $U(n)$ on some space, we can form the
associated fiber bundle. For instance, the associated bundle of the
$U(n)$-action on $\cC^n$ yields the universal line bundle $\xi_n$.

Likewise, the associated bundle of the action of $U(n)$ on $S^{2n-1}\subseteq
\cc^n$ is the unit sphere bundle $S(\xi_n)$, the unit sphere bundle. By
construction, the fiber of the map $EU(n)\times_{U(n)}S^{2n-1}\to BU(n)$ is
$S^{2n-1}$. Since
$$S^{2n-1} = U(n)/(1\times U(n-1)),$$
we can write
$$EU(n)\times_{U(n)}S^{2n-1} \simeq EU(n)\times_{U(n)} (U(n)/U(n-1)) \simeq
EU(n)/U(n-1) = BU(n-1).$$
In other words, $BU(n-1)$ is the unit sphere bundle of the tautologous line
bundle over $BU(n)$. This begets a fiber bundle:
$$S^{2n-1}\to BU(n-1)\to BU(n),$$
which provides an inductive tool (via the Serre spectral sequence) for
computing the homology of $BU(n)$. In \S \ref{gysin-sequence}, we observed that
the Serre spectral sequence for a spherical fibration was completely described
bythe Gysin sequence.

Recall that if $B$ is oriented and $S^{2n-1}\to E\xar{\pi} B$ is a spherical
bundle over $B$, then the Gysin sequence was a long exact sequence
$$
\cdots\to H^{q-1}(E) \xar{\pi_\ast} H^{q-2n}(B) \xar{e\cdot} H^q(B) \xar{\pi^\ast} H^q(E) \xar{\pi_\ast} \cdots
$$
Let us assume that the cohomology ring of $E$ is polynomial and concentrated in
even dimensions. For the base case of the induction, both these assumptions are
satisfied (since $BU(0) = \ast$ and $BU(1) = \CP^\infty$).

These assumptions imply that if $q$ is even, then the map $\pi_\ast$ is zero.
In particular, multiplication by $e|_{H^\mathrm{even}(B)}$ (which we will also
denote by $e$) is injective, i.e., $e$ is a nonzero divisor.  Similarly, if $q$
is odd, then $e\cdot H^{q-2n}(B) = H^q(B)$. But if $q=1$, then $H^{q-2n}(B) =
0$; by induction on $q$, we find that $H^\mathrm{odd}(B) = 0$. Therefore, if
$q$ is even, then $H^{q-2n+1}(B) = 0$. This implies that there is a short exact
sequence
\begin{equation}\label{inductive-step-cohomology}
    0\to H^\ast(B) \xar{e\cdot} H^\ast(B) \to H^\ast(E) \to 0.
\end{equation}
In particular, the cohomology of $E$ is the cohomology of $B$ quotiented by the
ideal generated by the nonzero divisor $e$.

For instance, when $n=1$, then $B=\CP^\infty$ and $E\simeq \ast$. We have the
canonical generator $e\in H^2(\CP^\infty)$; these deductions tell us the
well-known fact that $H^\ast(\CP^\infty) \simeq \Z[e]$.

Consider the surjection $H^\ast(B) \xar{\pi^\ast} H^\ast(E)$. Since $H^\ast(E)$
is polynomial, we can lift the generators of $H^\ast(E)$ to elements of
$H^\ast(B)$. This begets a splitting $s:H^\ast(E) \to H^\ast(B)$. The existence
of the Euler class $e\in H^\ast(B)$ therefore gives a map $H^\ast(E)[e]
\xar{\overline{s}} H^\ast(B)$. We claim that this map is an isomorphism.

This is a standard algebraic argument. Filter both sides by powers of $e$,
i.e., take the $e$-adic filtration on $H^\ast(E)[e]$ and $H^\ast(B)$. Clearly,
the associated graded of $H^\ast(E)[e]$ just consists of an infinite direct sum
of the cohomology of $E$. The associated graded of $H^\ast(B)$ is the same,
thanks to the short exact sequence \eqref{inductive-step-cohomology}. Thus the
induced map on the associated graded $\gr^\ast(\overline{s})$ is an
isomorphism. In this particular case (but not in general), we can conclude that
$\overline{s}$ is an isomorphism: in any single dimension, the filtration is
finite. Thus, using the five lemma over and over again, we see that the map
$\overline{s}$ an isomorphism on each filtered piece. This implies that
$\overline{s}$ itself is an isomorphism, as desired.
	
This argument proves that
$$H^\ast(BU(n-1)) = \Z[c_1,\cdots,c_{n-1}].$$
In particular, there is a map $\pi^\ast:H^\ast(BU(n)) \to H^\ast(BU(n-1))$
which an isomorphism in dimensions at most $2n$. Thus, the generators $c_i$
have \emph{unique} lifts to $H^\ast(BU(n))$. We therefore get:
\begin{theorem}
    There exist classes $c_i\in H^{2i}(BU(n))$ for $1\leq i\leq n$ such that:
    \begin{itemize}
	\item the canonical map $H^\ast(BU(n)) \xar{\pi_\ast} H^\ast(BU(n-1))$
	    sends
	    $$
	    c_i \mapsto \begin{cases}
		c_i & i<n\\
		0 & i=n,\text{ and }
	    \end{cases}
	    $$
	\item $c_n := (-1)^n e\in H^{2n}(BU(n))$.
    \end{itemize}
    Moreover,
    $$\boxed{H^\ast(BU(n)) \simeq \Z[c_1,\cdots,c_n]}.$$
\end{theorem}
\subsection{The splitting principle}
\begin{theorem}\label{splitting-principle}
    Let $\xi\downarrow X$ be an $n$-plane bundle. Then there exists a space
    $\Fl(\xi) \xar{\pi} X$ such that:
    \begin{enumerate}
	\item $\pi^\ast \xi = \lambda_1\oplus\cdots\lambda_n$, where the
	    $\lambda_i$ are line bundles on $Y$, and
	\item the map $\pi^\ast: H^\ast(X) \to H^\ast(\Fl(\xi))$ is monic.
    \end{enumerate}
\end{theorem}
\begin{proof}
    We have already (somewhat) studied this space. Recall that there is a
    vector bundle $\pi:\PP(\xi)\to X$ such that
    $$H^\ast(\PP(\xi)) = H^\ast(X)\langle 1,e,\cdots,e^{n-1}\rangle.$$
    Moreover, in \S \ref{grothendieck-chern}, we proved that there is a complex
    line bundle over $\PP(\xi)$ which is a subbundle of $\pi^\ast\xi$. In other
    words, $\pi^\ast\xi$ splits as a sum of a line bundle and some other bundle
    (by Corollary \ref{split}). Iterating this construction proves the
    existence of $\Fl(\xi)$.
\end{proof}
This proof does not give much insight into the structure of $\Fl(\xi)$.
Remember that the \emph{frame bundle} $\Fr(\xi)$ of $\xi$: an element of
$\Fr(\xi)$ is a linear, inner-product preserving map $\cC^n\to E(\xi)$.
This satisfies various properties; for instance:
$$E(\xi) = \Fr(\xi)\times_{U(n)}\cC^n.$$
Moreover,
$$\PP(\xi) = \Fr(\xi)\times_{U(n)} U(n)/(1\times U(n-1)).$$

The \emph{flag bundle} $\Fl(\xi)$ is defined to be
$$\Fl(\xi) = \Fr(\xi)\times_{U(n)} U(n)/(U(1)\times\cdots\times U(1)).$$
The product $U(1)\times\cdots\times U(1)$ is usually denoted $T^n$, since it is
the maximal torus in $U(n)$. For the universal bundle $\xi_n\downarrow BU(n)$,
the frame bundle is exactly $EU(n)$; therefore, $\Fl(\xi_n)$ is just the bundle
given by $BT^n\to BU(n)$. By construction, the fiber of this bundle is
$U(n)/T^n$. In particular, there is a monomorphism $H^\ast(BU(n))
\hookrightarrow H^\ast(BT^n)$. The cohomology of $BT^n$ is extremely simple ---
it is the cohomology of a product of $\CP^\infty$'s, so
$$H^\ast(BT^n) \simeq \Z[t_1,\cdots,t_n],$$
where $|t_k| = 2$. The $t_i$ are the Euler classes of $\pi_i^\ast\lambda_i$,
under the projection map $\pi_i:BT^n\to\CP^\infty$.
