\section{Cartesian closure and compactly generated spaces}

A lot of homotopy theory is about loop spaces and mapping spaces. 
The compact-open topology is available to us -- and we'll recall it later.
But it suffers from some defects. To clarify how a mapping object
should behave in an ideal world, I want to make another category-theoretical
digression

\begin{definition} Let $\cc$ be a category with finite products. 
It is {\em Cartesian closed} if for any object $X$ in $\cc$, the functor
$X\times-$ has a right adjoint. 
\end{definition}

\begin{example} If $\cc=\Set$, 
\[
\Set(W\times X,Y)=\Set(X,\Set(W,Y))\,,
\]
so the right adjoint is the co-representable functor $\Set(W,-)$. 
\end{example}

So mapping objects are ideally given as the right adjoint to Cartesian
product.

Many otherwise well-behaved categories are not Cartesian-closed. 
The category of abelian groups is an example. There is indeed a functor 
$Y\mapsto\Hom(W,Y)$, for any abelian group $W$; but it is the right adjoint
not of $W\times-$ but rather of $W\otimes-$. 

As a general thing, if a functor has a right adjoint then that right
adjoint is well-defined up to canonical natural isomorphism; so we will
always speak of {\em the} right adjoint. We'll write the right adjoint 
to $X\times-$ using exponential notation,
\[
Y\mapsto Y^W\,,
\]
so that there is a natural bijection
\[
\cc(W\times X,Y)=\cc(X,Y^W)\,.
\]

The category of topological spaces is supposed to behave more like $\Set$
than like $\Ab$, but it turns out that $\Top$ is not Cartesian closed either.
Small modifications of it are, however. We describe them below.

To better understand the consequence of Cartesian closure, it's convenient
to restate the definition of adjointness. So suppose that 
\[
F:\cc\rightleftarrows \cd:G
\]
is an adjoint pair of functors, so that
\[
\cd(FX,Y)=\cc(X,GY)
\]

Take $Y=FX$; the identity $1_{FX}$ corresponds to a map $\eta_X:X\to GFX$,
called the {\em unit} of the adjunction. 

Take $X=GY$; the identity $1_{GY}$ corresponds to a map $\epsilon_Y:FGY\to Y$,
called the {\em counit} of the adjunction.

Both these maps are natural transformations of endofunctors of $\cc$. 

\begin{example} Take for example the free group adjunction
\[
F:\Set\rightleftarrows\Gp:u\,.
\]
In this case, $\eta:X\to uFX$ sends a set to itself inside the free group
it generates; and $\epsilon:FuY\to Y$ sends a word in the free group on
the set underlying $Y$ to the product of those elements in $Y$. 
\end{example}

\begin{lemma} The unit and counit of an adjunction make the following
diagrams commute.
\[
\xymatrix{
GFGY \ar[r]^{G\epsilon_Y} & GY & & FGFX \ar[r]^{\epsilon_{FX}} & FX \\
GY \ar[u]^{\eta_{GY}} \ar[ur]_= & & & FX \ar[u]^{F\eta_X} \ar[ur]_=
}
\]
Conversely, equipping functors $F$ and $G$ with natural 
transformations $\eta$ and $\epsilon$ making these diagrams commute 
establishs $(F,G)$ as an adjoint pair.
\end{lemma}

In the Cartesian setting, then, we have natural transformations
\[
\eta_X:X\to(W\times X)^W\,,\quad \epsilon_Y:W\times Y^W\to Y\,.
\]
In case $\cc=\Set$, these are given by
\[
x\mapsto(w\mapsto(w,x))\,,\quad(w,f)\mapsto f(w)\,,
\]
that is, including a slice and evaluation. You should check that the 
two triangles commute in this case! 

Here are some direct consequences of Cartesian closure. Note: the assumption
that finite products exist in $\cc$ includes the case of the empty product,
which is a terminal object $*$.

\begin{prop}
Let $\cc$ be Cartesian closed. \\
(1) $(X,Z)\mapsto Z^X$ is a functor $\cc^{op}\times\cc\to\cc$.\\
(2) $\cc(X,Y)=\cc(*,Y^X)$.\\
(3) $X\times-$ commutes with all colimits.
\end{prop}

\subsection{CGHW spaces}\label{CGWHspaces}

The last Proposition shows that $\Top$ is not Cartesian closed:
A standard example from general topology shows that if $Y\to Z$ is a
quotient map, the induced map $X\times Y\to X\times Z$ may fail to be
a quotient map. But any quotient maps are precisely coequalizers in $\Top$. 

Henry Whitehead showed that crossing with a compact Hausdorff space
{\em does} preserve quotient maps. This will often suffice, but often not, 
and the convenience of working in a Cartesian closed category is compelling. 

Inspired by Whitehead's theorem, we agree to accept only properties of a space
that can be observed by mapping compact Hausdorff spaces into it. 

\begin{definition}
    Let $X$ be a space.
    A subspace $F\subseteq X$ is said to be \emph{compactly closed} if,
    for any map $k:K\to X$ from a compact Hausdorff space $K$, 
    the preimage $k^{-1}(F)\subseteq K$ is closed.
\end{definition}
It is clear that any closed subset is compactly closed, but
there might be compactly closed sets which are not closed in the topology 
on $X$. This motivates the definition of a $k$-space:
\begin{definition}
    A topological space $X$ is said to be a \emph{$k$-space}
    if every compactly closed set is closed.
\end{definition}
The $k$ comes either from ``kompact'' and/or the general topologist 
John Kelley.

A more categorical characterization of this property is: 
$X$ is a $k$-space if and only if
a map $X\to Y$ is continuous precisely when for every compact Hausdorff space $K$ and map $k:K\to X$ the composite $K\to X\to Y$ is continuous.
For instance, compact Hausdorff spaces are $k$-spaces. First countable (so metric spaces) and CW-complexes are also $k$-spaces.

While not all topological spaces are $k$-spaces, any space can be 
``$k$-ified.''
The procedure is simple: endow the underlying set of a space $X$ with 
the topology consisting of all compactly closed sets.
The reader should check that this is indeed a topology on $X$.
The resulting topological space is denoted $kX$.
This construction immediately implies, for instance, that the identity 
$kX\to X$ is continuous, and is the terminal map to $X$ from a $k$-space.

Let $k\Top$ be the category of $k$-spaces.
This is a subcategory of the category of topological spaces, and we will write
$i:k\Top\hookrightarrow \Top$ for the inclusion functor.
The process of $k$-ification gives a functor $\Top\to k\Top$, which has the property that
$$k\Top(X,kY)=\Top(iX,Y)\,.$$
This is another example of an adjunction!
In this case the counit $kiX\to X$ is a homeomorphism.
We can conclude from this that the product in $k\Top$ may be computed as 
\[
X\times^{k\Top}Y=k(iX\times iY)\,.
\]

The category $k\Top$ has good categorical properties inherited from $\Top$:
it is a complete and cocomplete category.
As we will now explain, this category has even better categorical properties
than $\Top$ does.

\subsection{Mapping spaces}\label{mappingspaces}
Let $X$ and $Y$ be topological spaces.
The set $\Top(X,Y)$ of continuous maps from $X$ to $Y$ admits an
interesting topology, the {\em compact-open topology}.
If $X$ and $Y$ are $k$-spaces, we can make a slight modification:
Start with topology on $k\Top(X,Y)$ generated by the sets $W(k,U)$,
where $k:K\to X$ is a map from a compact Hausdorff space and $U\subseteq Y$
is open, defined by
\[
W(k,U) = \{f:X\to Y: f(k(K))\subseteq U\}\,.
\]
This topology is probably not itself determined by maps from compact
Hausdorff spaces, so $k$-ify it. 
We write $Y^X$ for the resulting $k$-space; its underlying
set is still the set of continuous maps from $X$ to $Y$. With this definition,
we have:
\begin{prop}
The category $k\Top$ is Cartesian closed.
\end{prop}
\begin{proof}
See \cite[Proposition 2.11]{StricklandCGWH}.
\end{proof}

