\section{Properties of the classifying
space}\label{classifying-space-properties}
%A great reference for simplicial sets is Goerss-Jardine, but it's about 500 pages long.
%Another reference is Weibel's homological algebra.
%And also May's simplicial objects book.

%Remember that $\Deltab\ni[n]$ for $n\geq 0$ where $[n] = \{0,\cdots,n\}$.
%A simplicial object is a functor $X:\Deltab^{op}\to\cc$.
%Remember that:
%$$
%|X| = \coprod_{n\geq 0}\Delta^n\times X_n/\sim
%$$
%Remember that $s^i:[n]\to [n+1]$ repeats $i$. Part of this relation $\sim$ is that $(v,s_i x)\sim (s^i v,x)$.
%That means that $x\in X_{n-1}$ and $v\in\Deltab^n$.
%This tells us that $\img(s_i:X_{n-1}\to X_{n})$ indexes cells that lie in the $(n-1)$-skeleton $\mathrm{sk}_{n-1}|X| = \coprod_{k\leq n-1}\Delta^k\times X_k/\sim$.
%
%This gives two functors
%$$
%|-|:s\Set\stackrel{\rightarrow}{\leftarrow} \Top:\Sin
%$$
%and I defined the unit and counit last time.
One important result in the story of geometric realization introduced in the
last section is the following theorem of Milnor's.
\begin{theorem}[Milnor]
    Let $X$ be a space. The map $|\Sin(X)|\to X$ is a weak equivalence.
\end{theorem}
As a consequence, this begets a functorial CW-approximation to $X$.
Unforunately, it's rather large.

In the last section, we saw that $|-|$ was a left adjoint. Therefore, it
preserves colimits (Theorem \ref{adjointslimits}). Surprisingly, it also
preserves products:
\begin{exercise}[Hard]\label{realization-product}
    Let $X$ and $Y$ be simplicial sets. Their product $X\times Y$ is defined to
    be the simplicial set such that $(X\times Y)_n = X_n\times Y_n$. Under this
    notion of product, there is a homeomorphism
    $$|X\times Y|\xrightarrow{\simeq} |X|\times |Y|.$$
    It is important that this product is taken in the category of $k$-spaces.
\end{exercise}

Last time, we defined the classifying space $B\cc$ of $\cc$ to be $|N\cc|$.
\begin{theorem}\label{classifying-product}
    The natural map $B(\cc\times\cd)\xrightarrow{\simeq}B\cc\times B\cd$ is a
    homeomorphism\footnote{Recall that if $\cc$ and $\cd$ are categories, the
    product $\cc\times\cd$ is the category whose objects are pairs of objects
    of $\cc$ and $\cd$, and whose morphisms are pairs of morphisms in $\cc$ and
    $\cd$.}.
\end{theorem}
\begin{proof}
    It is clear that $N(\cc\times\cd) \simeq N\cc\times N\cd$. Since $B\cc =
    |N\cc|$, the desired result follows from Exercise
    \ref{realization-product}.
\end{proof}
In light of Theorem \ref{classifying-product}, it is natural to ask how natural
transformations behave under the classifying space functor. To discuss this, we
need some categorical preliminaries.

The category $\Cat$ is Cartesian closed (Definition \ref{cartesian-closed}).
Indeed, the right adjoint to the product is given by the functor
$\cd\mapsto\Fun(\cc,\cd)$, as can be directly verified.

Consider the category $[1]$. This is particularly simple: a functor $[1]\to\cc$
is just an arrow in $\cc$. It follows that a functor $[1]\to\cd^\cc$ is a
natural transformation between two functors $f_0$ and $f_1$ from $\cc$ to
$\cd$. By our discussion above, this is the same as a functor
$\cc\times[1]\to\cd$.

By Theorem \ref{classifying-product}, we have a homeomorphism $B([1]\times
\cc)\simeq B[1]\times B\cc$. One can show that $B[1] = \Delta^1$, so a natural
transformation between $f_0$ and $f_1$ begets a map $\Delta^1\times B\cc\to
B\cd$ between $Bf_0$ and $Bf_1$. Concretely:
\begin{lemma}\label{nat-trans-htpy}
    A natural transformation $\theta:f_0\to f_1$ where $f_0,f_1:\cc\to\cd$
    induces a homotopy $Bf_0\sim Bf_1:B\cc\to B\cd$.
\end{lemma}
An interesting comment is in order. The notion of a homotopy is ``reversible'',
but that is definitely not true for natural transformations! The functor $B$
therefore ``forgets the polarity in $\Cat$''.

Lemma \ref{nat-trans-htpy} is quite powerful: consider an adjunction $L\dashv
R$ where $L:\cc\to \cd$; then we have natural transformations given by the unit
$1_\cc\to RL$ and the counit $LR\to 1_\cd$. By Lemma \ref{nat-trans-htpy} we
get a homotopy equivalence between $B\cc$ and $B\cd$. In other words, two
categories that are related by any adjoint pair are homotopy equivalent.

A special case of the above discussion yields a rather surprising result.
Consider the category $[0]$. Let $\cd$ be another category such that there is
an adjoint pair $L\dashv R$ where $L:[0]\to \cd$. Then $L$ determines an object
$\ast$ of $\cd$. Let $d$ be any object of $\cd$. We have the counit $LR(d)\to
d$; but $LR(d) = \ast$, so there is a unique morphism $\ast\to X$. (To see
uniqueness, note that the adjunction $L\dashv R$ gives an identification
$\cd(\ast,X) = \cc(0,0) = 0$.) In other words, such a category $\cd$ is simply
a category with an initial object.

Arguing similarly, any category $\cd$ with adjunction $L\dashv R$ where
$L:\cd\to [0]$ is simply a category with a terminal object. From our discussion
above, we conclude that if $\cd$ is any category with a terminal (or initial)
object, then $B\cd$ is contractible.
