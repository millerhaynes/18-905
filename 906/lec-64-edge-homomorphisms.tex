\section{Edge homomorphisms, transgression}
Recall the Serre spectral sequence for a fibration $F\to E\to B$ has $E^2$-page
given by
$$
E^2_{s,t} = H_s(B;H_t(F)) \Rightarrow H_{s+t}(E).
$$
If $B$ is path-connected, $\widetilde{H}_t(F) = 0$ for $t<q$,
$\widetilde{H}_s(B) = 0$ for $s<p$, and $\pi_1(B)$ acts trivially on
$H_\ast(F)$, we showed that there is a long exact sequence (the Serre exact
sequence)
\begin{equation}\label{serre-exact}
    H_{p+q-1}(F)\xrightarrow{\bullet} H_{p+q-1}(E)\to H_{p+q-1}(B)\to
    H_{p+q-2}(F)\to\cdots
\end{equation}
Let us attempt to describe the arrow marked by $\bullet$. %If $t>q$, we know

Let $(E^r_{p,q},d^r)$ be any spectral sequence such that $E^r_{p,q} = 0$ if
$p<0$ or $q<0$; such a spectral sequence is called a \emph{first quadrant}
spectral sequence. The Serre spectral sequence is a first quadrant spectral
sequence. In a first quadrant spectral sequence, the $d^2$-differential
$d^2:E^2_{0,t}\to E^2_{-2,t+1}$ is zero, since $E^2_{s,t}$ vanishes for $s<0$.
This means that $H_t(F) = H_0(B;H_t(F)) = E^2_{0,t}$ surjects onto $E^3_{0,t}$.
Arguing similarly, this surjects onto $E^4_{0,t}$. Eventually, we find that
$E^{r}_{0,t} \simeq E^{t+2}_{0,t}$ for $r\geq t+2$.
In particular,
$$E^{t+2}_{0,t} \simeq E^\infty_{0,t} \simeq \gr_0 H_t(E) \simeq F_0 H_t(E),$$
which sits inside $H_t(E)$. The composite
$$E^2_{0,t} = H_t(F) \to E^3_{0,t}\to \cdots\to E^{t+2}_{0,t} \subseteq F_0
H_t(E)\to H_t(E)$$
is precisely the map $\bullet$! Such a map is known as an \emph{edge
homomorphism}.

The map $F\to E$ is the inclusion of the fiber; it induces a map $H_t(F)\to
H_t(E)$ on homology. We claim that this agrees with $\bullet$.
%We almost saw this in the construction of a sseq for a filtered complex.
Recall that $F_0H_t(E)$ is defined to be $\img(H_t(F_0 E) \to H_t(E))$. In the
construction of the Serre spectral sequence, we declared that $F_0 E$ is
exactly the preimage of the zero skeleton. Since $B$ is simply connected, we
find that $F_0 E$ is exactly the fiber $F$.

To conclude the proof of the claim, consider the following diagram:
\begin{equation*}
    \xymatrix{
	F\ar[r]\ar[d] & F\ar[d]\\
	F\ar[r]\ar[d] & E\ar[d]\\
	\ast\ar@{^(->}[r] & B
    }
\end{equation*}
The naturality of the Serre spectral sequence implies that there is an induced
map of spectral sequences. Tracing through the symbols, we find that this
observation proves our claim.

The long exact sequence \eqref{serre-exact} also contains a map $H_s(E)\to
H_s(B)$. The group $F_s H_s(E) = H_s(E)$ maps onto $\gr_s H_s(E) \simeq
E^\infty_{s,0}$. If $F$ is connected, then $H_s(B) = H_s(B;H_0(F)) =
E^2_{s,0}$. Again, the $d^2$-differential $d^2:E^2_{s+2,-1}\to E^2_{s,0}$ is
trivial (since the source is zero). Since $E^3 = \ker d^2$, we have an
injection $E^3_{s,0} \to E^2_{s,0}$. Repeating the same argument, we get
injections
$$E^\infty_{s,0} = E^{s+1}_{s,0}\to \cdots\to E^2_{s,0}\to E^2_{s,0} =
H_s(B).$$
Composing with the map $H_s(E)\to E^\infty_{s,0}$ gives the desired map $H_s(E)
\to H_s(B)$ in the Serre exact sequence. This composite is also known as an
edge homomorphism.

As above, this edge homomorphism is the map induced by $E\to B$. This can be
proved by looking at the induced map of spectral sequences coming from the
following map of fiber sequences:
\begin{equation*}
    \xymatrix{
	F\ar[r]\ar[d] & \ast\ar[d]\\
	E\ar[r]\ar[d] & B\ar[d]\\
	B\ar[r] & B
    }
\end{equation*}

The topologically mysterious map is the boundary map $\partial:H_{p+q-1}(B)\to
H_{p+q-2}(F)$. Such a map is called a \emph{transgression}. Again, let
$(E^r_{s,t},d^r)$ be a first quadrant spectral sequence. In our case,
$E^2_{n,0} = H_n(B)$, at least $F$ is connected. As above, we have injections
$$i:E^n_{n,0} \to \cdots\to E^3_{n,0} \to E^2_{n,0} = H_n(B).$$
Similarly, we have surjections
$$s:E^2_{0,n-1}\to E^3_{0,n-1}\to \cdots\to E^n_{0,n-1}.$$
There is a differential $d^n:E^n_{n,0}\to E^n_{0,n-1}$. The transgression is
defined as the \emph{linear relation} (not a function!) $E^2_{n,0}\to
E^2_{0,n-1}$ given by
$$x\mapsto i^{-1} d^n s^{-1}(x).$$
However, the reader should check that in our case, the transgression is indeed
a well-defined function.

Topologically, what is the origin of the transgression? There is a map
$H_n(E,F)\xrightarrow{\pi_\ast} H_n(B,\ast)$, as well as a boundary map
$\partial : H_n(E,F) \to H_{n-1}(F)$. We claim that:
$$\img \pi_\ast = \img(E^n_{n,0}\to H_n(B) = E^2_{n,0}),\quad
\partial\ker\pi_\ast = \ker(H_{n-1}(F) = E^2_{0,n-1} \to E^n_{0,n-1}).$$
\begin{proof}[Proof sketch]
    Let $x\in H_n(B)$. Represent it by a cycle $c\in Z_n(B)$. Lift it to a
    chain in the total space $E$. In general, this chain will not be a cycle
    (consider the Hopf fibration). The differentials record this boundary; let
    us recall the geometric construction of the differential. Saying that the
    class $x$ survives to the $E^n$-page is the same as saying that we can find
    a lift to a chain $\sigma$ in $E$, with $d\sigma\in S_{n-1}(F)$. Then
    $d^n(x)$ is represented by the class $[dc]\in H_{n-1}(F)$. This is
    precisely the trangression.

    Informally, we lift something from $H_n(B)$ to $S_n(E)$; this is
    well-defined up to something in $F$. In particular, we get an element in
    $H_n(E,F)$. We send it, via $\partial$, to an element of $H_{n-1}(F)$ ---
    and this is precisely the transgression.
\end{proof}
\subsection{An example}
We would like to compare the Serre exact sequence \eqref{serre-exact} with the
homotopy exact sequence:
$$\ast\to \pi_{p+q-1}(F)\to \pi_{p+q-1}(E)\to \pi_{p+q-1}(B)\xar{\partial}
\pi_{p+q-2}(F)\to \cdots$$
There are Hurewicz maps $\pi_{p+q-1}(X)\to H_{p+q-1}(X)$. We claim that there
is a map of exact sequences between these two long exact sequences.
\begin{equation*}
    \xymatrix{
	H_{p+q-1}(E) \ar[r]^{\pi_\ast} & H_{p+q-1}(B)\ar[r]_\partial &
	H_{p+q-2}(F)\ar[r] & \cdots\\
	\pi_{p+q-1}(E)\ar[r]_{\pi_\ast}\ar[u]_{h} &
	\pi_{p+q-1}(B)\ar[u]^h\ar[r] & \pi_{p+q-2}(F)\ar[r]\ar[u]^h &
	\cdots\\
    }
\end{equation*}
The leftmost square commutes by naturality of Hurewicz. The commutativity of
the righmost square is not immediately obvious. For this, let us draw in the
explicit maps in the above diagram:
\begin{equation*}
    \xymatrix{
	& & H_{p+q-1}(E,F)\ar[dl]\ar[dr] & &\\
	H_{p+q-1}(E) \ar[r]^{\pi_\ast} & H_{p+q-1}(B)\ar[rr]_\partial & &
	H_{p+q-2}(F)\ar[r] & \cdots\\
	\pi_{p+q-1}(E)\ar[r]_{\pi_\ast}\ar[dr]\ar[u]_{h} &
	\pi_{p+q-1}(B)\ar[u]^h\ar[rr] & & \pi_{p+q-2}(F)\ar[r]\ar[u]^h &
	\cdots\\
	& \pi_{p+q-1}(E,F)\ar[uuur]\ar[urr]\ar[u]^\cong_{s} & &
    }
\end{equation*}
The map marked $s$ is an isomorphism (and provides the long arrow in the above
diagram, which makes the square commute), since
$$
\pi_n(E,F) = \pi_{n-1}(\mathrm{hofib}(F\to E)) = \pi_{n-1}(\Omega B) =
\pi_n(B).
$$
Let us now specialize to the case of the fibration
$$\Omega X\to PX\to X.$$
Assume that $X$ is connected, and $\ast \in X$ is a chosen basepoint. Let
$p\geq 2$, and suppose that $\widetilde{H}_s(X) = 0$ for $s<p$. Arguing as in
\S \ref{loops-sn}, we learn that the Serre spectral sequence we know that the
homology of $\Omega X$ begins in dimension $p-1$ since $PX\simeq \ast$, so $q =
p-1$. Likewise, if we knew $\widetilde{H}_n(\Omega X) = 0$ for $n<p-1$, then
the same argument shows that $\widetilde{H}_n(X) = 0$ for $n<p$.
\subsection*{A surprise gust: the Hurewicz theorem}
The discussion above gives a proof of the Hurewicz theorem; this argument is
due to Serre.
\begin{theorem}[Hurewicz, Serre's proof]
    Let $p\geq 1$. Suppose $X$ is a pointed space with $\pi_i(X) = 0$ for
    $i<p$. Then $\widetilde{H}_i(X) = 0$ for $i<p$ and $\pi_p(X)^{ab}\to
    H_p(X)$ is an isomorphism.
\end{theorem}
\begin{proof}
    Let us assume the case $p=1$. This is classical: it is Poincar\'{e}'s
    theorem. We will only use this result when $X$ is a loop space, in which
    case the fundamental group is already abelian.

    Let us prove this by induction, using the loop space fibration. By
    assumption, $\pi_i(\Omega X) = 0$ for $i<p-1$. By our inductive hypothesis,
    $\widetilde{H}_i(\Omega X) = 0$ for $i<p-1$, and $\pi_{p-1}(\Omega X)
    \xrightarrow{\simeq} H_{p-1}(\Omega X)$. By our discussion above, we learn
    that $\widetilde{H}_i(X) = 0$ for $i<p$. The Hurewicz map
    $\pi_p(X)\xrightarrow{h}H_p(X)$ fits into a commutative diagram:
    \begin{equation*}
	\xymatrix{
	    \pi_{p-1}(\Omega X)\ar[r]^\simeq & H_{p-1}(\Omega X)\\
	    \pi_p(X)\ar[u]^\simeq \ar[r]_h &
	    H_p(X)\ar[u]^{\simeq}_{\text{transgression}}
	    }
    \end{equation*}
    It follows from the Serre exact sequence that the transgression is an
    isomorphism.
\end{proof}
%This proof has an enormous advantage, since you can make modifications that modify all primes except for a single prime, or get rational information.
%In other words, it's amenable to localizations.
%On Monday we'll talk about Serre classes and get information about homotopy groups way beyond conductivity of the space, if you do it right.
