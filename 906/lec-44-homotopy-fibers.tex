\section{Homotopy fibers}

Fix a map $p:E\to Y$. The pullback of $E$ along a map $f:X\to Y$ can vary
wildly as $f$ is deformed; it is far from being a homotopy invariant. 
Just think of the case $X=*$, for example, when the pullback along 
$f:*\to Y$ is the point preimage $p^{-1}(f(*))$. One of the great
features of fibrations is this:

\begin{prop}
Let $p:E\to Y$ be a fibration and $f_0,f_1:X\to Y$ two maps.  
Write $E_0$ and $E_1$ for pullbacks of $E$ along $f_0$ and $f_1$.
If $f_0$ and $f_1$ are homotopic then $E_0$ and $E_1$ are homotopy
equivalent.
\end{prop}
\begin{proof}
Let $h:I\times X\to Y$ be a homotopy from $f_0$ to $f_1$. Its adjoint
is a path $\widehat h:I\to Y^X$ from $f_0$ to $f_1$. The fibration $p:E\to Y$
induces a fibration $E^X\to Y^X$, and the path determines a homotopy 
equivalence from the fiber over $f_0$ to the fiber over $f_1$. 
These two fibers are $E_0$ and $E_1$. 
\end{proof}

\subsection{``Fibrant replacements''}
This result makes fibrations extremely valuable in homotopy theory. 
Not every map is a fibration, but every map can be ``replaced'' by one, up
to homotopy.
\begin{theorem}\label{fibrep}
    For any map $f:X\to Y$, there is a space $T(f)$, along with a fibration $p:T(f) \to Y$ and
    a homotopy equivalence $X \xar{\simeq} T(f)$, such that the following diagram
    commutes:
    \begin{equation*}
	\xymatrix{
	    X\ar[r]^{\simeq}\ar[dr]_f & T(f)\ar[d]^p\\
	    & Y.
	    }
    \end{equation*}
\end{theorem}
\begin{proof}
We already have one example of this in hand: The diagonal map 
$\Delta:X\to X\times X$ factors as 
    \begin{equation*}
	\xymatrix{
	    X\ar[r]^{c}\ar[dr]_\Delta & X^I\ar[d]^{\mathrm{ev}_{0,1}}\\
	    & X\times X\,.
	    }
    \end{equation*}
where $p$ evaluates at the end points of the unit interval and $c$ sends
$x$ to the constant path at $x$. We've seen that $\mathrm{ev}_{0,1}$ 
is a fibration.
The map $c$ has as homotopy inverse any evaluation map, say $\mathrm{ev}_0$.
The composite $\mathrm{ev}_0\circ c$ is the identity, while 
$c\circ\mathrm{ev}_0$ is homotopic to the identity via the homotopy $h$ with
\[
h(s,\omega)(t)=\omega(st)\,.
\]
This ``spaghette move'' shortens the path from $\omega$ at $s=1$ to the 
constant path $c_{\omega(0)}$ at $s=0$. 

Now we can construct the general case by considering the graph of $f$ instead 
of the diagonal (which is the graph of the identity map). So form the pullback
	\begin{equation*}
	    \xymatrix{
T(f)\ar[r]\ar[d] \ar@/_3pc/[dd]_p & Y^I\ar[d]^{\mathrm{ev}_{0,1}}\\
		    X\times Y\ar[r]^{f\times 1} \ar[d]^{\pr_2} & Y\times Y\\
Y & 
		}
	\end{equation*}
so that
\[
T(f)=\{(x,\omega)\in X\times Y^I:f(x) = \omega(0)\}\,.
\]
The map $T(f)\to X\times Y$ 
is a base-change of the fibration $\mathrm{ev}_{0,1}$ and so is
a fibration; the projection map $\pr_2:X\times Y\to Y$ is a fibration; so
their composite $p$ is a fibration. To get $X$ involved, look at the diagram
\[
\xymatrix{
X \ar[r]^f \ar[d]^\Delta \ar@{.>}[dr] & Y \ar[dr]^c \\
X\times X \ar[dr]^{1\times f} & T(f) \ar[r] \ar[d] & 
Y^I \ar[d]^{\mathrm{ev}_{01}} \\
& X\times Y \ar[r]^{f\times1} \ar[d]^{\pr_1} & Y\times Y\\
& X && \,.
}\]
The outside diagram commutes, so we get a map from $X$ to the pullback $T(f)$. 
In symbols,
\[
x\mapsto(x,c_{f(x)})\,.
\]
We claim that the composite  $T(f)\to X\times Y\to X$ is a homotopy inverse. 
The composite $X\to T(f)\to X\times Y\to X$ is the identity, so we just need
a homotopy joining the other composite to the identity. Such a homotopy
$I\times T(f)\to T(f)$ is provided by the same spaghetti move: 
\[
(s,x,\omega)\mapsto(x,t\mapsto\omega(st))\,.
\]
\end{proof}

\begin{example}[Path-loop fibration]
        If $X = \ast$, the space $T(f)$ consists of paths $\omega$ in $Y$ such that $\omega(0) = \ast$.
    In other words, $T(f) = Y^I_\ast$; this is called the \emph{path space} of $Y$, and is denoted by $P(Y,\ast)$, or simply by $PY$. It is
contractible, via the spaghetti move. 
    The fiber over $\ast$ of the fibration $p:T(f) = PY \to Y$ consists of paths that begin and end at $\ast$, i.e., 
    loops on $Y$ based at $\ast$.
    This is denoted $\Omega Y$, and is called the \emph{loop space} of $Y$.
    The fibration $p:PY \to Y$ is called the \emph{path-loop fibration}.
\end{example}

\begin{exercise}[``Cofibrant replacements'']\label{cofibrep}
    In this exercise, you will prove the analogue of Theorem \ref{fibrep} for cofibrations.
    Let $f:X\to Y$ be any map.
    Show that $f$ factors (functorially) as a composite $X \to M \to Y$, where $X\to M$ is a cofibration and $M\to Y$ is a homotopy
    equivalence.
\end{exercise}

\subsection{Homotopy fibers}
Let $f:E\to Y$ a map. As you move around $Y$, the point preimages
generally vary, even up to homotopy type. They may become empty,
for example. We saw above that assuming that $f$ is a fibration
cures this problem,
and we have just seen that any map can be replaced, up to homotopy,
by a fibration. So it is sensible to make the following definition:
\begin{definition}
Let $(Y,*)$ be a pointed space. 
The \emph{homotopy fiber} of a map $f:E\to Y$ is the fiber over $*$ of
the fibration replacing $f$; that is, it is the pullback in
    \begin{equation*}
	\xymatrix{
	    F(f,\ast)\ar[r]\ar[d] & T(f)\ar[r]^\simeq \ar[d]^p & X\ar[dl]^f\\
	    \ast \ar[r] & Y & \,.
	    }
    \end{equation*}
\end{definition}
Homotopy fibers over different points in the same path component are homotopy
equivalent. 

As a set, 
\begin{equation}
\label{sethomotopyfiber}
F(f,\ast) = \{(x,\omega)\in E\times Y^I: \omega(0)=f(x), \omega(1) = \ast\}.
\end{equation}
So an element in the homotopy fiber of $f$ over $*$ is not just a
point in $E$ lying over $*\in Y$; it is a point $x\in E$ together with a path 
in $Y$ joining $f(x)$ to $\ast$ in $Y$. 

The inclusion of the fiber over $*\in Y$, $f^{-1}(*)\hookrightarrow E$,
is univeral among maps $g:W\to E$ such that $fg=*$. What is the corresponding
universal property of the homotopy fiber? Adjointing a map $W\to T(f)$ 
gives us a map $g:W\to E$ together with a homotopy from $fg$ to the constant 
map with value $*$: a {\em null homotopy} of the composite. 

\begin{remark} In forming the homotopy fiber of $f$, we replaced $f$ by
a homotopy equivalent fibration and formed the pullback over $*$. 
We could alternatively have replaced the map $*\to Y$ by a fibration
(the path-loop fibration!) and formed the pullback along $f$. The result of
this operation is
\[
\{(x,\omega)\in X\times Y^I:\omega(0) = \ast, \omega(1) = f(x)\}\,.
\]

Our description of $F(f,\ast)$ in \eqref{sethomotopyfiber} is almost exactly the same --- except that
the directions of the paths are reversed. These two ways of thinking of the
homotopy fiber are completely equivalent.
\end{remark}
