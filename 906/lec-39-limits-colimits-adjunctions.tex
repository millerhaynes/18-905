\section{Limits, colimits, and adjunctions}\label{906}

\subsection{Limits and colimits}

I want to begin by developing a little more category theory.

\begin{definition}
Suppose $\cI$ is a small category (so that it has a \emph{set} of objects),
and let $\cc$ be another category.  Let $X:\cI\to\cc$ be a functor.
    A \emph{cone under $X$} is a natural transformation $\eta$ from $X$ to a constant functor; to be explicit, this means that for every object $i$ of $\cI$
we  have a map $\eta_i: X_i\to Y$, and these maps are compatible in the sense 
that for every $f:i\to j$ in $\cI$ the following diagram commutes:
    \begin{equation*}
	\xymatrix{
	    X_i\ar[d]^{f_\ast}\ar[r]^{\eta_i} & Y \ar[d]^= \\
	    X_j\ar[r]^{\eta_j} & Y\,.
	    }
    \end{equation*}
    A \emph{colimit} of $X$ is an initial cone $(L,\tau_i)$ under $X$; to be 
    explicit, this means that for any cone $(Y,\eta_i)$ under $X$,
    there exists a unique map $h:L\to Y$ such that $h\circ \tau_i = \eta_i$
for all $i$.
\end{definition}
As with any universal property, any two colimits are isomorphic by a unique
isomorphism; but existence is another matter. Also, 
as always for category theoretic concepts, some examples are in order.
\begin{example}\label{coproductsarecolimits}
    If $\cI$ is a discrete category (that is, the only maps are identity maps;
$\cI$ is entirely determined by its set of objects), 
the colimit of a functor $\cI\to\cc$
    is the coproduct in $\cc$ (if this coproduct exists!).
\end{example}
\begin{example}
In Lecture 23 we discussed directed posets and the direct limit of a directed 
system 
$X:\cI\to\cc$. The colimit simply generalizes this to arbitrary
indexing categories rather than restricting to directed partially 
ordered sets. 
\end{example}
\begin{example}\label{colimitgroupaction}
    Let $G$ be a group; we can view this as a category with one object, where the morphisms are the elements of the group and composition is given by the group structure.
    If $\cc = \Top$ is the category of topological spaces, a functor $G\to \cc$ is simply a
    group action on a topological space $X$.
    The colimit of this functor is the orbit space of the $G$-action on $X$.
\end{example}
\begin{example}
    Let $\cI$ be the category whose objects and morphisms are determined by the following directed graph:
    \begin{equation*}
	\xymatrix{ & a\ar[dl]\ar[dr] & \\
	b & & c.}
    \end{equation*}
    The colimit of a diagram $\cI\to \cc$ is called a \emph{pushout}.
With  $\cc=\Top$, again, a functor $\cI\to \cc$ is determined by a diagram of spaces:
    \begin{equation*}
	\xymatrix{
	    & A\ar[dl]_f\ar[dr]^g & \\
	    B & & C.
	    }
    \end{equation*}
    The colimit of such a functor is just the pushout $B\cup_A C:= B\sqcup C/\sim$, where $f(a)\sim g(a)$ for all $a\in A$.
    We have already seen this in action before: a special case of this construction appears in the process of attaching cells to build up a CW-complex.

    If $\cc$ is the category of groups, instead, the colimit of such a functor is the free product quotiented out
    by a certain relation; this is called the \emph{amalgamated free product}.
\end{example}
\begin{example}
    Suppose $\cI$ is the category defined by the following directed graph:
\begin{equation*}
    \begin{tikzcd}
	a\ar[r,shift left=.75ex]\ar[r,shift right=.75ex] & b.
    \end{tikzcd}
\end{equation*}
    The colimit of a diagram $\cI\to\cc$ is called the \emph{coequalizer} of the diagram. If $\cc=\Set$, the coequalizer of $f,g:A\rightrightarrows B$ is the 
quotient of $B$ by the equivalence relation generated by $f(a)\sim g(a)$ for
$a\in A$.
\end{example}

    One can also consider cones \emph{over} a diagram $X:\cI\to\cc$: this is simply a cone in the opposite category.
\begin{definition}
The \emph{limit} of a diagram $X:\cI\to\cc$ is a terminal object in cones over $X$.
\end{definition}

\begin{exercise}
    Revisit the examples provided above: what is the limit of each diagram?
For instance, a product is a limit over a discrete category, 
and the limit of the diagram described in Example \ref{colimitgroupaction} is just the fixed points.
\end{exercise}

\subsection{Adjoint functors}
The notion of a colimit as a special case of the more general concept of an 
adjoint functor, as long as we are dealing with a cocomplete category as in
the following definition.
\begin{definition}
    A category $\cc$ is \emph{cocomplete} if all functors from small categories to $\cc$ have colimits.
    Similarly,  $\cc$ is \emph{complete} if all functors from small categories
to $\cc$ have limits. 
\end{definition}
All examples considered above are both cocomplete and complete.

Let's write $\cc^\cI$ for the category of functors from $\cI$ to $\cc$,
and natural transformations between them.


There is a functor $c:\cc\to \cc^\cI$, given by sending any object to the constant functor taking on that value.
The process of taking the colimit of a diagram supplies us with a functor 
$\colim_{\cI}:\cc^\cI\to \cc$.
We can characterize this functor via the formula
\[
\cc(\colim{}_{i\in \cI} X_i,Y) = \cc^\cI(X,c_Y)\,,
\]
where $X$ is any functor from $\cI$ to $\cc$, $Y$ is any object of $\cc$,
and $c_Y$ denotes the constant functor with value $Y$.
This formula is reminiscent of the adjunction operator in linear algebra,
and is in fact our first example of an adjunction.

\begin{definition}
    Let $\cc,\cd$ be categories, and suppose given functors 
$F:\cc\to \cd$ and $G:\cd\to\cc$.
    An \emph{adjunction between $F$ and $G$} is an isomorphism:
    $$\cd(FX,Y) = \cc(X,GY),$$
    that is natural in $X$ and $Y$.
    In this situation, we say that $F$ is a \emph{left adjoint} of $G$ and $G$ is a \emph{right adjoint} of $X$.
    %People typically write left adjoints on the top.
\end{definition}
This notion was invented by the late MIT Professor Dan Kan.

We've already seen one example of adjoint functors. Here is another one.
\begin{definition}[Free groups]
    There is a forgetful functor $u:\mathbf{Grp}\to\mathbf{Set}$.
    Any set $X$ gives rise to a group $FX$, the free group on $X$ elements.
    It is determined by a universal property:
    set maps $X\to u\Gamma$ are the same as group maps $FX\to \Gamma$,
    where $\Gamma$ is any group.
    This is exactly saying that the free group functor the left adjoint to the forgetful functor $u$.
\end{definition}
In general, ``free objects'' come from left adjoints to forgetful functors.

As a general notational practice, try to write the left adjoint as the top 
arrow: 
\[
F:\cc\rightleftarrows\cd:G \qquad\hbox{or}\qquad G:\cd\leftrightarrows\cc:F \,.
\]
 



\subsection{The Yoneda lemma}
One of the many important principles in category theory is that an object is determined by the collection of all maps out of it.
The Yoneda lemma is a way of making this precise. Observe that 
for any $X\in\cc$ the association $Y\mapsto\cc(X,Y)$ gives us a functor
$\cc\to\Set$. This functor is said to be {\em corepresentable} by $X$.
Suppose that $G:\cc\to\Set$ is any functor. An element $x\in G(X)$ 
determines a natural transformation 
\[
\theta_x:\cc(X,-)\to G
\]
in the following way. Let $Y\in\mathrm{ob}\cc$. Map $\cc(X,Y)\to G(Y)$
by sending $f:X\to Y$ to $f_*(x)\in G(Y)$. 

\begin{lemma}[Yoneda lemma]
The association $x\mapsto\theta_x$ provides a bijection
    $$\mathrm{nt}(\cc(X,-),G)\xrightarrow{\cong}G(X).$$
\end{lemma}
\begin{proof}
The inverse is given as follows: Send a natural transformation 
$\theta:C(X,-)\to G$ to $\theta_X(1_X)\in G(X)$. 
\end{proof}
In particular, if $G$ is also corepresentable -- $G=\cc(Y,-)$, say --
then 
\[
\mathrm{nt}(\cc(X,-),\cc(Y,-))\cong \cc(Y,X)\,.
\]
Simply put, natural transformations $\cc(X,-)\to \cc(Y,-)$ are in natural
bijection with maps $Y\to X$. Consequently natural isomorphisms 
$\cc(X,-)\xrightarrow{\cong}\cc(Y,-)$ are in natural bijection with 
isomorphims $Y\xrightarrow{\cong}X$. 
So, for example, the object that corepresents a corepresentable functor 
is unique up to unique isomorphism.

From the Yoneda lemma, we can obtain some pretty miraculous conclusions.
For instance, functors with left and/or right adjoints are very well-behaved.
\begin{theorem}\label{adjointslimits}
    Let $F:\cc\to\cd$ be a functor.
    If $F$ admits a right adjoint then it preserves colimits.
    Dually, if $F$ admits a left adjoint then it preserves limits.
\end{theorem}
\begin{proof}
    We'll prove the first statement; the dual proof gives the other.
    Let $F:\cc\to\cd$ be a functor that admits a right adjoint $G$, and let $X:\cI\to\cc$ be a
    small $\cI$-indexed diagram in $\cc$. Suppose that the colimit 
$\colim_{\cI}X$ exists. From the definition of colimit, there is
    for any object $Y$ of $\cc$ an isomorphism
$$
\cc(\colim_{\cI} X, Y) \cong \lim{}_{\cI}\cc(X, Y)\,.
$$
   
    Let $Z$ be any object of $\cd$, and note the sequence of natural 
isomorphisms:
    \begin{align*}
	\cd(F(\colim_{\cI} X), Y) & \cong \cc(\colim_{\cI} X, G(Y))\\
	& \cong \lim{}_{\cI}\cc(X, G(Y))\\
	& \cong \lim{}_{\cI}\cd(F(X), Y)\\
	& \cong \cd(\colim_{\cI}F(X), Y).
    \end{align*}
    The Yoneda lemma now finishes the job.
\end{proof}
