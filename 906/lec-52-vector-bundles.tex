\section{Vector bundles}

Each point $x$ in a smooth manifold $M$ admits a ``tangent space.'' This is a
real vector space, whose elements are equivalence classes of smooth
paths $\sigma:\RR\to M$ such that $\sigma(0)=x$. The equivalence 
relation retains only the velocity vector at $t=0$. These vector spaces
``vary smoothly'' over the manifold. The notion of a vector bundle is a
topological extrapolation of this idea. 

Let $X$ be a topological space. To begin with, we will define a
{\em  vector space over} $B$ to be a map $E\to B$ along with
the following extra data:
\begin{itemize}
    \item an ``addition'' $\mu:E\times_B E\to E$, compatible with the maps
	down to $B$;
    \item a ``zero section'' $s:B\to E$ such that the composite $B\xar{s}E\to
	B$ is the identity;
    \item an inverse $\chi:E\to E$, compatible with the map down to $B$; and
    \item an action of $\RR$:
	\begin{equation*}
	    \xymatrix{
		\RR\times E\ar[dr]_{p\circ \pr_2}\ar@{=}[r] &
		(B\times\RR)\times_B E\ar[r]\ar[d] & E\ar[dl]^p\\
		& B &
	    }
	\end{equation*}
\end{itemize}
These data are required to render each fiber a real vector space of finite
dimension. 
\begin{example}\label{trivialvectorbundle}
    A ``trivial'' example of a vector space over $B$ is the projection
    $B\times V\to B$ where $V$ is a real vector space of finite dimension $n$.
This is the {\em trivial vector space} over $B$ of dimension $n$.
\end{example}
Vector spaces over $X$ form a category in which the morphisms are maps 
covering the identity map of $X$ that are linear on each fiber. 

\begin{example}
Let $p:S\to\RR$ have $p^{-1}(0)=\RR$ and $p^{-1}(s)=0$ for $s\neq0$. 
With the evident structure maps, this is a perfectly good (``skyscraper'') 
vector space over $\RR$. This example is peculiar, however; it is not
locally constant. Our definition of vector bundles will exlude it and
similar oddities. Sheaf theory is the proper home for examples like this.

But this example occurs naturally even if you restrict to trivial 
bundles and maps between them. The trivial bundle 
$\pr_1:\RR\times\RR\rightarrow\RR$
has as an endomorphism the map 
\[
(s,t)\mapsto(s,st)\,.
\]
This map is an isomorphism on almost all fibers, but is zero over $s=0$.
So if you want to form a kernel or a cokernel, you will get the skyscraper 
vector space over $\RR$. This puts limitations on the operations we can form 
on vector bundles, if we want them to result in vector bundles. 
\end{example}

\begin{definition}
A \emph{vector bundle} over $B$ is a vector space $E$ over $B$ that is locally
trivial; that is, every point $b\in B$ has a neighborhood over which $E$ is
isomorphic to a trivial bundle. 
\end{definition}
\begin{remark}
As in our definition of fiber bundles, we will always assume that a vector
bundle admits a numerable trivializing cover. And, to repeat, the fiber
dimensions of our vector bundles will always be finite. On the other hand,
there is nothing to stop us from replacing $\RR$ with $\CC$ or even with
the quaternions $\HH$, and talking about complex or quaternionic vector
bundles. 
\end{remark}
If $p:E\to B$ is a vector bundle, then $E$ is called the \emph{total space},
$p$ is called the \emph{projection map}, and $B$ is called the \emph{base
space}. There are various notations in use for vector bundles, and we will
switch among them. So we will use a Greek letter like $\xi$ or $\zeta$ to 
denote the entire structure, and $E(\xi),B(\xi)$ denotes the total
space and base space. We may write $\xi\downarrow B$ to indicate a vector
bundle over $B$, and, indeed, use $\downarrow$ rather than $\to$ for
projection maps in general. 
\begin{example}
The $n$-dimensional trivial bundle $B\times\RR^n\downarrow B$ 
will be denote by $n\epsilon$. 
\end{example}
\begin{example}
\label{grassmannianvb}
At the other extreme,
Grassmannians support highly nontrivial vector bundles.
We can form Grassmannians over any one of the three (skew)fields $\RR,\CC,\HH$.
Write $K$ for one of them, and consider the (left) $K$-vector space $K^n$. 
The {\em Grassmannian} (or {\em Grassmann manifold}) $\Gra_k(K^n)$ is 
the space of $k$-dimensional $K$-subspaces of $K^n$. As we saw last term,
this is a topologized as a quotient space of a Stiefel variety of $k$-frames
in $K^n$. To each point in $\Gra_k(K)$ is associated a $k$-dimensional subspace
of $K^n$. This provides us with a $k$-dimensional $K$-vector space $\xi_{n,k}$
over $\Gra_k(K^n)$, with total space
\begin{equation*}
	    E(\xi_{n,k}) = \{(V,x)\in\Gra_k(K^n)\times K^n:x\in V\}
	\end{equation*}
This is the {\em canonical} or {\em tautologous} vector bundle over 
$\Gra_k(K^n)$. It occurs as a subbundle of $n\epsilon$.  
	\begin{exercise}
	    Prove that $\xi_{n,k}$, as defined above, is locally trivial, so
	    is a vector bundle over $\Gra_k(K^n)$.
	\end{exercise}
	    For instance, when $k=1$, we have $\Gra_1(\RR^n) = \RP^{n-1}$.
The tautologous bundle $\xi_{n,1}$ is 1-dimensional; it is a {\em line bundle},
the canonical line bundle over $\RP^{n-1}$. We may write $\gamma$ for
this line bundle.  
\end{example}
\begin{example}
Let $M$ be a smooth manifold. Define $\tau_M$ to be the tangent
	bundle $TM\to M$ over $M$. For example, if $M = S^{n-1}$, then
	$$TS^{n-1} = \{(x,v)\in S^{n-1}\times\RR^n:v\cdot x = 0\}.$$
\end{example}

\subsection{Constructions with vector bundles}
One cannot take the kernels or cokernels of a map of vector bundles; 
but just about
anything which can be done for vector spaces can also be done for vector
bundles:
\begin{enumerate}
    \item The pullback of a vector bundle is again a vector bundle:
If $p^\prime:E^\prime\to B^\prime$ is a vector bundle then the
	 map $p$ in the diagram below is also a vector bundle.
	\begin{equation*}
	    \xymatrix{
		E\ar[r]^{\overline f} \ar[d]^p & E^\prime\ar[d]^{p^\prime}\\
		B\ar[r]_f & B^\prime
		}
	\end{equation*}
	For instance, if $B=\ast$, the pullback is just the fiber of $E^\prime$
	over the point $\ast\to B^\prime$. The cover by products of the 
elements of trivializing covers trivialize the product

 If $\xi$ is the bundle $E^\prime\to
	B^\prime$, we denote the pullback $E\to B$ by $f^\ast \xi$.

There's a convenient way to think of a pullback: the top map $\overline{f}$
in the pullback diagram has two key properties: It covers $f$, and it is a
linear isomorphism on fibers. These conditions suffice to present $p$ as 
the pullback of $p'$ along $f$. 
    \item If $p:E\to B$ and $p^\prime:E^\prime\to B^\prime$, then we can take
	the product $E\times E^\prime\xrightarrow{p\times p^\prime}B\times
	B^\prime$. 
    \item If $B=B^\prime$, we can form the pullback:
	\begin{equation*}
	    \xymatrix{
		E\oplus E^\prime\ar[r]\ar[d] & E\times E^\prime\ar[d]\\
		B\ar[r]^{\Delta} & B\times B
		}
	\end{equation*}
	The bundle $E\oplus E^\prime$ is called the \emph{Whitney sum}. For
	instance, 
	$$n\epsilon = \epsilon\oplus\cdots\oplus\epsilon.$$
    \item If $E,E^\prime\to B$ are two vector bundles over $B$, we can form
	another vector bundle $E\otimes E^\prime\to B$ by taking the
	fiberwise tensor product. Likewise, taking the fiberwise Hom 
produces a
	vector bundle $\Hom(E,E^\prime)\to B$.
\end{enumerate}


\begin{example}
    Recall from Example \ref{grassmannianvb}  the tautological bundle
    $\gamma$ over $\RP^{n-1}$. The tangent
    bundle $\tau_{\RP^{n-1}}$ also lives over $\RP^{n-1}$. As this is the first
    explicit pair of vector bundles over the same space, it is natural to
    wonder what is the relationship between these two bundles.
    We claim that
    $$\tau_{\RP^{n-1}} = \Hom(\gamma,\gamma^\perp).$$
    To see this, make use of the $2$-fold covering map $S^{n-1}\to
    \RP^{n-1}$. The projection map is smooth, and covered by a fiberwise
isomorphism of tangent bundles. The fibers $T_xS^{n-1}$ and $T_{-x}S^{n-1}$
are both identified with the orthogonal complement of $\RR x$ in $\RR^n$,
and the differential of the antipodal map sends $v$ to $-v$. So the tangent
vector to $\pm x\in\RP^{n-1}$ represented by $(x,v)$ 
is the same as the tangent vector represented by $(-x,-v)$. This describes
the vector bundle $\Hom(\gamma,\gamma^\perp)$.
\end{example}
\begin{exercise}
    Prove that 
    $\Gra_k(K^n)$, for $K=\RR,\cC,\HH$, then
    $$\tau_{\Gra_k(K^n)} = \Hom(\xi_{n,k},\xi_{n,k}^\perp).$$
\end{exercise}
\subsection{Metrics and splitting exact sequences}
A map of vector bundles, $\xi\to\eta$, over a fixed base can be identified
with a section of $\Hom(\xi,eta)$.  
We have seen that the kernel and cokernel of a homomorphism will be vector
bundles only of the rank is locally constant. 
\begin{exercise} Prove the converse. 
\end{exercise} 
In particular, we can form kernels of surjections and cokernels of injections;
and consider short exact sequences of vector bundles. It is a characteristic
of topology, as opposed to analyic or algebraic geometry, that short exact
sequences of vector bundles always split. To see this we use a ``metric.''
\begin{definition}
A \emph{metric} on a vector bundle is a continuous choice of inner products on
the fibers.
\end{definition}
\begin{lemma}
    Any (numerable) vector bundle $\xi$ over $X$ admits a metric.
\end{lemma}
\begin{proof}
This will use the fact that if $g,g^\prime$ are both inner
products on a vector space then $tg+(1-t)g^\prime$ is another; 
and more generally that the space of metrics forms a real affine space.

    Pick a trivializing open cover $\cU$ for $\xi$, and for each $U\in\cU$
an isomorphism $\xi|_U\cong U\times V_U$. Pick an inner product $g_U$
on each of the vector spaces $V_U$. Pick a partition of unity subordinate 
to $\cU$; that is, functions $\phi_U:U\to [0,1]$ such that the
    preimage of the complement of $0$ is $U$ and
    $$
\sum_{x\in U} \phi_U(x) = 1\,.
$$
Now the sum
    $$
g \coloneqq \sum_{U}\phi_U g_U
$$
is a metric on $\xi$.
\end{proof}
\begin{corollary}\label{split}
    Any exact sequence $0\to\xi^\prime\to\xi\to\xi^{\prime\prime}\to 0$ 
of vector bundles (over the same base) splits.
\end{corollary}
\begin{proof}
    Pick a metric for $\xi$. Using it, form the orthogonal complement 
$\xi^{'\perp}$. The composite
    $$
{\xi^\prime}^\perp\hookrightarrow\xi\to\xi^{\prime\prime}
$$
is an isomorphism: the dimensions of the fibers are the same. 
This provides a splitting of the surjection $\xi\to\xi''$ and hence 
of the short exact sequence.
\end{proof}

