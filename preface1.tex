\cleardoublepage
\section*{Preface}
Over the 2016--2017 academic year, I ran the graduate algebraic topology 
sequence at MIT. The first semester traditionally deals with singular homology
and cohomology and Poicar\'e duality; the second builds up basic homotopy 
theory, spectral sequences, and characteristic classes. 

My goal was to give a pretty standard classical approach to these subjects. 
In the first semester, I had various more specific objectives as well. 
I wanted to introduce students to the basic language of category theory
and simplicial sets, so useful throughout
mathematics and finding their first real manifestations in algebraic 
topology. I wanted to stress the methods of homological algebra,
for similar reasons. And I especially wanted 
to give an honest account of the machinery -- relative cap product and
\v{C}ech cohomology --  needed in the proof of Poincar\'e duality. 
The present document contains a bit more detail on these last matters
than was presented in the course itself.

On the other hand I barely touched on some important subjects. 
I did not talk about simplicial complexes at all, nor about the 
Lefschetz fixed point theorem. I gave only a brief summary of
the theory of covering spaces and the fundamental group, in preparation
for a proper understanding of orientations. I avoided some point set
topology by working with only compact subspaces rather than general closed 
subspaces in the development of Poincar\'e duality. 

I was lucky enough to have in the audience a student, Sanath Devalapurkar, 
who spontaneously decided to {live\TeX} the entire course. This resulted in 
a remarkably accurate record of what happened in the classroom -- right down
to random alarms ringing and embarassing jokes and mistakes on the blackboard. 
Sanath's \TeX{} forms the basis of these notes, and I am grateful to him 
for making them available. The attractive  drawings were provided by
another student, Xianglong Ni, who also carefully proofread the manuscript.

In the course of editing these notes, beyond correcting various errors 
(while hopefully not introducting too many new ones), 
I completed a few arguments
not done in detail in the actual lectures and rearranged some of the
material to take full advantage of hindsight. I tried not to do too much
damage to the light and spontaneous character of Sanath's original notes. 
I hope you find these notes useful, and I welcome comments or corrections!




\newpage
%Here are hyperlinks to the notes from each of the semesters.
%    \begin{enumerate}
%        \item The notes for 18.905 start at section \ref{905}.
%        \item The notes for 18.906 start at section \ref{906}.
%    \end{enumerate}

%\section*{Stuff to fix}
%\begin{itemize}
%    \item The original version of the notes used
%	\verb|\cc| to denote both a script C and the complex numbers.
%	Now, \verb|\cc| denotes a script C and \verb|\cC| denotes the complex
%	numbers.
%	Everytime you find this, edit it!
%    \item Be consistent about your use of $\Hom_\cc(C,D)$ and $\cc(C,D)$ for hom-sets.
%\end{itemize}



%\section*{Things to fix}
%\begin{itemize}
%    \item As of \today, all of Part II has been edited, with the exception of
%	\S \ref{basepoints}, \S \ref{section-serre-classes}, \S
%	\ref{mod-c-hurewicz}, \S \ref{dress-sseq}, \S \ref{leray-hirsch}, and
%	\S \ref{gysin-sequence}.
%    \item Part I remains to be edited.
%    \item The original version of the notes used \verb|\cc| to denote both a
%	script C and the complex numbers.  Now, \verb|\cc| denotes a script C
%	and \verb|\cC| denotes the complex numbers. This is a problem that
%	needs to be fixed everywhere.
%\end{itemize}
