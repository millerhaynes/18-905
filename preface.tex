\cleardoublepage
\section*{Preface}
Over the 2016--2017 academic year, I ran the standard algebraic topology 
sequence at MIT. The first semester deals with singular homology and 
cohomology, and Poicar\'e duality; the second builds up basic homotopy theory,
spectral sequences, and characteristic classes. 

I was lucky enough to have in the audience a student, Sanath Devalpurkar, 
who spontaneously decided to live\TeX the entire course. This resulted in 
a remarkably accurate record of what happened in the classroom -- right down
to random alarms ringing and embarassing mistakes on the blackboard. 
Sanath's \TeX forms the basis of these notes. 

My goal was to give a standard classical approach to these subjects. 
In the first semester, I tried to give an honest account of the relative cup
products needed in the proof of Poincar\'e duality. 


\newpage
%Here are hyperlinks to the notes from each of the semesters.
%    \begin{enumerate}
%        \item The notes for 18.905 start at section \ref{905}.
%        \item The notes for 18.906 start at section \ref{906}.
%    \end{enumerate}

%\section*{Stuff to fix}
%\begin{itemize}
%    \item The original version of the notes used
%	\verb|\cc| to denote both a script C and the complex numbers.
%	Now, \verb|\cc| denotes a script C and \verb|\cC| denotes the complex
%	numbers.
%	Everytime you find this, edit it!
%    \item Be consistent about your use of $\Hom_\cc(C,D)$ and $\cc(C,D)$ for hom-sets.
%\end{itemize}



%\section*{Things to fix}
%\begin{itemize}
%    \item As of \today, all of Part II has been edited, with the exception of
%	\S \ref{basepoints}, \S \ref{section-serre-classes}, \S
%	\ref{mod-c-hurewicz}, \S \ref{dress-sseq}, \S \ref{leray-hirsch}, and
%	\S \ref{gysin-sequence}.
%    \item Part I remains to be edited.
%    \item The original version of the notes used \verb|\cc| to denote both a
%	script C and the complex numbers.  Now, \verb|\cc| denotes a script C
%	and \verb|\cC| denotes the complex numbers. This is a problem that
%	needs to be fixed everywhere.
%\end{itemize}
